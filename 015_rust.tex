\section{Rust and its Borrow Checker}\label{sec:rust}

This section shows examples about the use of borrows in
Rust and how the borrow checker identifies illegal situations.

Rust deallocates the data structure owned by a variable
as soon as that variable goes out of scope. Consider for instance
the following function, where the \<Box::new(13)> is the allocation of
a data structure that wraps the integer $13$:

\begin{verbatim}
fn deallocate1() -> i32 { // accepted by the borrow checker
    let x = Box::new(13);
    return 17;
}
\end{verbatim}

\noindent
Local variable \<x> goes out of scope at the end of the function,
hence Rust deallocates the box there, automatically.

Assignments move the ownership of a value to their leftvalue. Consider
for instance the following function:

\begin{verbatim}
fn deallocate2() { // rejected by the borrow checker
    let x = Box::new(13);
    {
        let y = x;
    }
    println!("{}", x);
}
\end{verbatim}

\noindent
The ownership of the box moves from \<x> to \<y>. Since \<y> goes out of scope
at the end of the inner block, the box gets deallocated there, before the
\<println>. Consequently, the print statement is trying to print
already deallocated data\ie it is trying to access a dangling pointer.
Correctly, the borrow checker of Rust rejects the previous function.

Consider the following function now:

\begin{verbatim}
fn ok1() -> Box<i32> { // accepted by the borrow checker
    let x = Box::new(13);
    return x;
}
\end{verbatim}

\noindent
This time, the box owned by \<x> is assigned to the value returned by
the function, that acquires its ownership and hands it to the caller of
the function. Variable \<x> reaches the end of its scope at the end of the
function, but it owns no value there, hence Rust does not deallocate
anything inside \<ok1>.

Things become more complicated if borrows of data structures exist.
For instance, the following function tries to return a borrow of
a data structure that has been already deallocated:

\begin{verbatim}
fn dangling() -> &Box<i32> { // rejected by the borrow checker
    let i = Box::new(13);
    let result = &i;
    return result;
}
\end{verbatim}

\noindent
Local variable \<i> owns the box and goes out of scope at the end of the
function, where the box gets deallocated. Consequently, the return value of
the function is a dangling pointer. Correctly, Rust rejects this function.
Namely, its borrow checker, as formalized in~\cite{Pearce21}, computes the
following \emph{typing} (or \emph{type environment}) at the end of the function:
\[
\{\<i>\to\boxtype{\mathsf{int},\<result>\to\borrow\{\<i>\}}\}
\]
from where it is apparent that the function is trying to return a borrow
of a value that gets deallocated at the end of the function. Note that the borrow
checker allows one to
use types that are borrows of more leftvalues (not just variables),
such as $\borrow\{\<x>,\<**y>\}$.
Such types arise from the reconciliation of the typings at the end of
conditional statements (that we do not show in this paper).

Borrows are a sort of temporary ownership of a value. As a consequence,
that value can be modified only through the borrow, for the whole
duration of the borrow. Any other attempt to modify the value is rejected.
Consider for instance the following function:

\begin{verbatim}
fn writes_to_borrowed() { // rejected by the borrow checker
    let v = 13;
    let w = 17;
    let mut y = &v;
    let x = &y;
    y = &w;
    println!("{}{}{}{}", x, y, v, w);
}
\end{verbatim}

\noindent
Here, the \<y=\&w> statement is trying to modify the leftvalue \<y>
that, however, has been borrowed at the previous line. Correctly, the borrow
checker rejects this function. It computes the following typing
just before the \<y=\&w> statement:

\[
\{\<v>\to\mathsf{int},\<w>\to\mathsf{int},\<y>\to\borrow\{\<v>\},\<x>\to\borrow\{\<y>\}\}
\]

\noindent
from where it is apparent that \<y> is borrowed there. Therefore,
the subsequent assignment \<y=\&w> gets rejected.

Borrows in previous examples are immutable: the borrowed value can be read
from them, but cannot be modified from them.
Borrows can also be mutable, meaning that they allow one to modify the
borrowed value, with the dereference operator \<*>. In this sense,
a mutable borrow takes full responsibility about the borrowed value, for its
whole lifetime. When a mutable borrow to a value exists, that value cannot
be written \emph{nor read} from any other path. Consider for instance
the following function:

\begin{verbatim}
fn reads_from_mutably_borrowed() { // rejected by the borrow checker
    let mut z = 13;
    let y = &mut z;
    let x = z;
    println!("{}{}{}", x, y, z);
}
\end{verbatim}

\noindent
The statement \<x=z> tries to read \<z>, that has been mutably borrowed
at the previous line. Hence, the borrow checker rejects this function.
It computes the typing
\[
\{\<z>\to\mathsf{int},\<y>\to\mutborrow\{\<z>\}\}
\]
just before \<x=z>, from where it is apparent that \<z> is mutably borrowed there.

%For the same reason, a mutable borrow cannot coexist with any other borrow. Consider for instance
%the following function:
%
%\begin{verbatim}
%fn both_borrows() { // rejected by the borrow checker
%    let mut z = 13;
%    let b1 = &z;
%    let b2 = &mut z;
%    println!("{}{}{}", z, b1, b2);
%}
%\end{verbatim}
%
%\noindent
%The attempt to create a mutable borrow of \<z> fails because there is already a borrow of \<z>
%in scope (inside variable \<b1>). This function is consequently rejected by the borrow checker.

%Borrows are a \emph{deep} temporary ownership, in the sense that they take
%responsibility (temporarily) for the borrowed object and everything that is reachable from it.
%Consider for instance the following function:
%
%\begin{verbatim}
%fn both_borrows_complex() { // rejected by the borrow checker
%    let b = 13;
%    let x = Box::new(b);
%    let y = Box::new(x);
%    let mut z = Box::new(y);
%    let v = &**z;
%    let w = &mut z;
%    println!("{}{}", v, w);
%}
%\end{verbatim}
%
%\noindent
%The borrow checker rejects this function since it complains about
%\<z> being borrowed mutably by \<w> while it is also borrowed by \<v>.
%Actually, \<v> borrows \<**z>, but that is reachable from \<z>, hence
%the two borrows cannot coexist.
%The borrow checker computes the typing
%\[
%\{\<b>\to\mathsf{int},\<x>\to\boxtype{\mathsf{int}},\<y>\to\boxtype{\boxtype{\mathsf{int}}},
%\<z>\to\boxtype{\boxtype{\boxtype{\mathsf{int}}}},\<v>\to\borrow\{\<**z>\}\}
%\]
%just before the declaration of \<w>, from where it is apparent that \<**z> is
%already borrowed and hence \<z> is already borrowed as well and cannot be subsequently mutably borrowed.
