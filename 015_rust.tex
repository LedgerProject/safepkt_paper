\section{Rust and its Borrow Checker}\label{sec:rust}

This section illustrates various aspects Rust related to memory
allocation and borrowing, and provides an initial connection with
Featherweight Rust (FR).  A more detailed introduction to Rust can be
found elsewhere~\cite{RustBook,Rustonomicon}.

Rust deallocates the data owned by a variable as soon as that variable
goes out of scope. Consider the following, where the \<Box::new(13)>
allocates a new box on the heap which wraps the integer $13$:

\begin{lstlisting}
fn deallocate1() -> i32 {    // accepted by the borrow checker
    let x = Box::new(13);
    return 17;
}
\end{lstlisting}

\noindent
Local variable \lstinline{x} goes out of scope at the end of the
function, hence Rust deallocates the box there, automatically.
Assignments move the ownership of a value to their leftvalue. Consider
the following:

\begin{lstlisting}
fn deallocate2() {          // rejected by the borrow checker
    let x = Box::new(13);
    {
        let y = x;
    }
    println!("{}", x);
}
\end{lstlisting}

\noindent
The assignment moves ownership of the box from \lstinline{x} to
\lstinline{y}.  Since \lstinline{y} goes out of scope when the inner
block ends, the box is deallocated there. Consequently, the print
statement is trying to use deallocated data\ie it is trying to access
a dangling pointer.  Correctly, the borrow checker of Rust rejects
this.  Consider the following function now:

\begin{lstlisting}
fn ok1() -> Box<i32> {      // accepted by the borrow checker
    let x = Box::new(13);
    return x;
}
\end{lstlisting}

\noindent
Here, ownership of the box is transferred from \lstinline{x} to the
value, and subsequently to the caller of the function.  When variable
\lstinline{x} reaches the end of its scope it no longer owns a value
and, hence, Rust does not deallocate anything inside \<ok1>.

Things become more complicated if borrows of data structures exist.
For instance, the following function tries to return a borrow of
a data structure that has been already deallocated:

\begin{lstlisting}
fn dangling() -> &Box<i32> { // rejected by the borrow checker
    let i = Box::new(13);
    let result = &i;
    return result;
}
\end{lstlisting}

\noindent

Local variable \lstinline{i} owns the box and, when it goes out of
scope at the end of the function, the box is deallocated.  Variable
\lstinline{result} takes an {\em immutable borrow} of \lstinline{i}
(roughly a pointer to \lstinline{i} without ownership).  Thus, when
the box is deallocated, \lstinline{result} becomes a dangling pointer
which cannot safely be returned.  Again, Rust rejects this function.
Roughly, the borrow checker for FR~\cite{Pearce21} computes the
following \emph{typing} (or \emph{type environment}) at the end of the
function:
\[
\{i\to\boxtype{\mathsf{int},\<result>\to\borrow i}\}
\]

For simplicity, FR uses ${\tt int}$ to collectively represent integer
types in Rust (e.g. \lstinline{i32}, \lstinline{i64}, etc).  Likewise,
$\boxempty{\tt T}$ corresponds with \lstinline{Box<T>} and provides
the only form of dynamically allocated data in FR.  Finally, $\borrow
w$ (resp. $\mutborrow w$) where $w$ is a leftvalue is the type of an
immutable (resp. mutable) borrow.  Furthermore, since the borrow
checker allows arbitrary leftvalues here (i.e. not just variables), we
can have types such as $\borrow\mathtt{**}y$.

Borrows are a sort of temporary ownership of a value. As a consequence,
that value can be modified only through the borrow, for the whole
duration of the borrow. Any other attempt to modify the value is rejected.
Consider for instance the following function:

\begin{lstlisting}
fn writes_to_borrowed() { // rejected by the borrow checker
    let v = 13;
    let w = 17;
    let mut y = &v;
    let x = &y;
    y = &w;
    println!("{}{}{}{}", x, y, v, w);
}
\end{lstlisting}

\noindent
Here, the \<y=\&w> statement is trying to modify the leftvalue $y$
that, however, has been borrowed at the previous line. Correctly, the borrow
checker rejects this function. It computes the following typing
just before the \<y=\&w> statement:

\[
\{v\to\mathsf{int},w\to\mathsf{int},y\to\borrow v,x\to\borrow y\}
\]

\noindent
from where it is apparent that $y$ is borrowed there. Therefore,
the subsequent assignment \<y=\&w> gets rejected.

Borrows in previous examples are immutable: the borrowed value can be read
from them, but cannot be modified from them.
Borrows can also be mutable, meaning that they allow one to modify the
borrowed value, with the dereference operator \<*>. In this sense,
a mutable borrow takes full responsibility about the borrowed value, for its
whole lifetime. When a mutable borrow to a value exists, that value cannot
be written \emph{nor read} from any other path. Consider for instance
the following function:

\begin{lstlisting}
fn reads_from_mutably_borrowed() { // rejected by the borrow checker
    let mut z = 13;
    let y = &mut z;
    let x = z;
    println!("{}{}{}", x, y, z);
}
\end{lstlisting}

\noindent
The statement \<x=z> tries to read $z$, that has been mutably borrowed
at the previous line. Hence, the borrow checker rejects this function.
It computes the typing
\[
\{z\to\mathsf{int},y\to\mutborrow z\}
\]
just before \<x=z>, from where it is apparent that $z$ is mutably borrowed there.

%For the same reason, a mutable borrow cannot coexist with any other borrow. Consider for instance
%the following function:
%
%\begin{lstlisting}
%fn both_borrows() { // rejected by the borrow checker
%    let mut z = 13;
%    let b1 = &z;
%    let b2 = &mut z;
%    println!("{}{}{}", z, b1, b2);
%}
%\end{lstlisting}
%
%\noindent
%The attempt to create a mutable borrow of $z$ fails because there is already a borrow of $z$
%in scope (inside variable \<b1>). This function is consequently rejected by the borrow checker.

%Borrows are a \emph{deep} temporary ownership, in the sense that they take
%responsibility (temporarily) for the borrowed object and everything that is reachable from it.
%Consider for instance the following function:
%
%\begin{lstlisting}
%fn both_borrows_complex() { // rejected by the borrow checker
%    let b = 13;
%    let x = Box::new(b);
%    let y = Box::new(x);
%    let mut z = Box::new(y);
%    let v = &**z;
%    let w = &mut z;
%    println!("{}{}", v, w);
%}
%\end{lstlisting}
%
%\noindent
%The borrow checker rejects this function since it complains about
%$z$ being borrowed mutably by $w$ while it is also borrowed by $v$.
%Actually, $v$ borrows $\mathtt{**}z$, but that is reachable from $z$, hence
%the two borrows cannot coexist.
%The borrow checker computes the typing
%\[
%\{b\to\mathsf{int},x\to\boxtype{\mathsf{int}},y\to\boxtype{\boxtype{\mathsf{int}}},
%z\to\boxtype{\boxtype{\boxtype{\mathsf{int}}}},v\to\borrow\{\mathtt{**}z\}\}
%\]
%just before the declaration of $w$, from where it is apparent that $\mathtt{**}z$ is
%already borrowed and hence $z$ is already borrowed as well and cannot be subsequently mutably borrowed.
