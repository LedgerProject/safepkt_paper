\section{Preliminaries}\label{sec:featherweight}

This section gives a formal, simplified presentation of Featherweight
Rust (\FR/)~\cite{Pea21}.  This retains the key features of \FR/
relevant to our discussion but, for brevity, omits other aspects.
Roughly speaking, the main simplifications are:
%
\begin{itemize}

\item {\bf Compatibility}.  The original formulation of \FR/ supports
  a notion of {\em partial type}.  This allows the ``shadow'' of a
  variable's type to be retained in the environment after it has been
  moved, such that subsequent assignments can be checked for
  compatibility.  Since this is not important here, we reduce these
  shadow types to a single ``dangling'' type.
  
\item {\bf Borrows} The original formulation of \FR/ models borrows
  using {\em sets} of leftvalues.  Whilst strictly unnecessary, this
  allows \FR/ to be easily extended with control-flow constructs.
  Since this makes our presentation more complex without adding
  anything significant, we restrict borrows to a single leftvalue.

\item {\bf Misc.} We have transformed some definitions, originally
  given as typing rules in~\cite{Pea21}, to the definition of
  functions, such as functions $\mathsf{type}$ and $\mathsf{move}$
  later.  This makes them more compact and simplifies proofs involving
  them.
\end{itemize}

\begin{definition}[LVals]
  We assume a set of variables $\Vars$.  A context
  $\kappa\subseteq\Vars$ is a finite set of variables in scope.  The
  set $\Leftvalues_\kappa$ of \emph{leftvalues} over $\kappa$ is:
  \begin{displaymath}
    \mathsf{w} ::= x\;|\;\mathtt{*}\mathsf{w}, \text{where $x\in\kappa$}
  \end{displaymath}
  The \emph{root} of a leftvalue is then defined as:
  \begin{align*}
    \mathsf{root}(x) &= x\qquad\text{if $x\in\Vars$}\\
    \mathsf{root}(\mathtt{*}\mathsf{w}) &= \mathsf{root}(\mathsf{w}).
  \end{align*}
\end{definition}
\begin{definition}[Expressions]
  The set of \emph{expressions} \textsf{e} is defined as
  \begin{displaymath}
    \mathsf{e} ::= i\ |\ \mathsf{w}\ |\ \borrow\mathsf{w}\ |\ \mutborrow\mathsf{w}\ |\ \mathtt{box}\ \mathsf{e}
  \end{displaymath}
\end{definition}  
\begin{definition}[Terms]  
  We assume a set $\Lifetimes$ of \emph{lifetimes} $l$ which decorate
  blocks of code.  The set of \emph{terms} \textsf{t} is defined as (where $x\in\Vars$ and $l\in\Lifetimes$):
  \begin{displaymath}
    \mathsf{t} ::= \ \mathsf{w}=\mathsf{e}\;|\;\mathtt{let\ mut}\ x=\mathsf{e}\;|\;\{\;\mathsf{t}_1\;;\;\ldots\;;\;\mathsf{t}_n\;\}^l
  \end{displaymath}
\end{definition}
%
Intuitively, variables declared in a block with lifetime $l$ have
lifetime $l$ and are deallocated at the end of the block.  Lifetimes
are important for the borrow checker to ensure borrows do not outlive
their referents and become dangling.  The following illustrates a
simple (invalid) program:
\begin{displaymath}
{\tt \{\;let\;mut\;x=0;\;let\;mut\;p=\&x;\;\{\;let\;mut\;y=1;\;p=\&y;\;\}^m\;\}^l}
\end{displaymath}

This program creates a dangling reference when the inner block
completes and, hence, is rejected by the borrow checker.  The types
used in \FR/ are a simplification of those found in Rust, and include
only primitive types (such as $\mathsf{int}$) or structures
dynamically allocated in memory (collectively represented by a box),
but can also refer to a borrow or mutable borrow of a leftvalue.  This
is a simplification of the types from~\cite{Pea21}, that allow more
leftvalues in borrows. Since our simplified language has no
conditionals, there is no need for more than one leftvalue in borrows.

\begin{definition}[Types] The set of \emph{types} over a context $\kappa$ is defined as follows (where $\mathsf{w}\in\kappa$):
\begin{displaymath}
    \mathsf{T}_\kappa
    ::=\ \mathsf{int}\;|\;\borrow\mathsf{w}\;|\;\mutborrow\mathsf{w}\;|\;\boxtype{\mathsf{T}_\kappa}\;|\;\mathsf{dangling}
\end{displaymath}
\end{definition}

Here, type $\mathsf{dangling}$ is given to a variable whose value has
been \emph{moved}, that is, assigned to another owner.\footnote{This
  is a simplification of the $\mathsf{dangling}(T)$ type
  in~\cite{Pea21}, that embeds the \emph{ghost} type $T$ of a value
  that has been moved away.}  Consequently, the value exists but
cannot be accessed from that variable anymore.

\begin{definition}[Declared Types]
  The set of \emph{declared types}, $T^l$, over $\kappa$ associates
  types with lifetimes. We define $|T^l|=T$.
\end{definition}

Rust distinguishes types with {\em copy semantics} and types with {\em
  move semantics}.  Values whose type has copy semantics are copied
upon reading, while values whose type has move semantics are
\emph{moved} instead, in the sense that their original container loses
the ownership to the value. Only mutable borrows and dynamically
allocated data (i.e. boxes) have move semantics.

\begin{definition}[Copy and Move]\label{def:copy_move}
  Let $T\in\mathsf{T}_\kappa$. Then $T$ \emph{has move semantics}, and
  we write $\movetype(T)$, if and only if $T=\mutborrow\mathsf{w}$ or
  $T=\boxtype{T'}$ for some $T'$. In all other cases, $T$ \emph{has
    copy semantics}, and we write $\copytype(T)$.
\end{definition}

Another useful notion is that of \emph{full} types. They are types that do not contain
$\mathsf{dangling}$. This notion is important because, as we will see
in Sec.~\ref{sec:borrow_checker}, only values with full type can be borrowed in Rust.

\begin{definition}[Full type]\label{def:full}
  A type $T\in\mathsf{T}_\kappa$ is \emph{full}
  if and only if $\mathsf{dangling}$ does not occur inside $T$.
  We write it as $\mathsf{full}(T)$.
\end{definition}

We can now define the typings, or type environments, that is,
information about the types of the variables in scope at a given program point,
with their lifetime.

\begin{definition}[Typing]\label{def:typing}
  Given a context $\kappa$, a \emph{typing} $\tau$ over $\kappa$ is
  a map from each variable $v\in\kappa$ to a type $T$ and a lifetime $l$.
  We write this as $\tau(v)=T^l$.
\end{definition}

The types used in a typing can include borrows and mutable borrows.
The basic idea of the borrow checker is that the root of the
borrowed leftvalues (mutable or not) can only be used
in a restricted way~\cite{Pea21}.

\begin{definition}[Read/Write Prohibited]\label{def:prohibited}
  Let $\kappa$ be a context and $\tau$ a \emph{typing} over $\kappa$.
  Then $\mathsf{w}\in\Leftvalues_\kappa$ is
  \emph{read prohibited in $\tau$},
  written as $\mathsf{readProhibited}(\mathsf{w},\tau)$,
  if $\mathsf{root}(\mathsf{w})$ occurs in a mutable borrow inside $\tau$.
  Moreover, $\mathsf{w}$ is \emph{write prohibited in $\tau$}, written
  as $\mathsf{writeProhibited}(\mathsf{w},\tau)$,
  if $\mathsf{root}(\mathsf{w})$ occurs in a borrow or in a
  mutable borrow inside $\tau$.
\end{definition}

\noindent
It follows that $\mathsf{readProhibited}(\mathsf{w},\tau)$ entails
$\mathsf{writeProhibited}(\mathsf{w},\tau)$.

Def.~\ref{def:type} shows how a typing provides type and lifetime not just to variables,
but to leftvalues as well.
It is a translation\footnote{This definition was given as a type system in~\cite{Pea21}, while it is a recursive function here.} of
Def.~3.11 in~\cite{Pea21}.
It can be seen as a recursive algorithm for typing leftvalues
and, as such, it is heavily used in the borrow checker.
The algorithm queries the typing when the leftvalue is actually a variable,
and dereferences borrows and boxes when the leftvalue contains one or more
\<*> operations, further recurring in the case of borrows.
Types $\mathsf{int}$ and $\mathsf{dangling}$ cannot be dereferenced, hence the
algorithm fails on them.

\begin{definition}[Type of leftvalues]\label{def:type}
  Given a context $\kappa$, a typing $\tau$ over $\kappa$
  and $\mathsf{w}\in\Leftvalues_\kappa$, the partial function
  $\type(\mathsf{w},\tau)$ yields type and lifetime of $\mathsf{w}$ in $\tau$:
  \begin{align*}
    \type(x,\tau)&=\tau(x)\\
    \type(*\mathsf{w},\tau)&=\begin{cases}
    \text{undefined} & \text{if $\type(\mathsf{w},\tau)$ is undefined}\\
    \text{undefined} & \text{if $|\type(\mathsf{w},\tau)|=\mathsf{dangling}$}\\
    \text{undefined} & \text{if $|\type(\mathsf{w},\tau)|=\mathsf{int}$}\\
    \type(\mathsf{w}',\tau) & \text{if $|\type(\mathsf{w},\tau)|=\borrow\mathsf{w}'$}\\
    \type(\mathsf{w}',\tau) & \text{if $|\type(\mathsf{w},\tau)|=\mutborrow\mathsf{w}'$}\\
    T^l & \text{if $\type(\mathsf{w},\tau)=(\boxtype{T})^l$.}
    \end{cases}
  \end{align*}
\end{definition}

Def.~\ref{def:type} is clearly recursive, both on the structure of $\mathsf{w}$ and
on the leftvalues contained in the borrows or mutable borrows that occur in the typing.
In general, that recursion is not well-founded. In algorithmic terms, this means
that this algorithm for typing leftvalues might not terminate.
Consider for instance the typing $\{x\to\borrow\mathtt{*}x\}$:
the definition of $\type(\mathtt{*}x,\tau)$ ends in an infinite loop.
This example can be arbitrarily complicated, through the
use of more involved cycles that pass through more variables. As a consequence,
the natural question is to understand when the recursion in
Def.~\ref{def:type} is well-founded and if that is always the case when it is
used by the borrow checker of Rust.
