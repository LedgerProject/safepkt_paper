A distinguished feature of the Rust programming language is its
ability to deallocate dynamically-allocated data structures as soon as
they go out of scope, without relying on a garbage collector. At the
same time, Rust lets programmers create references, called
\emph{borrows}, to data structures. A static borrow checker enforces
that borrows can only be used in a controlled way, so that automatic
deallocation does not introduce dangling references.  Featherweight
Rust provides a formalisation for a subset of Rust where borrow
checking is encoded using flow typing~\cite{Pea21}.  However, we have
identified a source of non-termination within the calculus which
arises when typing environments contain cycles between variables.  In
fact, it turns out that well-typed programs cannot lead to such
environments --- but this is not immediately obvious from the
presentation.  This paper defines a simplificatoin of Featherweight
Rust, more amenable to formal proofs. Then it develops a sufficient
condition that forbids cycles and, hence, guarantees termination.
Furthermore, it proves that this condition is, in fact, maintained by
Featherweight Rust for well-typed programs.
