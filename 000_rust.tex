%
\documentclass{llncs}

\usepackage{url}
\usepackage{graphicx}
\usepackage[T1]{fontenc}
\usepackage[hidelinks]{hyperref}
\usepackage{pdfpages}
\usepackage{relsize}
\usepackage{tcolorbox}
\usepackage{amsmath,amssymb,amsfonts}
\usepackage{stmaryrd}

\usepackage{svg}
\newcommand{\orcid}[1]{\href{https://orcid.org/#1}{\includesvg[height = 2ex]{svg-inkscape/ORCID_iD}}}

\newcommand{\ie}{\textit{, ie.\ }}

\def\codesize{\smaller}
\def\<#1>{\codeid{#1}}
\newcommand{\codeid}[1]{\ifmmode{\mbox{\codesize\ttfamily{#1}}}\else{\codesize\ttfamily #1}\fi}

\newcommand{\borrow}{\&}
\newcommand{\mutborrow}{\<mut>}
\newcommand{\Vars}{\mathsf{Vars}}
\newcommand{\Locs}{\mathsf{Locs}}
\newcommand{\loc}{\mathsf{loc}}
\newcommand{\readlv}{\mathsf{read}}
\newcommand{\Lifetimes}{\mathsf{Lifetimes}}

\newcommand{\den}[2][]{\ensuremath{\llbracket #1\rrbracket{#2}}}

\usepackage{listings, xcolor}
\newcommand{\todo}[1]{\textcolor{red}{#1}}


\renewcommand{\UrlFont}{\ttfamily\codesize}

\definecolor{verylightgray}{rgb}{.97,.97,.97}

\lstdefinelanguage{Takamaka}{
        keywords=[1]{abstract, break, case, catch, class, continue, default, do
, else, false, finally, for, if, final, implements, extends, import, instanceof, interface, length, new, private, protected, public, return, super, switch, this, throw, true, try, while, var, null}, % generic keywords
        keywordstyle=[1]\color{blue}\bfseries,
        keywords=[2]{boolean, int, long, float, double, byte, short, char, void, enum}, % types; money and time units
        keywordstyle=[2]\color{teal}\bfseries,
        keywords=[3]{@Override,@View,@FromContract,@Payable}, % annotations
        keywordstyle=[3]\color{violet}\bfseries,
        identifierstyle=\color{black},
        sensitive=false,
        comment=[l]{//},
        morecomment=[s]{/*}{*/},
        commentstyle=\color{gray}\ttfamily,
        stringstyle=\color{red}\ttfamily,
        morestring=[b]',
        morestring=[b]"
}

\lstset{
        language=Takamaka,
        backgroundcolor=\color{verylightgray},
        extendedchars=true,
        basicstyle=\scriptsize\ttfamily,
        showstringspaces=false,
        showspaces=false,
        numbers=none,
        numberstyle=\scriptsize,
        numbersep=9pt,
        tabsize=2,
        breaklines=true,
        showtabs=false,
        captionpos=b
}

\lstdefinelanguage{JavaBytecode}{
        keywords=[1]{abstract, class, extends, public, private, protected}, % generic keywords
        keywordstyle=[1]\color{blue}\bfseries,
        keywords=[2]{boolean, int, long, float, double, byte, short, char, void}, % types; money and time units
        keywordstyle=[2]\color{teal}\bfseries,
        keywords=[3]{aload_0,aload_1,aload_2,aload_3,invokespecial,invokevirtual,checkcast,return}, % bytecodes
        keywordstyle=[3]\color{violet}\bfseries,
        identifierstyle=\color{black},
        sensitive=false,
        comment=[l]{//},
        morecomment=[s]{/*}{*/},
        commentstyle=\color{gray}\ttfamily,
        stringstyle=\color{red}\ttfamily,
        morestring=[b]',
        morestring=[b]"
}

\lstset{
        language=JavaBytecode,
        backgroundcolor=\color{verylightgray},
        extendedchars=true,
        basicstyle=\scriptsize\ttfamily,
        showstringspaces=false,
        showspaces=false,
        numbers=none,
        numberstyle=\scriptsize,
        numbersep=9pt,
        tabsize=2,
        breaklines=true,
        showtabs=false,
        captionpos=b
}

\begin{document}

\title{Semantics of the Rust Language for the Borrow Checker}
\titlerunning{Semantics of the Rust Language for the Borrow Checker}
\author{TBD}
\institute{TBD}

\maketitle

\begin{abstract}
  A distinguished feature of the Rust programming language
is its ability to deallocate dynamically-allocated
data structures as soon as they go out of scope, without relying on a garbage
collector. At the same time, Rust lets programmers create references,
called \emph{borrows}, to data structures. A borrow static checker enforces
that borrows can only be used in a controlled way, so that dangling references cannot arise.
The borrow checker has been formalized and proved correct~\cite{Pearce21}, although there
are situations when that checker might fail to terminate.
This paper provides a sufficient condition that guarantees termination for the borrow checker
of Rust and proves that that condition holds for programs that do not make use of \emph{reborrows}
(check at the end if this latter property actually holds).

  \keywords{TBD \and TBD}
\end{abstract}

\section{Introduction}\label{sec:introduction}

The Rust programming language is seeing widespread use in areas such
as system programming~\cite{ABGMMMS16,BBBPRR17,LCGPDL17}, blockchain
systems~\cite{HHHH18,NQ20}, smart contracts~\cite{Ash20,ZHCKHJJMS20}
and more~\cite{BHR18,AMPS19}.  A key feature of Rust is its ability to
automatically deallocate dynamically allocated data when it goes out
of scope. This is different from what happens in most other
programming languages, that either: require programmers free data
structures explicitly in code (such as C or C++); or, embed a garbage
collector that scans allocated memory and frees unreachable data (such
as Java, C\#, etc).  The former approach is error prone: programmers
might miss deallocations or might deallocate twice; the latter is
safer but more expensive and suboptimal: the garbage collector
consumes resources and unreachable data remains allocated until the
garbage collector is run.

In Rust, each data structure is \emph{owned} by a
variable~\cite{RustBook}. Once that variable goes out of scope, the
data structure it owns gets freed as well.  Data is divided into two
categories: that which can be {\em copied} (e.g. primitives); and that
which must be {\em moved} (e.g. mutable borrows).  For the latter,
assignments result in a transfer of ownership from rightvalue to
leftvalue.  In order to \emph{lend} a data structure as a parameter to
a function, \marginpar{Explain better}{Rust} allows the creation of
\emph{borrows}, that can be seen as pointers in traditional
programming languages.  However, since borrows are access paths to
data structures, the type checker of Rust must enforce strict rules on
their creation and lifetime. For instance, at each given instant, a
borrowed location cannot be written as long as the borrow exists.  The
implementation of the Rust compiler performs an additional
\emph{borrow checking} phase, after the traditional type checking is
over.  According to the developers of Rust, borrow checking is enough
to guarantee that automatic deallocation does not create dangling
pointers and multithreaded executions do not generate race conditions.

Featherweight Rust formalises a subset of Rust and includes a proof of
correctness for borrow checking~\cite{Pea21}.  In particular, borrow
checking is formalised as a flow-sensitive type system, whose types
include primitives (such as \<int>), dynamically allocated data
structures (collectively represented by a boxing operator) and borrows
of leftvalues, both for {\em reading} (immutable borrows) and {\em
  writing} (mutable borrows). The type system rules are given by
structural induction on the syntax of the Rust source code, and are
hence well-founded. However, they use, internally, a procedure to type
leftvalues. Since borrows include other leftvalues, we have discovered
this procedure may enter an infinite loop and, in such case, the
borrow checker would not terminate either.

\paragraph{Contribution.}

\marginpar{Simplified FR}{This} paper provides a sufficient condition
which ensures the borrow checker for Featherweight Rust
terminates~\cite{Pea21}.  Our insight is that, for well-typed
programs, this condition already holds for typing environments created
during borrow checking.  Hence, this is not a bug in Featherweight
Rust {\em per se}, but rather an important condition which was left
implicit.  Our approach shows that data structures are
\emph{linearizable} at run time and, hence, that our condition holds
for the specific kind of type environments the borrow checker builds
during execution.  This result is important in order to increase
confidence in the borrow checker of Rust.  Moreover, it provides a
notion of well-foundness for the recursion used in the borrow checker,
that future work can exploit in order to prove other properties by
induction.  For example, this is a necessary step towards a mechanical
proof of Featherweight Rust.

\section{Semantics}\label{sec:semantics}

\begin{definition}
  We assume there is a set of variables $\Vars$.
  A context $\kappa\subseteq\Vars$ is a finite set of variables in scope.
  The set $\Leftvalues_\kappa$ of \emph{leftvalues} over $\kappa$ is defined as:
  \begin{align*}
    \mathsf{w} ::= &\ x & \text{variable, with $x\in\kappa$}\\
    | &\ \mathtt{*}\mathsf{w} & \text{dereference.}
  \end{align*}
  The set of \emph{expressions} \textsf{e} is defined as
  \begin{align*}
    \mathsf{e} ::= &\ i & \text{integer}\\
    | &\ \mathsf{w} & \text{leftvalue}\\
    | &\ \borrow\mathsf{w} & \text{borrow}\\
    | &\ \mutborrow\mathsf{w} & \text{mutable borrow}\\
    | &\ \mathtt{box}\ \mathsf{e} & \text{heap allocation.}
  \end{align*}
  The sets of \emph{terms} \textsf{t} and \emph{blocks} \textsf{b}
  is defined as:
  \begin{align*}
    \mathsf{t} ::= &\ \mathsf{w}=\mathsf{e} & \text{assignment}\\
    | &\ \mathtt{let\ mut}\ x=\mathsf{e} & \text{declaration, with $x\in\Vars$}\\
    | &\ \{\mathsf{t_1};\ldots;\mathsf{t_n}\}^l & \text{block, with $n\ge 0$ and $l$ is a \emph{lifetime}.}
  \end{align*}
\end{definition}

\begin{example}\label{ex:program}
  The following valid program $P_1$ consists of a block that contains an inner block.
  \[
    \{
      \mathtt{let\ mut}\ x = \mathtt{box}\ 0;\
      \{\mathtt{let\ mut}\ y = \mutborrow x;\
      \mathtt{*}y = \mathtt{box}\ 1\}^l;\
      \mathtt{let\ mut}\ z = x
    \}^m
  \]
  \qed
\end{example}

The $*$ operator is called \emph{dereference}. In general, dereference means to access the
value bound to a reference. In Rust, borrows, mutable borrows and boxes are all references.
Hence, dereference can be used to read the value bound to a borrow, as in
\[
\{
\mathtt{let\ mut}\ x = 13;\
\mathtt{let\ mut}\ y = \borrow x;\
\mathtt{let\ mut}\ z = \mathtt{*}y
\}^l
\]
that reads the value bound to the address of $x$, hence copying $13$ into $z$.
Dereference can also be used to access the value bound to a mutable borrow, as in
\[
\{
\mathtt{let\ mut}\ x = 13;\
\mathtt{let\ mut}\ y = \mutborrow x;\
\mathtt{let\ mut}\ z = \mathtt{*}y
\}^l
\]
that, again, copies $13$ into $z$.
By using mutable borrows, one can also \emph{mutate} the referenced value, as in
\[
\{
\mathtt{let\ mut}\ x = \mathtt{box}\ 0;\
\mathtt{let\ mut}\ y = \mutborrow x;\
\mathtt{*}y = \mathtt{box}\ 1
\}^l
\]
that modifies $x$, so that it ends up being bound to a box containing $1$. Since $y$ holds a mutable borrow,
it is legal to mutate the value bound to it, which would not be the case with a non-mutable borrow.
Finally, the $*$ operator can be used to dereference a box, as in
\[
\{
\mathtt{let\ mut}\ x = \mathtt{box}\ 13;\
\mathtt{let\ mut}\ y = *x
\}^l
\]
that binds $y$ to $13$. This works for writing into a box as well:
\[
\{
\mathtt{let\ mut}\ x = \mathtt{box}\ 13;\
\mathtt{let\ mut}\ *x = 17
\}^l
\]
replaces $13$ with $17$ inside the box in $x$.

\begin{definition}[Root of leftvalues]\label{def:root}
  The \emph{root} of a leftvalue is the starting variable to which the dereferences of the
  leftvalue are applied. Namely, we define
  \begin{align*}
    \mathsf{root}(x) &= x\qquad\text{if $x\in\Vars$}\\
    \mathsf{root}(\mathtt{*}\mathsf{w}) &= \mathsf{root}(\mathsf{w}).
  \end{align*}
\end{definition}

\begin{definition}[Types]
  The set of \emph{types} over a context $\kappa$ is
  \begin{align*}
    \mathsf{T}_\kappa ::=&\ \mathsf{int} & \text{integer}\\
    | &\ \borrow\{\mathsf{w}_1,\ldots,\mathsf{w}_n\} & \text{borrow}\\
    | &\ \mutborrow\{\mathsf{w}_1,\ldots,\mathsf{w}_n\} & \text{mutable borrow}\\
    | &\ \boxtype{\mathsf{T}_\kappa} & \text{box}\\
    | &\ \mathsf{dangling} & \text{dangling}
  \end{align*}
  where $n\ge 1$ and the $\mathsf{w}_i$, with $1\le i\le n$, are leftvalues
  using only variables in $\dom(\kappa)$. This set is ordered by $\sqsubseteq$, defined as
  \begin{align*}
    \mathsf{int}&\sqsubseteq\mathsf{int}\\
    \borrow\{\mathsf{w}_1,\ldots,\mathsf{w}_n\} & \sqsubseteq
    \borrow\{\mathsf{w}'_1,\ldots,\mathsf{w}'_{n'}\} \quad\text{iff $\{\mathsf{w}_1,\ldots,\mathsf{w}_n\}\subseteq\{\mathsf{w}'_1,\ldots,\mathsf{w}'_{n'}\}$}\\
    \mutborrow\{\mathsf{w}_1,\ldots,\mathsf{w}_n\} & \sqsubseteq
    \mutborrow\{\mathsf{w}'_1,\ldots,\mathsf{w}'_{n'}\} \quad\text{iff $\{\mathsf{w}_1,\ldots,\mathsf{w}_n\}\subseteq\{\mathsf{w}'_1,\ldots,\mathsf{w}'_{n'}\}$}\\
    \boxtype{t_1}&\sqsubseteq\boxtype{t_2} \quad\text{iff $t_1\sqsubseteq t_2$}\\
    \mathsf{dangling}&\sqsubseteq\mathsf{dangling.}
  \end{align*}
\end{definition}

\begin{lemma}
  \label{lemma:partial-order-types}
  $(\mathsf{T}_\kappa,\sqsubseteq)$ is a partially ordered set.
\end{lemma}
\begin{proof}
  We proceed by structural induction.
  \begin{itemize}
    \item (Reflexivity)
    \begin{itemize}
      \item (Base) For any type $t$ that is not a box we have
      $t\sqsubseteq t$.
      \item (Induction step) If $t\sqsubseteq t$ for a type $t$ then
      $\boxtype{t}\sqsubseteq \boxtype{t}$.
    \end{itemize}
    %
    \item (Antisymmetry) Suppose that $t_1 \sqsubseteq t_2$ and
    $t_2 \sqsubseteq t_1$.
    \begin{itemize}
      \item (Base: $t_1$ or $t_2$ is not a box)
      If $t_1$ or $t_2$ is $\mathsf{int}$, then so is the other
      because $\mathsf{int}$ can only be greater than itself, and so $t_1 = t_2$.
      Similarly if $t_1$ or $t_2$ is $\mathsf{dangling}$.
      If $t_1$ or $t_2$ is a borrow, then so is the other because
      a borrow can only precede a borrow, and so we have $t_1 = t_2$
      by antisymmetry of $\subseteq$.
      Similarly if $t_1$ or $t_2$ is a mutable borrow.
      \item (Induction step) Suppose that $t_1=\boxtype{t'_1}$ and
      $t_2=\boxtype{t'_2}$ for types $t'_1,t'_2$ such that
      $(t'_1 \sqsubseteq t'_2 \land t'_2 \sqsubseteq t'_1) \Rightarrow t'_1 = t'_2$.
      As $t_1 \sqsubseteq t_2$ and $t_2 \sqsubseteq t_1$, we have
      $t'_1 \sqsubseteq t'_2$ and $t'_2 \sqsubseteq t'_1$, so
      $t'_1 = t'_2$, hence $t_1 = t_2$.
    \end{itemize}
    %
    \item (Transitivity) Suppose that $t_1 \sqsubseteq t_2$ and $t_2 \sqsubseteq t_3$.
    \begin{itemize}
      \item (Base: $t_1$, $t_2$ or $t_3$ is not a box)
      If $t_1$, $t_2$ or $t_3$ is $\mathsf{int}$, we necessarily have
      $t_1=t_2=t_3=\mathsf{int}$ because $\mathsf{int}$ is only related to
      itself, and so $t_1\sqsubseteq t_3$. Similarly if
      $t_1$, $t_2$ or $t_3$ is $\mathsf{dangling}$.
      If $t_1$, $t_2$ or $t_3$ is a borrow, then so are the others because
      a borrow is only related to a borrow, and so we have
      $t_1\sqsubseteq t_3$ by transitivity of $\subseteq$.
      Similarly if $t_1$, $t_2$ or $t_3$ is a mutable borrow.
      \item (Induction step) Suppose that $t_1=\boxtype{t'_1}$,
      $t_2=\boxtype{t'_2}$ and $t_3=\boxtype{t'_3}$ for types
      $t'_1,t'_2,t'_3$ such that
      $(t'_1 \sqsubseteq t'_2 \land t'_2 \sqsubseteq t'_3) \Rightarrow
      t'_1 \sqsubseteq t'_3$.
      As $t_1 \sqsubseteq t_2$ and $t_2 \sqsubseteq t_3$, we have
      $t'_1 \sqsubseteq t'_2$ and $t'_2 \sqsubseteq t'_3$, so
      $t'_1 \sqsubseteq t'_3$, hence $t_1 \sqsubseteq t_3$.
    \end{itemize}
  \end{itemize}
  \qed
\end{proof}

\begin{definition}
  The $\sqcap$ operator over $\mathsf{T}_\kappa$ is defined as
  \begin{align*}
    \mathsf{int}\sqcap\mathsf{int} &= \mathsf{int}\\
    \borrow\{\mathsf{w}_1,\ldots,\mathsf{w}_n\} \sqcap \borrow\{\mathsf{w}'_1,\ldots,\mathsf{w}'_{n'}\} &= \borrow(\{\mathsf{w}_1,\ldots,\mathsf{w}_n\}\cap\{\mathsf{w}'_1,\ldots,\mathsf{w}'_{n'}\})\\
    \mutborrow\{\mathsf{w}_1,\ldots,\mathsf{w}_n\} \sqcap \mutborrow\{\mathsf{w}'_1,\ldots,\mathsf{w}'_{n'}\} &= \mutborrow(\{\mathsf{w}_1,\ldots,\mathsf{w}_n\}\cap\{\mathsf{w}'_1,\ldots,\mathsf{w}'_{n'}\})\\
    (\boxtype{t_1}) \sqcap (\boxtype{t_2}) &=\boxtype{(t_1\sqcap t_2)}\quad\text{if $t_1\sqcap t_2$ is defined}\\
    \mathsf{dangling}\sqcap\mathsf{dangling} &= \mathsf{dangling}.
  \end{align*}
  In all other cases, $t_1\sqcap t_2$ is undefined.
\end{definition}

\begin{definition}
  The $\sqcup$ operator over $\mathsf{T}_\kappa$ is defined as
  \begin{align*}
    \mathsf{int}\sqcup\mathsf{int} &= \mathsf{int}\\
    \borrow\{\mathsf{w}_1,\ldots,\mathsf{w}_n\} \sqcup \borrow\{\mathsf{w}'_1,\ldots,\mathsf{w}'_{n'}\} &= \borrow(\{\mathsf{w}_1,\ldots,\mathsf{w}_n\}\cup\{\mathsf{w}'_1,\ldots,\mathsf{w}'_{n'}\})\\
    \mutborrow\{\mathsf{w}_1,\ldots,\mathsf{w}_n\} \sqcup \mutborrow\{\mathsf{w}'_1,\ldots,\mathsf{w}'_{n'}\} &= \mutborrow(\{\mathsf{w}_1,\ldots,\mathsf{w}_n\}\cup\{\mathsf{w}'_1,\ldots,\mathsf{w}'_{n'}\})\\
    (\boxtype{t_1}) \sqcup (\boxtype{t_2}) &=\boxtype{(t_1\sqcup t_2)}\quad\text{if $t_1\sqcup t_2$ is defined}\\
    \mathsf{dangling}\sqcup\mathsf{dangling} &= \mathsf{dangling}.
  \end{align*}
  In all other cases, $t_1\sqcup t_2$ is undefined.
\end{definition}

\begin{lemma}\label{lemma:glb-type}
  Let $I\subseteq\mathbb{N}$ and $\{t_i\}_{i\in I}\subseteq\mathsf{T}_\kappa$.
  Then, if $\sqcap_{i\in I}t_i$ is defined, it is the greatest lower bound of $\{t_i\}_{i\in I}$.
\end{lemma}
\begin{proof}
  Assume that $\mu_I = \sqcap_{i\in I}t_i$ is defined.
  We proceed by structural induction.
  \begin{itemize}
    \item (Base: there is a $t_k$ which is not a box)
    If $t_k=\mathsf{int}$ then, as $\mu_I$ is defined, for all $i\in I$
    we have $t_i=\mathsf{int}$. Therefore, $\mu_I=\mathsf{int}$
    and so $\mu_I$ is a lower bound of $\{t_i\}_{i\in I}$.
    Similarly if $t_k=\mathsf{dangling}$.
    If $t_k$ is a borrow then, for all $i\in I$, $t_i$ is a borrow, and so
    $\mu_I$ is a borrow which results from the intersection of the sets of
    leftvalues of all the $t_i$'s. Hence, $\mu_I$ is a lower bound of
    $\{t_i\}_{i\in I}$.
    Similarly if $t_k$ is a mutable borrow.

    Now, suppose that $\mu'_I$ is also a lower bound of $\{t_i\}_{i\in I}$.
    If $\mu'_I=\mathsf{int}$ then each $t_i$ is $\mathsf{int}$, so
    $\mu_I=\mathsf{int}$ and we have $\mu'_I\sqsubseteq\mu_I$.
    Similarly if $\mu'_I=\mathsf{dangling}$.
    If $\mu'_I=\borrow b$, then for all $i\in I$ we have $t_i = \borrow b_i$
    and $b \subseteq b_i$. Hence, $b \subseteq \cap_{i\in I}b_i$ and so
    $\borrow b \sqsubseteq \borrow\cap_{i\in I}b_i$. Consequently, we have
    $\mu'_I\sqsubseteq\mu_I$ because $\mu_I=\borrow\cap_{i\in I}b_i$ in this case.
    Similarly if $\mu'_I$ is a mutable borrow.
    %
    \item (Induction step) Suppose that for all $i\in I$ we have
    $t_i = \boxtype{t'_i}$. Since $\mu_I$ is defined, also
    $\sqcap_{i\in I}t'_i$ is defined and, by inductive hypothesis, it must be the greatest
    lower bound of $\{t'_i\}_{i\in I}$. Then,
    $\mu_I = \boxtype{(\sqcap_{i\in I}t'_i)}$ is a lower bound
    of $\{t_i\}_{i\in I}$. Suppose that $\mu'_I$ is also a lower bound
    of $\{t_i\}_{i\in I}$. Then, $\mu'_I=\boxtype{t}$ with
    $t\sqsubseteq t'_i$ for all $i\in I$, and so $t$ is a lower bound
    of $\{t'_i\}_{i\in I}$; hence, we have $t\sqsubseteq \sqcap_{i\in I}t'_i$
    because $\sqcap_{i\in I}t'_i$ is the greatest lower bound and,
    consequently, $\mu'_I\sqsubseteq\mu_I$.
  \end{itemize}
  \qed
\end{proof}

\begin{lemma}\label{lemma:technical-type}
  Let $t\in\mathsf{T}_\kappa$. For any $t_1,t_2\in\mathsf{T}_\kappa$,
  if $t\sqsubseteq t_1$ and $t\sqsubseteq t_2$ then $t_1\sqcap t_2$ is defined.
\end{lemma}
\begin{proof}
  We proceed by structural induction on $t$.
  \begin{itemize}
    \item (Base: $t$ is not a box)
    Let $t_1,t_2\in\mathsf{T}_\kappa$ with $t\sqsubseteq t_1$ and $t\sqsubseteq t_2$.
    If $t$ is $\mathsf{int}$ (resp. a borrow, a mutable borrow or $\mathsf{dangling}$)
    then, necessarily, $t_1$ and $t_2$ are $\mathsf{int}$ (resp. a borrow,
    a mutable borrow or $\mathsf{dangling}$), hence $t_1\sqcap t_2$ is defined.
    %
    \item (Induction step) Suppose that $t = \boxtype{t'}$.
    Let $t_1,t_2\in\mathsf{T}_\kappa$ with $t\sqsubseteq t_1$ and $t\sqsubseteq t_2$.
    Then, necessarily, $t_1 = \boxtype{t'_1}$ and $t_2 = \boxtype{t'_2}$
    with $t'\sqsubseteq t'_1$ and $t'\sqsubseteq t'_2$. By induction
    hypothesis, $t'_1\sqcap t'_2$ is defined, so
    $(\boxtype{t'_1}) \sqcap (\boxtype{t'_2}) = t_1 \sqcap t_2$ is defined.
    \qed
  \end{itemize}
\end{proof}

\begin{definition}[Typing]\label{def:typing}
  Given a context $\kappa$, a \emph{typing} $\tau$ over $\kappa$ is
  a map from the variables in $\kappa$ % $\tau$
  to types.
\end{definition}

A typing provides types to variables and can be extended in order to type
leftvalues as well, as shown below. Def.~\ref{def:type} is a translation of
Def.~3.11 from~\cite{Pearce21}.

\begin{definition}[Type of leftvalues]\label{def:type}
  Given a context $\kappa$, a typing $\tau$ over $\kappa$
  and $\mathsf{w}\in\Leftvalues_\kappa$, the partial function
  $\type(\mathsf{w},\tau)$ yields the type of $\mathsf{w}$ in $\tau$:
  \begin{align*}
    \type(x,\tau)&=\tau(x)\\
    \type(*\mathsf{w},\tau)&=\begin{cases}
    \text{undefined} & \text{if $\type(\mathsf{w},\tau)$ is undefined}\\
    \text{undefined} & \text{if $\type(\mathsf{w},\tau)=\mathsf{dangling}$}\\
    \text{undefined} & \text{if $\type(\mathsf{w},\tau)=\mathsf{int}$}\\
    \sqcup_{1\le i\le n}\type(\mathsf{w}_i,\tau) & \text{if $\type(\mathsf{w},\tau)=\borrow\{\mathsf{w}_1,\ldots,\mathsf{w}_n\}$}\\
    \sqcup_{1\le i\le n}\type(\mathsf{w}_i,\tau) & \text{if $\type(\mathsf{w},\tau)=\mutborrow\{\mathsf{w}_1,\ldots,\mathsf{w}_n\}$}\\
    t & \text{if $\type(\mathsf{w},\tau)=\boxtype{t}$}\\
    \text{undefined} & \text{otherwise.}
    \end{cases}
  \end{align*}
\end{definition}

\noindent
Note that $\type$ recurs in a way that, in general,
is not well-founded, for the cases of $\borrow$ and $\borrow\mathtt{mut}$.
Moreover, the $\sqcup$, in those two cases, might not be defined.
As hinted in~\cite{Pearce21}, that corresponds to a situation when the program
is not well typed, in the traditional sense of type checking.

\begin{example}\label{ex:type_divergence}
  Consider the following typing:
  \[
  \{x\to\borrow\{\mathtt{*}x\}\}.
  \]
  Then the definition of $\type(\mathtt{*}x,\tau)$ ends in an infinite loop.
\end{example}

\noindent
Ex.~\ref{ex:type_divergence} can be arbitrarily complicated, through the
use of more involved cycles that pass through more variables. As a consequence,
the natural question is to understand when Def.~\ref{def:type} is well-founded.
For that, we start by defining a concept of dependencies between leftvalues.

\begin{definition}[Dependencies between leftvalues]\label{def:dependencies}
  Given a context $\kappa$ and a typing $\tau$ over $\kappa$, the \emph{dependencies between leftvalues}
  induced by $\tau$ are the relation $\gg$ defined as
  \[
  \mathsf{closure}\left(\{\mathtt{*}\mathsf{w}\gg\mathsf{w}\mid\mathsf{w}\in\Leftvalues_\kappa\}
  \cup\bigcup\limits_{x\in\kappa}\mathsf{dependencies}(x,\tau(x))\right)
  \]
  where
  \begin{align*}
    \mathsf{dependencies}(\mathsf{w},\mathsf{int})&=\varnothing\\
    \mathsf{dependencies}(\mathsf{w},\mathsf{dangling})&=\varnothing\\
    \mathsf{dependencies}(\mathsf{w},\boxtype{t})&=\mathsf{dependencies}(\mathtt{*}\mathsf{w},t)\\
    \mathsf{dependencies}(\mathsf{w},\borrow\{\mathsf{w}_1,\ldots,\mathsf{w}_n\})&=\{\mathtt{*}\mathsf{w}\gg\mathsf{w}_i\mid 1\le i\le n\}\\
    \mathsf{dependencies}(\mathsf{w},\mutborrow\{\mathsf{w}_1,\ldots,\mathsf{w}_n\})&=\{\mathtt{*}\mathsf{w}\gg\mathsf{w}_i\mid 1\le i\le n\}
  \end{align*}
  and
  \[
  \mathsf{closure}(R)=R\cup\left\{\underbrace{\mathtt{*}\cdots\mathtt{*}}_{n}\mathsf{w}_1\gg\mathsf{w}_3\left|
  \begin{array}{l}
    \mathsf{w}_1\gg\mathsf{w}_2\in R,\ \underbrace{\mathtt{*}\cdots\mathtt{*}}_{n\ge 0}\mathsf{w}_2\gg\mathsf{w}_3\in R\\
    \text{and $\mathsf{w_3}$ is a borrow that occurs in $\tau$}
  \end{array}\right.\right\}.
  \]
\end{definition}

\noindent
Note that the closure in Def.~\ref{def:dependencies} makes $\gg$ transitive.

Prop.~\ref{prop:acyclicity} proves that the dependencies between leftvalues
model the pattern of the recursion in Def.~\ref{def:type}.

\begin{proposition}\label{prop:acyclicity}
  Given a context $\kappa$ and a typing $\tau$ over $\kappa$
  such that the dependencies between leftvalues induced by $\tau$ are acyclical,
  the recursion used in the definition of function $\type$ is well-founded and consistent with $\gg$.
\end{proposition}
\begin{proof}
  First we prove that $\gg$ is a well-founded relation. Assume the contrary. Then
  there is an infinite sequence of leftvalues
  $\mathsf{w}_1\gg\mathsf{w}_2\gg\cdots\gg\mathsf{w}_n\gg\cdots$.
  Since $\mathsf{w}\gg\mathsf{w}'\in\mathsf{dependencies}(x,t)$ entails that
  $\mathsf{w}'$ is one of the leftvalues that occur in the borrows of $\tau$ and
  since there is only a finite number of such $\mathsf{w}'$, the acyclicity of $\gg$ entails that
  there must be a $\mathsf{w}_k=\underbrace{\mathtt{*}\cdots\mathtt{*}}_{\text{$n$}}x$
  such that the subsequent $\mathsf{w}_i$, $i>k$,
  are not leftvalues that occur in the borrows in $\tau$. By Def.~\ref{def:dependencies},
  it can only be $\mathsf{w}_{k+i}=\underbrace{\mathtt{*}\cdots\mathtt{*}}_{\text{$\le n-i$}}x$
  and consequently the length of the sequence is $k+n$ at most, impossible since we assumed that
  it was infinite.

  We now prove that
  \begin{enumerate}
  \item for every $\mathsf{w}\in\Leftvalues_\kappa$, the
    recursive uses $\type(\mathsf{w}',\tau)$ that occur for the definition
    of $\type(\mathsf{w},\tau)$ are such that $\mathsf{w}\gg\mathsf{w}'$;
  \item when $\type(\mathsf{w},\tau)=\borrow\{\mathsf{w}_1,\ldots,\mathsf{w}_n\}$
    or $\type(\mathsf{w},\tau)=\mutborrow\{\mathsf{w}_1,\ldots,\mathsf{w}_n\}$
    then $\mathtt{*}\mathsf{w}\gg\mathsf{w}_i$ for every $1\le i\le n$.
  \end{enumerate}
  Note that (1) by itself means that
  the recursion used in the
  definition of function $\type$ is well-founded and consistent with $\gg$, but we will also
  need (2) in order to prove (1).
  We prove (1) and (2) by induction on $\mathsf{w}$ with respect to $\gg$.

  \begin{itemize}
  \item (Base case)
    If $\mathsf{w}$
    has no $\mathsf{w}'$ such that $\mathsf{w}\gg\mathsf{w}'$, then it must be $\mathsf{w}\in\Vars$.
    Hence there are no recursive uses of $\type$ in the definition of
    $\type(\mathsf{w},\tau)$ and (1) holds. Moreover, in this case
    $\type(\mathsf{w},\tau)=\tau(\mathsf{w})$ and if
    $\tau(\mathsf{w})=\borrow\{\mathsf{w}_1,\ldots,\mathsf{w}_n\}$ by Def.~\ref{def:dependencies}
    we conclude that $\mathtt{*}\mathsf{w}\gg\mathsf{w}_i$ for every $1\le i\le n$, hence (2) holds.
    Similarly when $\tau(\mathsf{w})=\mutborrow\{\mathsf{w}_1,\ldots,\mathsf{w}_n\}$.
  \item (Inductice case)
    Assume now that (1) and (2) hold for $\mathsf{w}$. Consider
    the recursive uses $\type(\mathsf{w}',\tau)$ in the definition of
    $\type(\mathtt{*}\mathsf{w},\tau)$. One such recursive use is
    $\type(\mathsf{w},\tau)$ and $\mathtt{*}\mathsf{w}\gg\mathsf{w}$.
    Others are inside the computation of $\type(\mathsf{w},\tau)$ and by inductive hypothesis
    (1) holds, that is, they occur on $\mathsf{w}'$ such that $\mathsf{w}\gg\mathsf{w}'$.
    Hence $\mathtt{*}\mathsf{w}\gg\mathsf{w}\gg\mathsf{w}'$ and by transitivity
    $\mathtt{*}\mathsf{w}\gg\mathsf{w}'$. Finally, there are recursive uses
    when $\type(\mathsf{w},\tau)=\borrow\{\mathsf{w}_1,\ldots,\mathsf{w}_n\}$
    or when $\type(\mathsf{w},\tau)=\mutborrow\{\mathsf{w}_1,\ldots,\mathsf{w}_n\}$
    namely, uses of
    $\type(\mathsf{w}_i,\tau)$ with $1\le i\le n$. By inductive hypothesis, we know that
    (2) holds for $\mathsf{w}$, that is, $\mathtt{*}\mathsf{w}\gg\mathsf{w}_i$ for every
    $1\le i\le n$. This concludes the inductive case for the proof of (1).
    Let us prove (2) for $\mathtt{*}\mathsf{w}$ now. Assume then that
    $\type(\mathtt{*}\mathsf{w},\tau)=\borrow\{\mathsf{w}_1,\ldots,\mathsf{w}_n\}$ (the case
    $\type(\mathtt{*}\mathsf{w},\tau)=\mutborrow\{\mathsf{w}_1,\ldots,\mathsf{w}_n\}$ is similar).
    By Def.~\ref{def:type}, there are two possibilities:
    \begin{itemize}
    \item $\mathtt{*}\mathsf{w}=\underbrace{\mathtt{*}\cdots\mathtt{*}}_{\text{$m+1$}}x$
      and $\tau(x)=\underbrace{\boxempty\cdots\boxempty}_{\text{$m+1$}}
      \borrow\{\mathsf{w}_1,\ldots,\mathsf{w}_n\}$
      for some $m\ge 0$, where $x=\mathsf{root}(\mathsf{w})$.
      By Def.~\ref{def:dependencies}, the relation $\gg$ includes
      \begin{align*}
        \mathsf{dependencies}(x,\tau(x))&=\mathsf{dependencies}(x,\underbrace{\boxempty\cdots\boxempty}_{\text{$m+1$}}\borrow\{\mathsf{w}_1,\ldots,\mathsf{w}_n\})\\
        &=\mathsf{dependencies}(\underbrace{\mathtt{*}\cdots\mathtt{*}}_{\text{$m+1$}}x,
        \borrow\{\mathsf{w}_1,\ldots,\mathsf{w}_n\})\\
        &=\mathsf{dependencies}(\mathtt{*}\mathsf{w},\borrow\{\mathsf{w}_1,\ldots,\mathsf{w}_n\})\\
        &=\{\mathtt{**}\mathsf{w}\gg\mathsf{w}_i\mid 1\le i\le n\}
      \end{align*}
      and (2) holds for $\mathtt{*}\mathsf{w}$.
    \item $\borrow\{\mathsf{w}_1,\ldots,\mathsf{w}_n\}=\sqcup_{1\le i\le n'}\type(\mathsf{w}_i',\tau)$
      with $\type(\mathsf{w},\tau)=\borrow\{\mathsf{w}'_1,\ldots,\mathsf{w}'_{n'}\}$.
      The only possibility is that $\type(\mathsf{w}_j',\tau)=\borrow W_j$ for every $1\le j\le n'$,
      with $\{\mathsf{w}_1,\ldots,\mathsf{w}_n\}=\cup_{1\le j\le n'}W_i$.
      By inductive hypothesis of (2), we know that
      $\mathtt{*}\mathsf{w}\gg\mathsf{w}_j'$ for every $1\le j\le n'$ and that
      $\mathtt{*}\mathsf{w}_j'\gg\mathsf{w}''$ for every $\mathsf{w}''\in W_j$ and
      every $1\le j\le n'$. Note that $\mathsf{w}''$ is one of the leftvalues that occur
      in the borrows of $\tau$, by Def.~\ref{def:type}.
      By closure (Def.~\ref{def:dependencies}), we conclude that
      $\mathtt{**}\mathsf{w}\gg\mathsf{w}''$ for every $\mathsf{w}''\in W_j$ and
      every $1\le j\le n'$. Since each $\mathsf{w}_i$ belongs to some $W_j$, we conclude that
      $\mathtt{**}\mathsf{w}\gg\mathsf{w}_i$ for every $1\le i\le n$ and (2) holds for $\mathtt{*}\mathsf{w}$.
    \end{itemize}
  \end{itemize}
\end{proof}


\section{Conclusion}\label{sec:conclusion} 




\bibliographystyle{plain}
\bibliography{biblio}

\end{document}
