%
\documentclass{llncs}

\usepackage{url}
\usepackage{graphicx}
\usepackage[T1]{fontenc}
\usepackage[hidelinks]{hyperref}
\usepackage{pdfpages}
\usepackage{relsize}
\usepackage{tcolorbox}
\usepackage{amsmath,amssymb,amsfonts}
\usepackage{stmaryrd}
\usepackage{sidenotes}
\usepackage{xcolor}
\usepackage{ulem} % remove at the end

\usepackage{svg}
\newcommand{\orcid}[1]{\href{https://orcid.org/#1}{\includesvg[height = 2ex]{svg-inkscape/ORCID_iD}}}

\newcommand{\ie}{\textit{, ie.\ }}
\newcommand{\wrt}{\textit{wrt.\ }}
\newcommand{\issue}[2]{{\color{red}{#1}\sidenote{#2}}}

\def\codesize{\smaller}
\def\<#1>{\codeid{#1}}
\newcommand{\codeid}[1]{\ifmmode{\mbox{\codesize\ttfamily{#1}}}\else{\codesize\ttfamily #1}\fi}

\newcommand{\borrow}{\&}
\newcommand{\mutborrow}{\<mut>}
\newcommand{\Vars}{\mathsf{Vars}}
\newcommand{\Locs}{\mathsf{Locs}}
\newcommand{\loc}{\mathsf{loc}}
\newcommand{\drop}{\mathsf{drop}}
\newcommand{\Lifetimes}{\mathsf{Lifetimes}}
\newcommand{\type}{\mathsf{type}}
\newcommand{\dom}{\mathsf{dom}}

\newcommand{\den}[2][]{\ensuremath{\llbracket #1\rrbracket{#2}}}
\newcommand{\denl}[3][]{\ensuremath{\llbracket #1\rrbracket^{#2}{#3}}}

\newcommand{\ran}{\mathit{ran}}

\usepackage{listings, xcolor}
\newcommand{\todo}[1]{\textcolor{red}{#1}}


\renewcommand{\UrlFont}{\ttfamily\codesize}

\definecolor{verylightgray}{rgb}{.97,.97,.97}

\lstdefinelanguage{Takamaka}{
        keywords=[1]{abstract, break, case, catch, class, continue, default, do
, else, false, finally, for, if, final, implements, extends, import, instanceof, interface, length, new, private, protected, public, return, super, switch, this, throw, true, try, while, var, null}, % generic keywords
        keywordstyle=[1]\color{blue}\bfseries,
        keywords=[2]{boolean, int, long, float, double, byte, short, char, void, enum}, % types; money and time units
        keywordstyle=[2]\color{teal}\bfseries,
        keywords=[3]{@Override,@View,@FromContract,@Payable}, % annotations
        keywordstyle=[3]\color{violet}\bfseries,
        identifierstyle=\color{black},
        sensitive=false,
        comment=[l]{//},
        morecomment=[s]{/*}{*/},
        commentstyle=\color{gray}\ttfamily,
        stringstyle=\color{red}\ttfamily,
        morestring=[b]',
        morestring=[b]"
}

\lstset{
        language=Takamaka,
        backgroundcolor=\color{verylightgray},
        extendedchars=true,
        basicstyle=\scriptsize\ttfamily,
        showstringspaces=false,
        showspaces=false,
        numbers=none,
        numberstyle=\scriptsize,
        numbersep=9pt,
        tabsize=2,
        breaklines=true,
        showtabs=false,
        captionpos=b
}

\lstdefinelanguage{JavaBytecode}{
        keywords=[1]{abstract, class, extends, public, private, protected}, % generic keywords
        keywordstyle=[1]\color{blue}\bfseries,
        keywords=[2]{boolean, int, long, float, double, byte, short, char, void}, % types; money and time units
        keywordstyle=[2]\color{teal}\bfseries,
        keywords=[3]{aload_0,aload_1,aload_2,aload_3,invokespecial,invokevirtual,checkcast,return}, % bytecodes
        keywordstyle=[3]\color{violet}\bfseries,
        identifierstyle=\color{black},
        sensitive=false,
        comment=[l]{//},
        morecomment=[s]{/*}{*/},
        commentstyle=\color{gray}\ttfamily,
        stringstyle=\color{red}\ttfamily,
        morestring=[b]',
        morestring=[b]"
}

\lstset{
        language=JavaBytecode,
        backgroundcolor=\color{verylightgray},
        extendedchars=true,
        basicstyle=\scriptsize\ttfamily,
        showstringspaces=false,
        showspaces=false,
        numbers=none,
        numberstyle=\scriptsize,
        numbersep=9pt,
        tabsize=2,
        breaklines=true,
        showtabs=false,
        captionpos=b
}

\begin{document}

\title{A Soundness Proof of Borrowing\\by Abstract Interpretation of Rust}
\titlerunning{A Soundness Proof of Borrowing by Abstract Interpretation of Rust}
\author{TBD}
\institute{TBD}

\maketitle

\begin{abstract}
  This paper provides a formal foundation of the semantics of
a significant subset of the Rust programming
language, defines a type system over it, and gives a soundness proof
of the type system through the formal framework of abstract interpretation.
This allows one to derive proofs of specific features of the
type system of Rust, including borrowing of values and their related
constraints, that are simpler than those already proposed in the past,
since abstract intepretation provides a standard
methodology for definitions and proofs.
The same semantics is the basis of future implementations of static analyses
for Rust, built over this formal framework.

  \keywords{TBD \and TBD}
\end{abstract}

\section{Introduction}\label{sec:introduction}

This section will introduce the context of the paper: Rust, its very
specific type system that guarantees memory safety, attempts and difficulties
of its formalizations, the advantage of a formalization based on abstract
interpretation (never done before for the type system of Rust).

\textbf{TODO}

\section{Semantics}\label{sec:semantics}

\begin{definition}
  We assume there is a set of variables $\Vars$
  and a finite set $\Lifetimes$ of \emph{lifetimes}.
  The set of \emph{leftvalues} \textsf{w} is defined as:
  \begin{align*}
    \mathsf{w} ::= &\ x & \text{variable, with $x\in\Vars$}\\
    | &\ \mathtt{*}\mathsf{w} & \text{dereference}
  \end{align*}
  The set of \emph{expressions} \textsf{e} is defined as
  \begin{align*}
    \mathsf{e} ::= &\ i & \text{integer}\\
    | &\ \mathsf{w} & \text{leftvalue}\\
    | &\ \borrow\mathsf{w} & \text{borrow}\\
    | &\ \mathtt{mut}\borrow\mathsf{w} & \text{mutable borrow}\\
    | &\ \mathtt{box}\ \mathsf{e} & \text{heap allocation}
  \end{align*}
  The sets of \emph{terms} \textsf{t} and \emph{blocks} \textsf{b}
  is defined as:
  \begin{align*}
    \mathsf{t} ::= &\ \mathsf{w}=\mathsf{e} & \text{assignment}\\
    | &\ \mathtt{let\ mut}\ x=\mathsf{e} & \text{declaration, with $x\in\Vars$}\\
    | &\ \{\mathsf{t_1};\ldots;\mathsf{t_n}\}^l & \text{block, with $n\ge 0$ and $l\in\Lifetimes$}
  \end{align*}
\end{definition}

\begin{definition}[Static types]
  We assume that expressions in a program have a static type, written $\type(e)$, that allows one
  to identify expressions that contain a box or a mutable borrow. We will write that
  ``$\type(e)$ is a box'', and that ``$\type(e)$ is a $\texttt{mut}\borrow$'' for that, respectively.
  Expressions whose type is a box or a $\texttt{mut}\borrow$ are said to have move semantics.
  All other expressions have copy semantics.
\end{definition}

\begin{example}\label{ex:program}
  The following valid program $P_1$ consists of a block that contains
  an inner block.
  \[
    \{
      \mathtt{let\ mut}\ x = \mathtt{box}\ 0;\
      \{\mathtt{let\ mut}\ y = \mathtt{mut}\borrow x;\
      \mathtt{*}y = \mathtt{box}\ 1\}^l;\
      \mathtt{let\ mut}\ z = x
    \}^m
  \]
  \qed
\end{example}

\begin{definition}
  We assume that there is an infinite set $\Locs$ of \emph{locations}.
  The set of \emph{values} is defined as $\mathbb{V}=\mathbb{Z}\cup\Locs$,
  where $\mathbb{Z}$ is the set of integers.

  The set $\mathbb{E}$
  of \emph{environments} contains bindings from variables to locations,
  decorated with a \emph{lifetime}:
  % and existing in two versions, owned and not owned:
  \[
  \mathbb{E}=\{x\to^m\ell\mid x\in\Vars,\ m\in\Lifetimes\text{ and }\ \ell\in\Locs\}
  \]
  with the constraint that, for each $\rho\in\mathbb{E}$, there is at most a binding for each
  given variable $x$. Moreover, we write $\rho(x)=\ell$ when
  $x\rightarrow^m\ell\in\rho$, for some $m$.

  The set $\mathbb{H}$ of \emph{heaps} contains bindings from locations to values,
  existing in three versions: owned, not owned and mutable borrows:
  \[
  \mathbb{H}=
  \underbrace{\{\ell\Rightarrow\ell'\mid\ell,\ell'\in\Locs\}}_{\text{owned bindings}}
  \cup
  \underbrace{\{\ell\rightarrow v\mid\ell\in\Locs\text{ and }v\in\mathbb{V}\}}_{\text{not owned bindings}}
  \cup
  \underbrace{\{\ell\leadsto\ell'\mid\ell,\ell'\in\Locs\}}_{\text{mutable borrows}}
  \]
  with the constraint that, for each $h\in\mathbb{H}$, there is at most a binding for each
  location $\ell$ (owned, not owned or mutable borrow). We write $h(\ell)=v$ meaning that
  either $\ell\Rightarrow v\in h$ or $\ell\rightarrow v\in h$ or $\ell\leadsto v\in h$.
  Given $L\subseteq\Locs$,
  we write $h|_{-L}$ meaning $h$ where the bindings for the locations in $L$
  have been removed (if any).

  The set of \emph{stores} $\mathbb{S}$ is the set of pairs
  $\mathbb{E}\times\mathbb{H}$, whose elements are written as $\rho\star h$.
  Given $s=\rho\star h\in\mathbb{S}$, we define $s(x)=\rho(x)$ for all
  $x\in\Vars$, and $s(\ell)=h(\ell)$ for all $\ell\in\Locs$.
  Given $L\subseteq\Locs$, we define $(\rho\star h)|_{-L}=\rho\star(h|_{-L})$.
\end{definition}

\noindent
We observe that the above definition of $\mathbb{H}$ allows one to represent mutable borrows
through the $\leadsto$ bindings, but also normal (that is, non-mutable) borrows, that are
just a special case of not owned bindings, hence through the $\to$ bindings.

\begin{definition}[Dangling locations]
  Let $h\in\mathbb{H}$. A location $\ell$ is \emph{dangling} in $h$ if
  and only if $h(\ell)$ is undefined.
\end{definition}

\begin{definition}[Borrowed locations]
  Let $h\in\mathbb{H}$. % and $l'\in\Locs$.
  A location $\ell$ is \emph{non-mutably borrowed} in $h$
  if there exists $\ell'\in\Locs$ such that $\ell'\to\ell$.
  A location $\ell$ is \emph{mutably borrowed} in $h$
  if there exists $\ell'\in\Locs$ such that $\ell'\leadsto\ell$.
  A location $\ell$ is \emph{borrowed} in $h$
  if it is mutably or non-mutably borrowed in $h$.
\end{definition}

%\begin{definition}[Consistent heaps]
%  Given $h\in\mathbb{H}$, it is \emph{consistent} when
  %\begin{enumerate}
  %\item no location is at the same time mutably and non-mutably borrowed in $h$
  %\item
%  no location is mutably borrowed more than once, that is, for every
%    $\ell\in\Locs$ there are no $\ell'\in\Locs$ and $\ell''\in\Locs$ such that
%    $\ell'\not=\ell''$, $\ell'\leadsto\ell\in h$ and $\ell''\leadsto\ell\in h$.
  %\end{enumerate}
%  We say that a store is consistent when its heap is consistent.
%\end{definition}

\begin{definition}[Locate]\label{def:locate}
  The partial function $\loc(\mathsf{w},s)$ determines the location that holds
  the value of the leftvalue $\mathsf{w}$ in $s\in\mathbb{S}$, as follows:
  \begin{align*}
    \loc(x,s) &= s(x)\\
    \loc(*\mathsf{w},s) &= \left\{\begin{array}{ll}
      s(\loc(\mathsf{w},s)) & \text{if } s(\loc(\mathsf{w},s)) \in \Locs\\
      \text{undefined} & \text{otherwise}
    \end{array}\right.
  \end{align*}
\end{definition}

\noindent
Note that $\loc$ is a partial function since the second case of
Def.~\ref{def:locate} is undefined when
$s(\loc(\mathsf{w},s))\not\in\Locs$. In the following, when $\loc$ is undefined, the
semantics will be stuck.

\begin{example}
  Let $\rho = \left\{x\rightarrow^m\ell_x\right\}$.
  \begin{itemize}
    \item Let $s = \rho\star h$ with
    $h = \left\{\ell_x\rightarrow\ell,\ \ell\rightarrow 1\right\}$.
    Then, $s(\loc(x,s)) = s(s(x)) = s(\rho(x)) = s(\ell_x) =
    h(\ell_x) = \ell\in\Locs$, so
    $\loc(\mathtt{*}x,s)= s(\loc(x,s)) = \ell$.
    %
    \item Let $s' = \rho\star h'$ with
    $h' = \left\{\ell_x\rightarrow 1\right\}$. Then,
    $s'(\loc(x,s')) = s'(s'(x)) = s'(\rho(x)) = s'(\ell_x) =
    h'(\ell_x) = 1 \not\in\Locs$, so $\loc(\mathtt{*}x,s')$ is undefined.
    \qed
  \end{itemize}
\end{example}

The evaluation of a leftvalue yields the value of the leftvalue and a potentially updated heap:

\begin{definition}[Semantics of leftvalues]\label{def:semantics_leftvalues}
  Given $s=\rho\star h$, we define
  \[
  \den[\mathsf{w}]{s}=\begin{cases}
  \text{undefined} & \text{if $v\in\Locs$ and $v$ is dangling in $h$}\\
  \langle v, h\rangle & \text{otherwise, if $\type(\mathsf{w})$ has copy semantics}\\
  \langle v, h|_{-\loc(\mathsf{w},s)}\rangle & \text{otherwise, if $\type(\mathsf{w})$ has move semantics}
  \end{cases}
  \]
  where $v=s(\loc(\mathsf{w},s))$.
\end{definition}

\noindent
Note that the second case above makes $\mathsf{w}$ point to a dangling location, hence
$\mathsf{w}$ becomes unusable.

The evaluation of an expression yields the value of the expression and a potentially updated heap:

\begin{definition}[Semantics of expressions]\label{def:semantics_expressions}
  Given $s=\rho\star h$, we define
  \begin{align*}
    \den[i]{s}&=\langle i,h\rangle\\
    \den[\borrow\mathsf{w}]{s}&=\langle\loc(\mathsf{w},s),h\rangle\\
    \den[\mathtt{mut}\borrow\mathsf{w}]{s}&=\langle\loc(\mathsf{w},s),h\rangle\\
    \den[\mathtt{box}\ \mathsf{e}]{s}&=
    \begin{cases}
      \langle\ell,h'\cup\{\ell\Rightarrow v\}\rangle & \text{if $\type(e)$ is a box}\\
      \langle\ell,h'\cup\{\ell\leadsto v\}\rangle & \text{otherwise, if $\type(e)$ is $\mathtt{mut}\borrow$}\\
      \langle\ell,h'\cup\{\ell\to v\}\rangle & \text{otherwise}
    \end{cases}\\
    & \text{where $\den[\mathsf{e}]{s}=\langle v, h'\rangle$ and $\ell$ is fresh}
  \end{align*}
  For expressions that are leftvalues, see Def.~\ref{def:semantics_leftvalues}.
\end{definition}

The set of locations reachable by following owned bindings only, and starting from a set of locations $\psi$,
is defined as follows:
%
\begin{definition}[Owned reachable locations]
  Given $\psi\subseteq\Locs$ and $h\in\mathbb{H}$, we define
  \begin{align*}
  \omega^0(\psi,h)&=\psi\\ %\{\ell'\in\Locs\mid\ell\in\psi\text{ and }\ell\Rightarrow\ell'\in h\}\\
  \omega^{i+1}(\psi,h)&=\{\ell'\in\Locs\mid\ell\in\omega^i(\psi,h)\text{ and }\ell\Rightarrow\ell'\in h\}
  \end{align*}
  and
  \[
  \omega(\psi,h)=\bigcup\limits_{i\ge 0}\omega^i(\psi,h)~.
  \]
\end{definition}

The function $\drop$ removes the owned locations reachable from a set of locations.
%
\begin{definition}[Drop]\label{def:drop}
  The function $\drop:\wp(\Locs)\times\mathbb{H}\to\mathbb{H}$ is defined as
  \[
  \drop(\psi,h)=h|_{-{\omega(\psi,h)}}~.
  \]
\end{definition}

\begin{definition}[Semantics of terms]\label{def:semantics_terms}
  Given $s=\rho\star h$ and $m\in\Lifetimes$, we define
  \begin{align*}
    \denl[x=\mathsf{e}]{m}{s}&=
    \begin{cases}
      \text{undefined} & \text{if $\ell$ is borrowed in $h'$}\\
      \langle\rho,h''[\ell\Rightarrow v]\rangle & \text{otherwise, if $\type(\mathsf{e})$ is a box}\\
      \langle\rho,h''[\ell\leadsto v]\rangle & \text{otherwise, if $\type(\mathsf{e})$ is a $\mathtt{mut}\borrow$}\\
                                             & \text{and $v$ is not mutably borrowed in $h'$}\\
      \text{undefined} & \text{otherwise, if $\type(\mathsf{e})$ is a $\mathtt{mut}\borrow$}\\
                       & \text{and $v$ is mutably borrowed in $h'$}\\
      \langle\rho,h''[\ell\to v]\rangle & \text{otherwise}
    \end{cases}\\
    &\text{where $\den[\mathsf{e}]{s}=\langle v,h'\rangle$, $\ell=\loc(x,\langle\rho,h'\rangle)$ and
      $h''=\drop(\{\ell\},h')$}\\
    \mbox{}\\
    \denl[*\mathsf{w}=\mathsf{e}]{m}{s}&=
    \begin{cases}
      \text{undefined} & \text{if $\ell$ is immutably borrowed in $h'$ or $\ell_{\mathsf{w}}\leadsto\ell\not\in h'$}\\
      \langle\rho,h''[\ell\Rightarrow v]\rangle & \text{otherwise, if $\type(\mathsf{e})$ is a box}\\
      \langle\rho,h''[\ell\leadsto v]\rangle & \text{otherwise, if $\type(\mathsf{e})$ is a $\mathtt{mut}\borrow$}\\
      \langle\rho,h''[\ell\to v]\rangle & \text{otherwise}
    \end{cases}\\
    &\text{where $\den[\mathsf{e}]{s}=\langle v,h'\rangle$, $\ell_{\mathsf{w}}=\loc(\mathsf{w},\langle\rho,h'\rangle)$,
      $\ell=s(\ell_{\mathsf{w}})$ and $h''=\drop(\{\ell\},h')$}\\
    \mbox{}\\
    \denl[\mathtt{let\ mut}\ x=\mathsf{e}]{m}{s}&=
    \begin{cases}
      \langle\rho[x\to^m\ell],h'[\ell\Rightarrow v]\rangle & \text{if $\type(\mathsf{e})$ is a box}\\
      \langle\rho[x\to^m\ell],h'[\ell\leadsto v]\rangle & \text{otherwise, if $\type(\mathsf{e})$ is a $\mathtt{mut}\borrow$}\\
                                                        & \text{and $v$ is not mutably borrowed in $h'$}\\
      \text{undefined} & \text{otherwise, if $\type(\mathsf{e})$ is a $\mathtt{mut}\borrow$}\\
                       & \text{and $v$ is mutably borrowed in $h'$}\\
      \langle\rho[x\to^m\ell],h'[\ell\to v]\rangle & \text{otherwise}
    \end{cases}\\
    &\text{where $\den[\mathsf{e}]{s}=\langle v,h'\rangle$ and $\ell$ is fresh}\\
    \mbox{}\\
    \denl[\{\mathsf{t_1};\ldots;\mathsf{t_n}\}^l]{m}{s}&=\langle\rho,\drop(\{\ell\mid x\to^l\ell\in\rho'\},h')\rangle\\
    &\text{where $\langle\rho',h'\rangle=\denl[\mathsf{t_n}]{l}{\denl[\mathsf{t_{n-1}}]{l}{\cdots\denl[\mathsf{t_1}]{l}{s}}}$.}
  \end{align*}
\end{definition}

\begin{example}
  Consider again the program $P_1$ of Ex.~\ref{ex:program}:
  \[
    \{
      \mathtt{let\ mut}\ x = \mathtt{box}\ 0;\
      \{\mathtt{let\ mut}\ y = \mathtt{mut}\borrow x;\
      \mathtt{*}y = \mathtt{box}\ 1\}^l;\
      \mathtt{let\ mut}\ z = x
    \}^m
  \]
  \begin{itemize}
    \item Let $s_0=\emptyset \star \emptyset$.
    We have $\den[0]{s_0} = \langle 0,\emptyset\rangle$, hence
    $\den[\mathtt{box}\ 0]{s_0} =
    \langle\ell,\{\ell\to 0\}\rangle$ where $\ell$ is fresh.
    %
    Consequently, we have
    $\denl[\mathtt{let\ mut}\ x = \mathtt{box}\ 0]{m}{s_0} =
    \langle \rho_1,\ h_1\rangle$ where
    \[\rho_1=\{x\to^m\ell_x\} \qquad
    h_1 = \{\ell_x\Rightarrow \ell,\ \ell\to 0\}\]
    and $\ell_x$ is fresh.
    %
    \item Let $s_1 = \rho_1 \star h_1$.
    Then, we have $\loc(x,s_1) = \ell_x$, so
    $\den[\mathtt{mut}\borrow x]{s_1} = \langle\ell_x, h_1\rangle$.
    Therefore, we have
    $\denl[\mathtt{let\ mut}\ y = \mathtt{mut}\borrow x]{l}{s_1} =
    \langle \rho_2,\ h_2\rangle$
    where
    \begin{align*}
      \rho_2 & = \rho_1[y\to^l\ell_y] = \{y\to^l\ell_y,\ x\to^m\ell_x\}\\
      h_2 & = h_1[\ell_y\leadsto \ell_x] =
      \{\ell_y\leadsto \ell_x,\ \ell_x\Rightarrow \ell,\ \ell\to 0\}
    \end{align*}
    and $\ell_y$ is fresh.
    %
    \item Let $s_2 = \rho_2 \star h_2$.
    We have $\den[1]{s_2} = \langle 1,h_2\rangle$, hence
    $\den[\mathtt{box}\ 1]{s_2} =
    \langle\ell',h_2[\ell'\to 1]\rangle$ where $\ell'$ is fresh.
    %
    If we let $h'_2 = h_2[\ell'\to 1]$ and $s'_2=\langle\rho_2,h'_2\rangle$,
    we have $\loc(\mathtt{*}y,s'_2) = s'_2(\loc(y,s'_2)) =
    s'_2(s'_2(y)) = s'_2(\rho_2(y)) = s'_2(\ell_y) = h'_2(\ell_y) = \ell_x$
    is not borrowed in $h'_2$.
    Moreover, $h^{''}_2=\drop(\{\ell_x\},h'_2) =
    \{\ell_y\leadsto \ell_x,\ell'\to 1\}$. Consequently,
    $\denl[\mathtt{*}y = \mathtt{box}\ 1]{l}{s_2} =
    \langle\rho_3,h_3\rangle$ where
    \[\rho_3 = \rho_2
    \text{ and }
    h_3 = \{\ell_y\leadsto\ell_x,\ell_x\Rightarrow\ell',
    \ell'\to 1\}\]
    %
    \item Let us evaluate
    $\denl[\{\mathtt{let\ mut}\ y = \mathtt{mut}\borrow x;\
    \mathtt{*}y = \mathtt{box}\ 1\}^l]{m}{s_1}$. Above, we have
    computed $\denl[\mathtt{*}y = \mathtt{box}\ 1]{l}{
    \denl[\mathtt{let\ mut}\ y = \mathtt{mut}\borrow x]{l}{s_1}} =
    \langle\rho_3,h_3\rangle$. Moreover,
    $\drop(\{\ell\mid x\to^l\ell\in\rho_3\},h_3) = \drop(\{\ell_y\},h_3) =
    \{\ell_x\Rightarrow\ell',\ell'\to 1\}$. Consequently,
    $\denl[\{\mathtt{let\ mut}\ y = \mathtt{mut}\borrow x;\
    \mathtt{*}y = \mathtt{box}\ 1\}^l]{m}{s_1} =
    \langle \rho_4,h_4\rangle$ where
    \[\rho_4 = \rho_1 \text{ and } h_4 = \{\ell_x\Rightarrow\ell',\ell'\to 1\}\]
    %
    \item Let $s_4 = \rho_4 \star h_4$. As $\type(x)$ is a box,  $x$ has
    move semantics. Consequently, $\den[x]{s_4} =
    \langle s_4(\loc(x,s_4)), {h_4}|_{-\loc(x,s_4)}\rangle$
    with $\loc(x,s_4)=\ell_x$ and $s_4(\loc(x,s_4)) = s_4(\ell_x)
    = h_4(\ell_x)=\ell'$\ie
    $\den[x]{s_4} = \langle \ell', \{\ell'\to 1\} \rangle$.
    So, $\denl[\mathtt{let\ mut}\ z = x]{m}{s_4} = \langle \rho_5,h_5\rangle$
    where
    \begin{align*}
      \rho_5 & = \rho_4[z\to^m \ell_z] = \{z\to^m \ell_z,\ x\to^m\ell_x\}\\
      h_5 & = \{\ell_z\Rightarrow \ell',\ \ell'\to 1\}
    \end{align*}
    and $\ell_z$ is fresh.
  \end{itemize}
  \qed
\end{example}

\begin{example}
  Consider the following illegal program $P_2$
  that attempts to create two mutable references
  to the same piece of data in the same scope:
  \[
    \{
      \mathtt{let\ mut}\ x = 13;\
      \mathtt{let\ mut}\ y = \mathtt{mut}\borrow x;\
      \mathtt{let\ mut}\ z = \mathtt{mut}\borrow x
    \}^m
  \]
  \begin{itemize}
    \item Let $s_0=\emptyset \star \emptyset$.
    We have $\den[13]{s_0} = \langle 13,\emptyset\rangle$, hence
    $\denl[\mathtt{let\ mut}\ x = 13]{m}{s_0} =
    \langle \rho_1,\ h_1\rangle$ where
    \[\rho_1=\{x\to^m\ell_x\} \qquad
    h_1 = \{\ell_x\to 13\}\]
    and $\ell_x$ is fresh.
    %
    \item Let $s_1 = \rho_1 \star h_1$.
    Then, we have $\loc(x,s_1) = \ell_x$, so
    $\den[\mathtt{mut}\borrow x]{s_1} = \langle \ell_x,h_1\rangle$.
    Therefore, we have
    $\denl[\mathtt{let\ mut}\ y = \mathtt{mut}\borrow x]{m}{s_1} =
    \langle \rho_2,\ h_2\rangle$
    where
    \begin{align*}
      \rho_2 & = \rho_1[y\to^m\ell_y] = \{y\to^m\ell_y,\ x\to^m\ell_x\}\\
      h_2 & = h_1[\ell_y\leadsto \ell_x] =
      \{\ell_y\leadsto \ell_x,\ \ell_x\to 13\}
    \end{align*}
    and $\ell_y$ is fresh.
    %
    \item Let $s_2 = \rho_2 \star h_2$.
    Then, we have $\loc(x,s_2) = \ell_x$, so
    $\den[\mathtt{mut}\borrow x]{s_2} = \langle \ell_x,h_2\rangle$.
    As $h_2[\ell_z\leadsto \ell_x]$ is not consistent for any fresh location
    $\ell_z$, we have that
    $\denl[\mathtt{let\ mut}\ z = \mathtt{mut}\borrow x]{m}{s_2}$
    is undefined.
  \end{itemize}
  \qed
\end{example}

\begin{example}
  Consider the following valid program $P'_2$ which is a variant
  of $P_2$ with immutable references:
  \[
    \{
      \mathtt{let\ mut}\ x = 13;\
      \mathtt{let\ mut}\ y = \borrow x;\
      \mathtt{let\ mut}\ z = \borrow x
    \}^m
  \]
  \begin{itemize}
    \item Let $s_0=\emptyset \star \emptyset$.
    We have $\den[13]{s_0} = \langle 13,\emptyset\rangle$, hence
    $\denl[\mathtt{let\ mut}\ x = 13]{m}{s_0} =
    \langle \rho_1,\ h_1\rangle$ where
    \[\rho_1=\{x\to^m\ell_x\} \qquad
    h_1 = \{\ell_x\to 13\}\]
    and $\ell_x$ is fresh.
    %
    \item Let $s_1 = \rho_1 \star h_1$.
    Then, we have $\loc(x,s_1) = \ell_x$, so
    $\den[\borrow x]{s_1} = \langle \ell_x,h_1\rangle$.
    Therefore, we have
    $\denl[\mathtt{let\ mut}\ y = \borrow x]{m}{s_1} =
    \langle \rho_2,\ h_2\rangle$
    where
    \begin{align*}
      \rho_2 & = \rho_1[y\to^m\ell_y] = \{y\to^m\ell_y,\ x\to^m\ell_x\}\\
      h_2 & = h_1[\ell_y\to \ell_x] =
      \{\ell_y\to \ell_x,\ \ell_x\to 13\}
    \end{align*}
    and $\ell_y$ is fresh.
    %
    \item Let $s_2 = \rho_2 \star h_2$.
    Then, we have $\loc(x,s_2) = \ell_x$, so
    $\den[\borrow x]{s_2} = \langle \ell_x,h_2\rangle$.
    Therefore, we have
    $\denl[\mathtt{let\ mut}\ z = \borrow x]{m}{s_2} =
    \langle \rho_3,\ h_3\rangle$
    where
    \begin{align*}
      \rho_3 & = \rho_2[z\to^m\ell_z] = \{z\to^m\ell_z,\ y\to^m\ell_y,\ x\to^m\ell_x\}\\
      h_3 & = h_2[\ell_z\to \ell_x] =
      \{\ell_z\to \ell_x,\ \ell_y\to \ell_x,\ \ell_x\to 13\}
    \end{align*}
    and $\ell_y$ is fresh.
  \end{itemize}
  \qed
\end{example}

\begin{example}
  Consider the following illegal program $P_3$
  where variable $x$ is assigned to $1$ while
  borrowed to $y$:
  \[
    \{
      \mathtt{let\ mut}\ x = 0;\
      \mathtt{let\ mut}\ y = \mathtt{mut}\borrow x;\
      x = 1
    \}^m
  \]
  \begin{itemize}
    \item Let $s_0=\emptyset \star \emptyset$.
    We have $\den[0]{s_0} = \langle 0,\emptyset\rangle$, hence
    $\denl[\mathtt{let\ mut}\ x = 0]{m}{s_0} =
    \langle \rho_1,\ h_1\rangle$ where
    \[\rho_1=\{x\to^m\ell_x\} \qquad
    h_1 = \{\ell_x\to 0\}\]
    and $\ell_x$ is fresh.
    %
    \item Let $s_1 = \rho_1 \star h_1$.
    Then, we have $\loc(x,s_1) = \ell_x$, so
    $\den[\mathtt{mut}\borrow x]{s_1} = \langle \ell_x,h_1\rangle$.
    Therefore, we have
    $\denl[\mathtt{let\ mut}\ y = \mathtt{mut}\borrow x]{m}{s_1} =
    \langle \rho_2,\ h_2\rangle$
    where
    \begin{align*}
      \rho_2 & = \rho_1[y\to^m\ell_y] = \{y\to^m\ell_y,\ x\to^m\ell_x\}\\
      h_2 & = h_1[\ell_y\leadsto \ell_x] =
      \{\ell_y\leadsto \ell_x,\ \ell_x\to 0\}
    \end{align*}
    and $\ell_y$ is fresh.
    %
    \item Let $s_2 = \rho_2 \star h_2$.
    Then, we have $\den[1]{s_2} = \langle 1,h'_2\rangle$ with $h'_2=h_2$.
    Moreover, $\loc(x,\langle\rho_2,h'_2\rangle) = \ell_x$ is
    mutably borrowed in $h'_2$.
    Therefore, $\denl[x = 1]{m}{s_2}$ is undefined.
  \end{itemize}
  \qed
\end{example}

\begin{example}
  Consider the following valid program $P_4$
  where the first declaration of variable $x$ is
  shadowed by the second:
  \[
    \{
      \mathtt{let\ mut}\ x = 0;\
      \mathtt{let\ mut}\ x = 1
    \}^m
  \]
  \begin{itemize}
    \item Let $s_0=\emptyset \star \emptyset$.
    We have $\den[0]{s_0} = \langle 0,\emptyset\rangle$, hence
    $\denl[\mathtt{let\ mut}\ x = 0]{m}{s_0} =
    \langle \rho_1,\ h_1\rangle$ where
    \[\rho_1=\{x\to^m\ell_x\} \qquad
    h_1 = \{\ell_x\to 0\}\]
    and $\ell_x$ is fresh.
    %
    \item Let $s_1 = \rho_1 \star h_1$.
    We have $\den[1]{s_1} = \langle 1,h_1\rangle$, hence
    $\denl[\mathtt{let\ mut}\ x = 1]{m}{s_1} =
    \langle \rho_2,\ h_2\rangle$ where
    \[\rho_2=\{x\to^m\ell'_x\} \qquad
    h_2 = \{\ell_x\to 0,\ \ell'_x\to 1\}\]
    and $\ell'_x$ is fresh.
  \end{itemize}
  \qed
\end{example}

\begin{example}
  Consider the following illegal program $P_5$ that attempts to create
  a borrowed reference to variable $z$ that exists outside of its lifetime:
  {\small
  \[
    \{
      \mathtt{let\ mut}\ x = 0;\
      \mathtt{let\ mut}\ y = \mathtt{mut}\borrow x;\
      \{\mathtt{let\ mut}\ z = 0;\ y = \mathtt{mut}\borrow z\}^l;\
      \mathtt{let\ mut}\ w = y
    \}^m
  \]}
  \begin{itemize}
    \item Let $s_0=\emptyset \star \emptyset$.
    We have $\den[0]{s_0} = \langle 0,\emptyset\rangle$, hence
    $\denl[\mathtt{let\ mut}\ x = 0]{m}{s_0} =
    \langle \rho_1,\ h_1\rangle$ where
    \[\rho_1=\{x\to^m\ell_x\} \qquad
    h_1 = \{\ell_x\to 0\}\]
    and $\ell_x$ is fresh.
    %
    \item Let $s_1 = \rho_1 \star h_1$.
    Then, we have $\loc(x,s_1) = \ell_x$, so
    $\den[\mathtt{mut}\borrow x]{s_1} = \langle\ell_x, h_1\rangle$.
    Therefore, we have
    $\denl[\mathtt{let\ mut}\ y = \mathtt{mut}\borrow x]{m}{s_1} =
    \langle \rho_2,\ h_2\rangle$
    where
    \begin{align*}
      \rho_2 & = \rho_1[y\to^m\ell_y] = \{y\to^m\ell_y,\ x\to^m\ell_x\}\\
      h_2 & = h_1[\ell_y\leadsto \ell_x] =
      \{\ell_y\leadsto \ell_x,\ \ell_x\to 0\}
    \end{align*}
    and $\ell_y$ is fresh.
    %
    \item Let $s_2 = \rho_2 \star h_2$.
    We have $\den[0]{s_2} = \langle 0,h_2\rangle$, hence
    $\denl[\mathtt{let\ mut}\ z = 0]{l}{s_2} =
    \langle \rho_3,\ h_3\rangle$ where
    \begin{align*}
      \rho_3 & = \rho_2[z\to^l\ell_z] =
      \{z\to^l\ell_z,\ y\to^m\ell_y,\ x\to^m\ell_x\}\\
      h_3 & = h_2[\ell_z\to 0] =
      \{\ell_z\to 0,\ \ell_y\leadsto \ell_x,\ \ell_x\to 0\}
    \end{align*}
    and $\ell_z$ is fresh.
    %
    \item Let $s_3 = \rho_3 \star h_3$.
    Then, we have $\loc(z,s_3) = \ell_z$, so
    $\den[\mathtt{mut}\borrow z]{s_3} = \langle\ell_z, h'_3\rangle$
    with $h'_3=h_3$.
    Moreover, $\loc(y,\langle \rho_3,h'_3\rangle) = \ell_y$ is not
    borrowed in $h'_3$ and $\drop(\{\ell_y\},h'_3) =
    \{\ell_z\to 0,\ell_x\to 0\}$. Consequently,
    $\denl[y = \mathtt{mut}\borrow z]{l}{s_3} = \langle \rho_4,\ h_4\rangle$
    where
    \[\rho_4=\rho_3 \text{ and }
    h_4=\{\ell_y\leadsto\ell_z,\ \ell_z\to 0,\ \ell_x\to 0\}\]
    %
    \item Let us evaluate
    $\denl[\{\mathtt{let\ mut}\ z = 0;\ y = \mathtt{mut}\borrow z\}^l]{m}{s_2}$.
    Above, we have computed $\denl[y = \mathtt{mut}\borrow z]{l}{
    \denl[\mathtt{let\ mut}\ z = 0]{l}{s_2}} = \langle\rho_4,h_4\rangle$.
    Moreover, we also have
    $\drop(\{\ell\mid x\to^l\ell\in\rho_4\},h_4) = \drop(\{\ell_z\},h_4) =
    \{\ell_y\leadsto\ell_z,\ \ell_x\to 0\}$. Consequently,
    $\denl[\{\mathtt{let\ mut}\ z = 0;\ y = \mathtt{mut}\borrow z\}^l]{m}{s_2} =
    \langle \rho_5,h_5\rangle$ where
    \[\rho_5 = \rho_2 \text{ and }
    h_5 = \{\ell_y\leadsto\ell_z,\ \ell_x\to 0\}\]
    %
    \item Let $s_5 = \rho_5 \star h_5$. As $\type(y)$ has move semantics
    and $\loc(y,s_5)=\ell_y$, we have
    $\den[y]{s_5} = \langle s_5(\ell_y), h_5|_{-\ell_y}\rangle =
    \langle \ell_z, \{\ell_x\to 0\}\rangle$. Therefore,
    $\denl[\mathtt{let\ mut}\ w = y]{m}{s_5} =
    \langle \rho_6,\ h_6\rangle$ where
    \begin{align*}
      \rho_6 & = \rho_5[w\to^m\ell_w] =
      \{w\to^m\ell_w,\ y\to^m\ell_y,\ x\to^m\ell_x\}\\
      h_6 & = \{\ell_w \leadsto \ell_z,\ \ell_x\to 0\}
    \end{align*}
    and $\ell_w$ is fresh.
  \end{itemize}
  \qed
\end{example}

\begin{example}\label{ex:program-reborrow-valid}
  Consider the following valid program $P_6$:
  \[
    \{
      \mathtt{let\ mut}\ x = 0;\
      \{
        \mathtt{let\ mut}\ y = \mathtt{mut}\borrow x;\
        \{
          \mathtt{let\ mut}\ z = \borrow*y;\
        \}^l;\
        *y = 1
      \}^n
    \}^m
  \]
  \begin{itemize}
    \item Let $s_0=\emptyset \star \emptyset$.
    We have $\den[0]{s_0} = \langle 0,\emptyset\rangle$, hence
    $\denl[\mathtt{let\ mut}\ x = 0]{m}{s_0} =
    \langle \rho_1,\ h_1\rangle$ where
    \[\rho_1=\{x\to^m\ell_x\} \qquad
    h_1 = \{\ell_x\to 0\}\]
    and $\ell_x$ is fresh.
    %
    \item Let $s_1 = \rho_1 \star h_1$.
    Then, we have $\loc(x,s_1) = \ell_x$, so
    $\den[\mathtt{mut}\borrow x]{s_1} = \langle\ell_x, h_1\rangle$.
    Therefore, we have
    $\denl[\mathtt{let\ mut}\ y = \mathtt{mut}\borrow x]{n}{s_1} =
    \langle \rho_2,\ h_2\rangle$
    where
    \begin{align*}
      \rho_2 & = \rho_1[y\to^n\ell_y] = \{y\to^n\ell_y,\ x\to^m\ell_x\}\\
      h_2 & = h_1[\ell_y\leadsto \ell_x] =
      \{\ell_y\leadsto \ell_x,\ \ell_x\to 0\}
    \end{align*}
    and $\ell_y$ is fresh.
    \item Let $s_2 = \rho_2 \star h_2$.
    We have $\loc(*y,s_2) = s_2(\loc(y,s_2)) = s_2(\ell_y) = h_2(\ell_y) = \ell_x$,
    so $\den[\borrow*y]{s_2} = \langle \ell_x,h_2\rangle$.
    Moreover, $\type(\borrow*y)$ is not a box nor a $\mathtt{mut}\borrow$ and
    $h_2[\ell_z\to \ell_x] =
    \{\ell_z\to \ell_x,\ \ell_y\leadsto \ell_x,\ \ell_x\to 0\}$ is not consistent
    because $\ell_x$ is both mutably and non-mutably borrowed.
    Consequently, $\denl[\mathtt{let\ mut}\ z = \borrow*y]{l}{s_2}$
    is undefined.
  \end{itemize}
  \qed
\end{example}

\begin{example}
  Consider the following program $P'_6$ which is a variant of $P_6$:
  \[
  \{
    \mathtt{let\ mut}\ x = 0;\
    \{
      \mathtt{let\ mut}\ y = \mathtt{mut}\borrow x;\
        \{
          \mathtt{let\ mut}\ z = \borrow*y;\
          *y = 1
        \}^l
      \}^n
    \}^m
  \]
  $P'_6$ is illegal because the immutable reborrow $\borrow *y$ in the inner
  block prohibits the writing $*y = 1$ in this block. Starting from the
  store $s_0=\emptyset\star\emptyset$, one gets the same result
  as in Ex.~\ref{ex:program-reborrow-valid}.
  \qed
\end{example}

\section{Abstract Semantics}\label{sec:abstract_semantics}

\begin{definition}[Types]
  The set of \emph{types} over a context $\kappa$ (Def.~\ref{def:context}) is
  \begin{align*}
    \mathsf{T}_\kappa ::=&\ \mathsf{int} & \text{integer}\\
    | &\ \borrow\{\mathsf{w}_1,\ldots,\mathsf{w}_n\} & \text{borrow}\\
    | &\ \mutborrow\{\mathsf{w}_1,\ldots,\mathsf{w}_n\} & \text{mutable borrow}\\
    | &\ \boxtype{\mathsf{T}_\kappa} & \text{box}\\
    | &\ \mathsf{dangling} & \text{dangling}
  \end{align*}
  where $n\ge 1$ and the $\mathsf{w}_i$, with $1\le i\le n$, are leftvalues
  using only variables in $\dom(\kappa)$. This set is ordered by $\sqsubseteq$, defined as
  \begin{align*}
    \mathsf{int}&\sqsubseteq\mathsf{int}\\
    \borrow\{\mathsf{w}_1,\ldots,\mathsf{w}_n\} & \sqsubseteq
    \borrow\{\mathsf{w}'_1,\ldots,\mathsf{w}'_{n'}\} \quad\text{iff $\{\mathsf{w}_1,\ldots,\mathsf{w}_n\}\subseteq\{\mathsf{w}'_1,\ldots,\mathsf{w}'_{n'}\}$}\\
    \mutborrow\{\mathsf{w}_1,\ldots,\mathsf{w}_n\} & \sqsubseteq
    \mutborrow\{\mathsf{w}'_1,\ldots,\mathsf{w}'_{n'}\} \quad\text{iff $\{\mathsf{w}_1,\ldots,\mathsf{w}_n\}\subseteq\{\mathsf{w}'_1,\ldots,\mathsf{w}'_{n'}\}$}\\
    \boxtype{t_1}&\sqsubseteq\boxtype{t_2} \quad\text{iff $t_1\sqsubseteq t_2$}\\
    \mathsf{dangling}&\sqsubseteq\mathsf{dangling.}
  \end{align*}
\end{definition}

\begin{definition}[Copy and move types]\label{def:copy_move}
  Let $t\in\mathsf{T}_\kappa$. Then $t$ \emph{has move semantics}, and we write
  $\movetype(t)$, if and only if $t=\mutborrow\{\mathsf{w}_1,\ldots,\mathsf{w}_n\}$ or
  $t=\boxtype{t'}$ for some $t'$. In all other cases, $t$ \emph{has copy semantics},
  and we write $\copytype(t)$.
\end{definition}

\begin{lemma}
  \label{lemma:partial-order-types}
  $(\mathsf{T}_\kappa,\sqsubseteq)$ is a partially ordered set.
\end{lemma}
\begin{proof}
  We proceed by structural induction.
  \begin{itemize}
    \item (Reflexivity)
    \begin{itemize}
      \item (Base) For any type $t$ that is not a box we have
      $t\sqsubseteq t$.
      \item (Induction step) If $t\sqsubseteq t$ for a type $t$ then
      $\boxtype{t}\sqsubseteq \boxtype{t}$.
    \end{itemize}
    %
    \item (Antisymmetry) Suppose that $t_1 \sqsubseteq t_2$ and
    $t_2 \sqsubseteq t_1$.
    \begin{itemize}
      \item (Base: $t_1$ or $t_2$ is not a box)
      If $t_1$ or $t_2$ is $\mathsf{int}$, then so is the other
      because $\mathsf{int}$ can only be greater than itself, and so $t_1 = t_2$.
      Similarly if $t_1$ or $t_2$ is $\mathsf{dangling}$.
      If $t_1$ or $t_2$ is a borrow, then so is the other because
      a borrow can only precede a borrow, and so we have $t_1 = t_2$
      by antisymmetry of $\subseteq$.
      Similarly if $t_1$ or $t_2$ is a mutable borrow.
      \item (Induction step) Suppose that $t_1=\boxtype{t'_1}$ and
      $t_2=\boxtype{t'_2}$ for types $t'_1,t'_2$ such that
      $(t'_1 \sqsubseteq t'_2 \land t'_2 \sqsubseteq t'_1) \Rightarrow t'_1 = t'_2$.
      As $t_1 \sqsubseteq t_2$ and $t_2 \sqsubseteq t_1$, we have
      $t'_1 \sqsubseteq t'_2$ and $t'_2 \sqsubseteq t'_1$, so
      $t'_1 = t'_2$, hence $t_1 = t_2$.
    \end{itemize}
    %
    \item (Transitivity) Suppose that $t_1 \sqsubseteq t_2$ and $t_2 \sqsubseteq t_3$.
    \begin{itemize}
      \item (Base: $t_1$, $t_2$ or $t_3$ is not a box)
      If $t_1$, $t_2$ or $t_3$ is $\mathsf{int}$, we necessarily have
      $t_1=t_2=t_3=\mathsf{int}$ because $\mathsf{int}$ is only related to
      itself, and so $t_1\sqsubseteq t_3$. Similarly if
      $t_1$, $t_2$ or $t_3$ is $\mathsf{dangling}$.
      If $t_1$, $t_2$ or $t_3$ is a borrow, then so are the others because
      a borrow is only related to a borrow, and so we have
      $t_1\sqsubseteq t_3$ by transitivity of $\subseteq$.
      Similarly if $t_1$, $t_2$ or $t_3$ is a mutable borrow.
      \item (Induction step) Suppose that $t_1=\boxtype{t'_1}$,
      $t_2=\boxtype{t'_2}$ and $t_3=\boxtype{t'_3}$ for types
      $t'_1,t'_2,t'_3$ such that
      $(t'_1 \sqsubseteq t'_2 \land t'_2 \sqsubseteq t'_3) \Rightarrow
      t'_1 \sqsubseteq t'_3$.
      As $t_1 \sqsubseteq t_2$ and $t_2 \sqsubseteq t_3$, we have
      $t'_1 \sqsubseteq t'_2$ and $t'_2 \sqsubseteq t'_3$, so
      $t'_1 \sqsubseteq t'_3$, hence $t_1 \sqsubseteq t_3$.
    \end{itemize}
  \end{itemize}
  \qed
\end{proof}

\begin{definition}
  The $\sqcap$ operator over $\mathsf{T}_\kappa$ is defined as
  \begin{align*}
    \mathsf{int}\sqcap\mathsf{int} &= \mathsf{int}\\
    \borrow\{\mathsf{w}_1,\ldots,\mathsf{w}_n\} \sqcap \borrow\{\mathsf{w}'_1,\ldots,\mathsf{w}'_{n'}\} &= \borrow(\{\mathsf{w}_1,\ldots,\mathsf{w}_n\}\cap\{\mathsf{w}'_1,\ldots,\mathsf{w}'_{n'}\})\\
    \mutborrow\{\mathsf{w}_1,\ldots,\mathsf{w}_n\} \sqcap \mutborrow\{\mathsf{w}'_1,\ldots,\mathsf{w}'_{n'}\} &= \mutborrow(\{\mathsf{w}_1,\ldots,\mathsf{w}_n\}\cap\{\mathsf{w}'_1,\ldots,\mathsf{w}'_{n'}\})\\
    (\boxtype{t_1}) \sqcap (\boxtype{t_2}) &=\boxtype{(t_1\sqcap t_2)}\quad\text{if $t_1\sqcap t_2$ is defined}\\
    \mathsf{dangling}\sqcap\mathsf{dangling} &= \mathsf{dangling}.
  \end{align*}
  In all other cases, $t_1\sqcap t_2$ is undefined.
\end{definition}

\begin{lemma}\label{lemma:glb-type}
  Let $I\subseteq\mathbb{N}$ and $\{t_i\}_{i\in I}\subseteq\mathsf{T}_\kappa$.
  Then, if $\sqcap_{i\in I}t_i$ is defined, it is the greatest lower bound of $\{t_i\}_{i\in I}$.
\end{lemma}
\begin{proof}
  Assume that $\mu_I = \sqcap_{i\in I}t_i$ is defined.
  We proceed by structural induction.
  \begin{itemize}
    \item (Base: there is a $t_k$ which is not a box)
    If $t_k=\mathsf{int}$ then, as $\mu_I$ is defined, for all $i\in I$
    we have $t_i=\mathsf{int}$. Therefore, $\mu_I=\mathsf{int}$
    and so $\mu_I$ is a lower bound of $\{t_i\}_{i\in I}$.
    Similarly if $t_k=\mathsf{dangling}$.
    If $t_k$ is a borrow then, for all $i\in I$, $t_i$ is a borrow, and so
    $\mu_I$ is a borrow which results from the intersection of the sets of
    leftvalues of all the $t_i$'s. Hence, $\mu_I$ is a lower bound of
    $\{t_i\}_{i\in I}$.
    Similarly if $t_k$ is a mutable borrow.

    Now, suppose that $\mu'_I$ is also a lower bound of $\{t_i\}_{i\in I}$.
    If $\mu'_I=\mathsf{int}$ then each $t_i$ is $\mathsf{int}$, so
    $\mu_I=\mathsf{int}$ and we have $\mu'_I\sqsubseteq\mu_I$.
    Similarly if $\mu'_I=\mathsf{dangling}$.
    If $\mu'_I=\borrow b$, then for all $i\in I$ we have $t_i = \borrow b_i$
    and $b \subseteq b_i$. Hence, $b \subseteq \cap_{i\in I}b_i$ and so
    $\borrow b \sqsubseteq \borrow\cap_{i\in I}b_i$. Consequently, we have
    $\mu'_I\sqsubseteq\mu_I$ because $\mu_I=\borrow\cap_{i\in I}b_i$ in this case.
    Similarly if $\mu'_I$ is a mutable borrow.
    %
    \item (Induction step) Suppose that for all $i\in I$ we have
    $t_i = \boxtype{t'_i}$. Since $\mu_I$ is defined, also
    $\sqcap_{i\in I}t'_i$ is defined and, by inductive hypothesis, it must be the greatest
    lower bound of $\{t'_i\}_{i\in I}$. Then,
    $\mu_I = \boxtype{(\sqcap_{i\in I}t'_i)}$ is a lower bound
    of $\{t_i\}_{i\in I}$. Suppose that $\mu'_I$ is also a lower bound
    of $\{t_i\}_{i\in I}$. Then, $\mu'_I=\boxtype{t}$ with
    $t\sqsubseteq t'_i$ for all $i\in I$, and so $t$ is a lower bound
    of $\{t'_i\}_{i\in I}$; hence, we have $t\sqsubseteq \sqcap_{i\in I}t'_i$
    because $\sqcap_{i\in I}t'_i$ is the greatest lower bound and,
    consequently, $\mu'_I\sqsubseteq\mu_I$.
  \end{itemize}
  \qed
\end{proof}

\begin{lemma}\label{lemma:technical-type}
  Let $t\in\mathsf{T}_\kappa$. For any $t_1,t_2\in\mathsf{T}_\kappa$,
  if $t\sqsubseteq t_1$ and $t\sqsubseteq t_2$ then $t_1\sqcap t_2$ is defined.
\end{lemma}
\begin{proof}
  We proceed by structural induction on $t$.
  \begin{itemize}
    \item (Base: $t$ is not a box)
    Let $t_1,t_2\in\mathsf{T}_\kappa$ with $t\sqsubseteq t_1$ and $t\sqsubseteq t_2$.
    If $t$ is $\mathsf{int}$ (resp. a borrow, a mutable borrow or $\mathsf{dangling}$)
    then, necessarily, $t_1$ and $t_2$ are $\mathsf{int}$ (resp. a borrow,
    a mutable borrow or $\mathsf{dangling}$), hence $t_1\sqcap t_2$ is defined.
    %
    \item (Induction step) Suppose that $t = \boxtype{t'}$.
    Let $t_1,t_2\in\mathsf{T}_\kappa$ with $t\sqsubseteq t_1$ and $t\sqsubseteq t_2$.
    Then, necessarily, $t_1 = \boxtype{t'_1}$ and $t_2 = \boxtype{t'_2}$
    with $t'\sqsubseteq t'_1$ and $t'\sqsubseteq t'_2$. By induction
    hypothesis, $t'_1\sqcap t'_2$ is defined, so
    $(\boxtype{t'_1}) \sqcap (\boxtype{t'_2}) = t_1 \sqcap t_2$ is defined.
    \qed
  \end{itemize}
\end{proof}

\begin{definition}[Typing]\label{def:typing}
  Given a context $\kappa$, a \emph{typing} over $\kappa$ is
  a finite set of bindings from variables to types, decorated with a lifetime:
  \[
  \{x\to^m t\mid x\to^m\in\kappa\text{ and }t\in\mathsf{T_\kappa}\}
  \]
  such that exactly one binding exists in for each given variable in $\dom(\kappa)$.
\end{definition}

\begin{definition}[Dependencies between leftvalues]\label{def:dependencies}
  Given a context $\kappa$ and a typing $a$ over $\kappa$, the \emph{dependencies between leftvalues}
  induced by $a$ are the relation $\gg$ defined as
  \[
  \mathsf{closure}(\{\mathtt{*}\mathsf{w}\gg\mathsf{w}\mid\mathsf{w}\in\Leftvalues_\kappa\}
  \cup\bigcup\limits_{x\to^mt\in a}\mathsf{dependencies}(x, t, a))
  \]
  where
  \begin{align*}
    \mathsf{dependencies}(\mathsf{w},\mathsf{int},a)&=\varnothing\\
    \mathsf{dependencies}(\mathsf{w},\mathsf{dangling},a)&=\varnothing\\
    \mathsf{dependencies}(\mathsf{w},\boxtype{t},a)&=\mathsf{dependencies}(\mathtt{*}\mathsf{w},t,a)\\
    \mathsf{dependencies}(\mathsf{w},\borrow\{\mathsf{w}_1,\ldots,\mathsf{w}_n\},a)&=\{\mathtt{*}\mathsf{w}\gg\mathsf{w}_i\mid 1\le i\le n\}\\
    \mathsf{dependencies}(\mathsf{w},\mutborrow\{\mathsf{w}_1,\ldots,\mathsf{w}_n\},a)&=\{\mathtt{*}\mathsf{w}\gg\mathsf{w}_i\mid 1\le i\le n\}.
  \end{align*}
  and
  \[
  \mathsf{closure}(R)=R\cup\left\{\underbrace{\mathtt{*}\cdots\mathtt{*}}_{n}\mathsf{w}_1\gg\mathsf{w}_3\left|
  \begin{array}{l}
    \mathsf{w}_1\gg\mathsf{w}_2\in R,\ \underbrace{\mathtt{*}\cdots\mathtt{*}}_{n\ge 0}\mathsf{w}_2\gg\mathsf{w}_3\in R\\
    \text{and $\mathsf{w_3}$ is a borrow that occurs in $a$}
  \end{array}\right.\right\}.
  \]
\end{definition}

\noindent
Note that the closure in Def.~\ref{def:dependencies} makes $\gg$ transitive.

\begin{definition}[Abstract location for reading]\label{def:locreada}
  Given a context $\kappa$, a typing $a$ over $\kappa$
  and $\mathsf{w}\in\Leftvalues_\kappa$, the partial function
  $\locreada(\mathsf{w},a)$ yields the type of the location that holds the value of $\mathsf{w}$
  and a Boolean mark that informs if the evaluation of $\mathsf{w}$ passes through a borrow:
  \begin{align*}
    \locreada(x,a)&=\langle a(x),\mathit{false}\rangle\\
    \locreada(*\mathsf{w},a)&=\begin{cases}
    \text{undefined} & \text{if $\locreada(\mathsf{w},a)$ is undefined}\\
    \text{undefined} & \text{if $\locreada(\mathsf{w},a)=\langle\mathsf{dangling},\_\rangle$}\\
    \text{undefined} & \text{if $\locreada(\mathsf{w},a)=\langle\mathsf{int},\_\rangle$}\\
    \langle\sqcup_{1\le i\le n}t_i,\mathit{true}\rangle & \text{if $\locreada(\mathsf{w},a)=\langle\borrow\{\mathsf{w}_1,\ldots,\mathsf{w}_n\},\_\rangle$}\\
    & \quad\text{and $\locreada(\mathsf{w}_i,a)=\langle t_i,\_\rangle$}\\
    \langle\sqcup_{1\le i\le n}t_i,\mathit{true}\rangle & \text{if $\locreada(\mathsf{w},a)=\langle\mutborrow\{\mathsf{w}_1,\ldots,\mathsf{w}_n\},\_\rangle$}\\
    & \quad\text{and $\locreada(\mathsf{w}_i,a)=\langle t_i,\_\rangle$}\\
    \langle t,b\rangle & \text{if $\locreada(\mathsf{w},a)=\langle\boxtype{t},b\rangle$}\\
    \text{undefined} & \text{otherwise.}
    \end{cases}
  \end{align*}
\end{definition}

\noindent
Note that the defintion of $\locreada$ recurs in a way that, in general,
is not well-founded, for the cases of $\borrow$ and $\mutborrow$. Moreover, the $\sqcup$
operator used in the same cases might not be defined. Intuitively, these situations
represent a type error and we can discard them, since we only consider
programs that type-check correctly. Hence the sense of the restrictions that
Def.~\ref{def:abstract-store} will require.

\begin{proposition}\label{prop:acyclicity}
  Given a context $\kappa$ and a typing $a$ over $\kappa$
  such that the dependencies between leftvalues induced by $a$ are acyclical,
  the recursion used in the definition of function $\locreada$ is well-founded and consistent
  with $\gg$.
\end{proposition}
\begin{proof}
  First we prove that $\gg$ is a well-founded relation. Assume the contrary. Then
  there is an infinite sequence of leftvalues
  $\mathsf{w}_1\gg\mathsf{w}_2\gg\cdots\gg\mathsf{w}_n\gg\cdots$.
  Since $\mathsf{w}\gg\mathsf{w}'\in\mathsf{dependencies}(x,t,a)$ entails that
  $\mathsf{w}'$ is one of the leftvalues that occur in the borrows of $a$ and
  since there is only a finite number of such $\mathsf{w}'$, the acyclicity of $\gg$ entails that
  there must be a $\mathsf{w}_k=\underbrace{\mathtt{*}\cdots\mathtt{*}}_{\text{$n$}}x$
  such that the subsequent $\mathsf{w}_i$, $i>k$,
  are not leftvalues that occur in the borrows in $a$. By Def.~\ref{def:dependencies},
  it can only be $\mathsf{w}_{k+i}=\underbrace{\mathtt{*}\cdots\mathtt{*}}_{\text{$\le n-i$}}x$
  and consequently the length of the sequence is $k+n$ at most, impossible since we assumed that
  it was infinite.

  We now prove that
  \begin{enumerate}
  \item for every $\mathsf{w}\in\Leftvalues_\kappa$, the
    recursive uses $\locreada(\mathsf{w}',a)$ that occur for the definition
    of $\locreada(\mathsf{w},a)$ are such that $\mathsf{w}\gg\mathsf{w}'$;
  \item when $\locreada(\mathsf{w},a)=\langle\borrow\{\mathsf{w}_1,\ldots,\mathsf{w}_n\},\_\rangle$
    or $\locreada(\mathsf{w},a)=\langle\mutborrow\{\mathsf{w}_1,\ldots,\mathsf{w}_n\},\_\rangle$
    then $\mathtt{*}\mathsf{w}\gg\mathsf{w}_i$ for every $1\le i\le n$.
  \end{enumerate}
  Note that (1) by itself means that
  the recursion used in the
  definition of function $\locreada$ is well-founded and consistent with $\gg$, but we will also
  need (2) in order to prove (1).
  We prove (1) and (2) by induction on $\mathsf{w}$ with respect to $\gg$.

  \begin{itemize}
  \item (Base case)
    If $\mathsf{w}$
    has no $\mathsf{w}'$ such that $\mathsf{w}\gg\mathsf{w}'$, then it must be $\mathsf{w}\in\Vars$.
    Hence there are no recursive uses of $\locreada$ in the definition of
    $\locreada(\mathsf{w},a)$ and (1) holds. Moreover, in this case
    $\locreada(\mathsf{w},a)=\langle a(\mathsf{w}),\mathsf{false}\rangle$ and if
    $a(\mathsf{w})=\borrow\{\mathsf{w}_1,\ldots,\mathsf{w}_n\}$ by Def.~\ref{def:dependencies}
    we conclude that $\mathtt{*}\mathsf{w}\gg\mathsf{w}_i$ for every $1\le i\le n$, hence (2) holds.
    Similarly when $a(\mathsf{w})=\mutborrow\{\mathsf{w}_1,\ldots,\mathsf{w}_n\}$.
  \item (Inductice case)
    Assume now that (1) and (2) hold for $\mathsf{w}$. Consider
    the recursive uses $\locreada(\mathsf{w}',a)$ in the definition of
    $\locreada(\mathtt{*}\mathsf{w},a)$. One such recursive use is
    $\locreada(\mathsf{w},a)$ and $\mathtt{*}\mathsf{w}\gg\mathsf{w}$.
    Others are inside the computation of $\locreada(\mathsf{w},a)$ and by inductive hypothesis
    (1) holds, that is, they occur on $\mathsf{w}'$ such that $\mathsf{w}\gg\mathsf{w}'$.
    Hence $\mathtt{*}\mathsf{w}\gg\mathsf{w}\gg\mathsf{w}'$ and by transitivity
    $\mathtt{*}\mathsf{w}\gg\mathsf{w}'$. Finally, there are recursive uses
    when $\locreada(\mathsf{w},a)=\langle\borrow\{\mathsf{w}_1,\ldots,\mathsf{w}_n\},\_\rangle$
    or when $\locreada(\mathsf{w},a)=\langle\mutborrow\{\mathsf{w}_1,\ldots,\mathsf{w}_n\},\_\rangle$
    namely, uses of
    $\locreada(\mathsf{w}_i,a)$ with $1\le i\le n$. By inductive hypothesis, we know that
    (2) holds for $\mathsf{w}$, that is, $\mathtt{*}\mathsf{w}\gg\mathsf{w}_i$ for every
    $1\le i\le n$. This concludes the inductive case for the proof of (1).
    Let us prove the (2) for $\mathtt{*}\mathsf{w}$ now. Assume then that
    $\locreada(\mathtt{*}\mathsf{w},a)=\langle\borrow\{\mathsf{w}_1,\ldots,\mathsf{w}_n\},\_\rangle$
    (the case
    $\locreada(\mathtt{*}\mathsf{w},a)=\langle\mutborrow\{\mathsf{w}_1,\ldots,\mathsf{w}_n\},\_\rangle$
    is similar).
    By Def.~\ref{def:locreada}, there are two possibilities:
    \begin{itemize}
    \item $\mathtt{*}\mathsf{w}=\underbrace{\mathtt{*}\cdots\mathtt{*}}_{\text{$m+1$}}x$
      and $a(x)=\underbrace{\boxempty\cdots\boxempty}_{\text{$m+1$}}
      \borrow\{\mathsf{w}_1,\ldots,\mathsf{w}_n\}$
      for some $m\ge 0$, where $x=\mathsf{root}(\mathsf{w})$.
      By Def.~\ref{def:dependencies}, the relation $\gg$ includes
      \begin{align*}
        \mathsf{dependencies}(x,a(x),a)&=\mathsf{dependencies}(x,\underbrace{\boxempty\cdots\boxempty}_{\text{$m+1$}}\borrow\{\mathsf{w}_1,\ldots,\mathsf{w}_n\},a)\\
        &=\mathsf{dependencies}(\underbrace{\mathtt{*}\cdots\mathtt{*}}_{\text{$m+1$}}x,
        \borrow\{\mathsf{w}_1,\ldots,\mathsf{w}_n\},a)\\
        &=\mathsf{dependencies}(\mathtt{*}\mathsf{w},\borrow\{\mathsf{w}_1,\ldots,\mathsf{w}_n\},a)\\
        &=\{\mathtt{**}\mathsf{w}\gg\mathsf{w}_i\mid 1\le i\le n\}
      \end{align*}
      and (2) holds for $\mathtt{*}\mathsf{w}$.
    \item $\borrow\{\mathsf{w}_1,\ldots,\mathsf{w}_n\}=\sqcup_{1\le i\le n'}\locreada(\mathsf{w}_i',a)$
      with $\locreada(\mathsf{w},a)=\langle\borrow\{\mathsf{w}'_1,\ldots,\mathsf{w}'_{n'}\},\_\rangle$.
      The only possibility is that $\locreada(\mathsf{w}_j',a)=\borrow W_j$ for every $1\le j\le n'$,
      with $\{\mathsf{w}_1,\ldots,\mathsf{w}_n\}=\cup_{1\le j\le n'}W_i$.
      By inductive hypothesis of (2), we know that
      $\mathtt{*}\mathsf{w}\gg\mathsf{w}_j'$ for every $1\le j\le n'$ and that
      $\mathtt{*}\mathsf{w}_j'\gg\mathsf{w}''$ for every $\mathsf{w}''\in W_j$ and
      every $1\le j\le n'$. Note that $\mathsf{w}''$ is one of the leftvalues that occur
      in the borrows of $a$, by Def.~\ref{def:locreada}. By closure (Def.~\ref{def:dependencies}) we conclude that
      $\mathtt{**}\mathsf{w}\gg\mathsf{w}''$ for every $\mathsf{w}''\in W_j$ and
      every $1\le j\le n'$. Since each $\mathsf{w}_i$ belongs to some $W_j$, we conclude that
      $\mathtt{**}\mathsf{w}\gg\mathsf{w}_i$ for every $1\le i\le n$ and (2) holds for $\mathtt{*}\mathsf{w}$.
    \end{itemize}
  \end{itemize}
\end{proof}

\begin{definition}[Abstract stores]\label{def:abstract-store}
  Given a context $\kappa$, an \emph{abstract store} $a$ over $\kappa$ is either $\bot$ or $\top$
  or a typing over $\kappa$ such that
  \begin{itemize}
  \item the dependencies between leftvalues induced by $a$ are acyclical;
  \item for each $\borrow\{\mathsf{w}_1,\ldots,\mathsf{w}_n\}$
    or $\mutborrow\{\mathsf{w}_1,\ldots,\mathsf{w}_n\}$ that occurs in
    $a$, the type $\sqcup_{1\le i\le n}t_i$ is defined,
    where $\locreada(\mathsf{w}_i,a)=\langle t_i,\_\rangle$.
  \end{itemize}
\end{definition}

\begin{definition}
  The greatest lower bound operator over $\mathbb{AS}_\kappa$ is defined as
  $\bot\sqcap a=\bot$, $a\sqcap \bot=\bot$, $\top\sqcap a=a$, $a\sqcap\top=a$ and,
  when $a_1,a_2\not=\bot,\top$:
  \[
  a_1\sqcap a_2=\begin{cases}
  \bot\qquad\text{if $\exists x\to^m t_1\in a_1$ and $x\to^m t_2\in a_2$ s.t.\ $t_1\sqcap t_2$ is undefined}\\
  \{x\to^m(t_1\sqcap t_2)\mid x\to^m t_1\in a_1\text{ and }x\to^m t_2\in a_2\}\qquad\text{otherwise.}
  \end{cases}
  \]
\end{definition}

\begin{proposition}\label{prop:complete_lattice}
  The abstract domain $\mathbb{AS}_\kappa$ is a complete lattice \wrt $\sqsubseteq$,
  with $\sqcap$ as greatest lower bound operator.
\end{proposition}
\begin{proof}
  % \issue{TODO}{All this proof should be checked}
  First we prove that $(\mathbb{AS}_\kappa,\sqsubseteq)$ is a partially ordered set.
  \begin{itemize}
    \item (Reflexivity) Let $a\in\mathbb{AS}_\kappa$. If $a=\top$ or $a=\bot$, then
    $a\sqsubseteq a$.
    Otherwise, for every $x\to^mt\in a$ there is $x\to^mt\in a$ with
    $t\sqsubseteq t$ (see Lem.~\ref{lemma:partial-order-types}), hence we have
    $a\sqsubseteq a$.
    %
    \item (Antisymmetry) Suppose that $a_1 \sqsubseteq a_2$ and
    $a_2 \sqsubseteq a_1$. If $a_1=\top$, then, as $a_1 \sqsubseteq a_2$,
    we necessarily have $a_2=\top$, so $a_1=a_2$.
    If $a_1=\bot$ then, as $a_2\sqsubseteq a_1$, we necessarily have $a_2=\bot$,
    so $a_1=a_2$.
    Identically, if $a_2=\top$ or $a_2=\bot$ then we have $a_1=a_2$.
    Now, suppose that $a_1,a_2\not=\top,\bot$. As $a_1 \sqsubseteq a_2$, for every
    $x\to^mt_1\in a_1$ there is $x\to^mt_2\in a_2$ such that
    $t_1\sqsubseteq t_2$ and, as $a_2 \sqsubseteq a_1$, there is
    $x\to^mt'_1\in a_1$ such that $t_2\sqsubseteq t'_1$.
    As in $a_1$ there is exactly one binding for $x\in\dom(\kappa)$, we have
    $t_1=t'_1$. Consequently, $t_1\sqsubseteq t_2$ and $t_2\sqsubseteq t_1$
    so, by Lem.~\ref{lemma:partial-order-types}, $t_1=t_2$. So, we have
    proved that for every $x\to^mt_1\in a_1$ we have $x\to^mt_1\in a_2$.
    Identically, by considering $a_2 \sqsubseteq a_1$, we prove that
    for every $x\to^mt_2\in a_2$ we have $x\to^mt_2\in a_1$.
    Therefore, $a_1=a_2$.
    %
    \item (Transitivity) Suppose that $a_1 \sqsubseteq a_2$ and
    $a_2 \sqsubseteq a_3$.
    If $a_1=\bot$ then $a_1\sqsubseteq a_3$.
    If $a_1=\top$ then $a_2=\top$ and so $a_3=\top$, hence $a_1\sqsubseteq a_3$.
    If $a_2=\bot$ then $a_1=\bot$, so $a_1\sqsubseteq a_3$.
    If $a_2=\top$ then $a_3=\top$, hence $a_1\sqsubseteq a_3$.
    If $a_3=\bot$ then $a_2=\bot$ and so $a_1=\bot$, hence $a_1\sqsubseteq a_3$.
    If $a_3=\top$ then $a_1\sqsubseteq a_3$.
    Now suppose that $a_1,a_2,a_3\not=\top,\bot$.
    Then, for every $x\to^mt_1\in a_1$ there is $x\to^mt_2\in a_2$ such that
    $t_1\sqsubseteq t_2$ and there is $x\to^mt_3\in a_3$ such that
    $t_2\sqsubseteq t_3$. Hence, by Lem.~\ref{lemma:partial-order-types},
    we have $t_1\sqsubseteq t_3$. Therefore, we have $a_1\sqsubseteq a_3$.
  \end{itemize}

  Let $I\subseteq\mathbb{N}$ and $\{a_i\}_{i\in I}\subseteq\mathbb{AS}_\kappa$.
  We prove that $\mu_I=\sqcap_{i\in I}a_i$ is the greatest lower bound of
  $\{a_i\}_{i\in I}$.
  \begin{itemize}
    \item First, we prove that $\mu_I$ is a lower bound.
    Let $k\in I$.
    If $a_k=\top$ then we have $\mu_I \sqsubseteq a_k$.
    If $a_k=\bot$ then $\mu_I=\bot$ and $\mu_I\sqsubseteq a_k$.
    If $\mu_I = \top$ then $\{a_i\}_{i\in I}=\{\top\}$, so we have $a_k=\top$,
    hence $\mu_I \sqsubseteq a_k$.
    If $\mu_I=\bot$ then $\mu_I\sqsubseteq a_k$.
    Otherwise (\textit{ie.} $\mu_I,a_k\neq\top,\bot$),
    let $x\to^m t\in \mu_I$. Then, for all $i\in I$ either
    $a_i=\top$ or there is a binding $x\to^m t_i$ in $a_i$, and we have
    $t = \sqcap_{i\in J}t_i$ where $J=\{i\in I\mid a_i\neq\top\}$. In particular,
    there is a binding $x\to^m t_k$ in $a_k$ and, by Lem.~\ref{lemma:glb-type},
    $\sqcap_{i\in J}t_i \sqsubseteq t_k$. Consequently, we have
    $\mu_I \sqsubseteq a_k$.
    \item Now, we prove that $\mu_I$ is the greatest.
    Suppose that $\mu'_I$ is also a lower bound of $\{a_i\}_{i\in I}$.
    If $\mu'_I=\top$ then necessarily $\{a_i\}_{i\in I}=\{\top\}$, so
    we have $\mu_I=\top$, hence $\mu'_I \sqsubseteq \mu_I$.
    If $\mu'_I=\bot$ or $\mu_I=\top$ then $\mu'_I \sqsubseteq \mu_I$.
    Otherwise, if $\mu_I=\bot$ then either $\bot\in\{a_i\}_{i\in I}$ and so
    $\mu'_I=\bot$, hence $\mu'_I \sqsubseteq \mu_I$; or, for some
    $a_i,a_j$ there is  $x\to^m t_i\in a_i$ and $x\to^m t_j\in a_j$ s.t.\
    $t_i\sqcap t_j$ is undefined; by Lem.~\ref{lemma:technical-type}, this
    is not possible, because there would be exactly one binding
    $x\to^m t\in\mu'_I$ and we would have $t \sqsubseteq t_i$ and
    $t \sqsubseteq t_j$.
    Otherwise (\textit{ie.} $\mu'_I,\mu_I\neq\bot,\top$),
    let $x\to^m t\in \mu'_I$. As $\mu'_I$ is a lower bound of $\{a_i\}_{i\in I}$,
    for all $i\in I$ either $a_i=\top$ or there is $x\to^m t_i\in a_i$ such
    that $t\sqsubseteq t_i$, so $t$ is a lower bound of $\{t_i\}_{i\in J}$
    where $J=\{i\in I\mid a_i\neq\top\}$.
    As $\mu_I\neq\bot$, $\sqcap_{i\in J}t_i$ is defined, so by Lem.~\ref{lemma:glb-type}
    we have $t\sqsubseteq \sqcap_{i\in J}t_i$. Note that
    $x\to^m \sqcap_{i\in J}t_i\in \mu_I$. Therefore, we have
    $\mu'_I \sqsubseteq \mu_I$.
  \end{itemize}

  Finally, $\top$ is the top element of $\mathbb{AS}_\kappa$
  since for every $a\in\mathbb{AS}_\kappa$ we have $a\sqsubseteq\top$.
  Hence there is a least upper bound operator, \issue{induced by $\sqcap$}{Will add reference later},
  and $\mathbb{AS}_\kappa$ is a complete lattice.
  \qed
\end{proof}

\begin{definition}[Value compatibility]\label{def:value_compatibility}
  Let $v\in\mathbb{V}$ be a value, $\rho\star h\in\mathbb{S}_\kappa$ and
  $t\in\mathsf{T}_\kappa$ not a box. Then $v$ is \emph{compatible} with $t$ in $\rho\star h$,
  that we write as $v\sim_{\rho\star h}t$, is defined as
  \begin{align*}
    v&\sim_{\rho\star h}\mathsf{int} && \text{iff $v\in\mathbb{Z}$}\\
    v&\sim_{\rho\star h}\borrow\{\mathsf{w}_1,\ldots,\mathsf{w}_n\} && \text{iff $v\in\Locs$ and}\\
    &&& \exists i\in\{1,\ldots,n\}.\loc(\mathsf{w}_i,\rho\star h)=v\text{ and}\\
    &&& \forall\mathsf{w}\in\Leftvalues.\left(\loc(\mathsf{w},\rho\star h)=v\text{ implies }\mathsf{w}\in\{\mathsf{w}_1,\ldots,\mathsf{w}_n\}\right)\\
    v&\sim_{\rho\star h}\mutborrow\{\mathsf{w}_1,\ldots,\mathsf{w}_n\} && \text{iff $v\in\Locs$ and}\\
    &&& \exists i\in\{1,\ldots,n\}.\loc(\mathsf{w}_i,\rho\star h)=v\text{ and}\\
    &&& \forall\mathsf{w}\in\Leftvalues.\left(\loc(\mathsf{w},\rho\star h)=v\text{ implies }\mathsf{w}\in\{\mathsf{w}_1,\ldots,\mathsf{w}_n\}\right)\\
    v&\sim_{\rho\star h}\mathsf{dangling} && \text{never.}
  \end{align*}
\end{definition}

\begin{definition}[Concretization map]\label{def:concretization}
  An abstract store $a\in\mathbb{AS}_\kappa$ represents a set of concrete stores
  in $\mathbb{S}_\kappa$, according to the \emph{concretization map}
  $\gamma_\kappa:\mathbb{AS}_\kappa\to\wp(\mathbb{S}_\kappa)$, defined as
  $\gamma_\kappa(\bot)=\varnothing$, $\gamma_\kappa(\top)=\wp(\mathbb{S}_\kappa)$ and,
  for $a\not=\bot,\top$:
  \[
  \gamma_\kappa(a)=\left\{\rho\star h\in\mathbb{S}_\kappa\left|\begin{array}{l}
  \text{for every }x\to^m\ell\in\rho\text{ there is }x\to^m t\in a\text{ s.t.\ }\ell\approx_{\rho\star h}t
  \end{array}
  \right.\right\}
  \]
  where $\ell\approx_{\rho\star h}t$ is defined as
  \begin{align*}
    \ell&\approx_{\rho\star h}\mathsf{int} && \text{iff $h(\ell)\sim_{\rho\star h}\mathsf{int}$}\\
    \ell&\approx_{\rho\star h}\borrow\{\mathsf{w}_1,\ldots,\mathsf{w}_n\} && \text{iff $\exists\ell'.\ell\to\ell'\in h$ and $\ell'\sim_{\rho\star h}\borrow\{\mathsf{w}_1,\ldots,\mathsf{w}_n\}$}\\
    \ell&\approx_{\rho\star h}\mutborrow\{\mathsf{w}_1,\ldots,\mathsf{w}_n\} && \text{iff $\exists\ell'.\ell\leadsto\ell'\in h$ and $\ell'\sim_{\rho\star h}\mutborrow\{\mathsf{w}_1,\ldots,\mathsf{w}_n\}$}\\
    \ell&\approx_{\rho\star h}\boxtype{t} && \text{iff $\exists\ell'.\ell\Rightarrow\ell'\in h$ and $\ell'\approx_{\rho\star h}t$}\\
    \ell&\approx_{\rho\star h}\mathsf{dangling} && \text{iff $h(\ell)$ is undefined.}
  \end{align*}
\end{definition}


\begin{lemma}\label{lem:co-additive-type}
  Let $I\subseteq\mathbb{N}$ and $\{t_i\}_{i\in I}\subseteq\mathsf{T}_\kappa$.
  For any $\rho\star h\in\mathbb{S}_\kappa$ and $\ell\in\Locs$,
  \[(\sqcap_{i\in I}t_i \text{ is defined} \land
  \ell\approx_{\rho\star h}\sqcap_{i\in I}t_i)
  \iff \forall i\in I\ \ell\approx_{\rho\star h}t_i\]
\end{lemma}
\begin{proof}
  % \issue{TODO}{Fix Def. 18 (borrows)}
  We proceed by structural induction.
  \begin{itemize}
    \item (Base: there is a $t_k$ which is not a box)
    Let $\rho\star h\in\mathbb{S}_\kappa$ and $\ell\in\Locs$.
    \begin{itemize}
      \item Suppose that $\sqcap_{i\in I}t_i$ is defined and
      $\ell\approx_{\rho\star h}\sqcap_{i\in I}t_i$.
      If $t_k$ is $\mathsf{int}$ (resp. $\mathsf{dangling}$) then,
      as $\sqcap_{i\in I}t_i$ is defined, each $t_i$ is $\mathsf{int}$
      (resp. $\mathsf{dangling}$) and we have $\sqcap_{i\in I}t_i=\mathsf{int}$
      (resp. $\sqcap_{i\in I}t_i=\mathsf{dangling}$).
      Moreover, as $\ell\approx_{\rho\star h}\sqcap_{i\in I}t_i$, we have
      $h(\ell)\in\mathbb{Z}$ (resp. $h(\ell)$ is undefined). Consequently,
      for all $i\in I$ we have $\ell\approx_{\rho\star h}t_i$.
      If $t_k$ is a borrow (resp. mutable borrow) then, as $\sqcap_{i\in I}t_i$
      is defined, each $t_i$ is a borrow (resp. mutable borrow) and
      $\sqcap_{i\in I}t_i=\borrow\{\mathsf{w}_1,\ldots,\mathsf{w}_n\}$
      (resp. $\sqcap_{i\in I}t_i=\mutborrow\{\mathsf{w}_1,\ldots,\mathsf{w}_n\}$)
      is the intersection of the sets of leftvalues of all the $t_i$'s. Moreover, as
      $\ell\approx_{\rho\star h}\sqcap_{i\in I}t_i$, there is a location $\ell'$
      s.t. $\ell\to\ell'\in h$ (resp. $\ell\leadsto\ell'\in h$) and
      for all leftvalue $\mathsf{w}$, $\loc(\mathsf{w},\rho\star h)=\ell'$
      implies $\mathsf{w}\in\{\mathsf{w}_1,\ldots,\mathsf{w}_n\}$. Consequently,
      for all $i\in I$ we have $\ell\approx_{\rho\star h}t_i$ because the set
      of leftvalues of $t_i$ includes $\{\mathsf{w}_1,\ldots,\mathsf{w}_n\}$.
      \item Now, suppose that for all $i\in I$ we have $\ell\approx_{\rho\star h}t_i$.
      If $t_k$ is $\mathsf{int}$ (resp. $\mathsf{dangling}$) then
      $h(\ell)\in\mathbb{Z}$ (resp. $h(\ell)$ is undefined). Consequently,
      for all $i\in I$ we have $t_i = \mathsf{int}$ (resp. $t_i = \mathsf{dangling}$).
      So, $\sqcap_{i\in I}t_i = \mathsf{int}$
      (resp. $\sqcap_{i\in I}t_i=\mathsf{dangling}$) is defined and
      $\ell\approx_{\rho\star h}\sqcap_{i\in I}t_i$.
      If $t_k=\borrow\{\mathsf{w}_{k,1},\ldots,\mathsf{w}_{k,n_k}\}$
      then there is a location $\ell'$ s.t. $\ell\to\ell'\in h$
      and for all leftvalue $\mathsf{w}$,
      $\loc(\mathsf{w},\rho\star h)=\ell'$
      implies $\mathsf{w}\in\{\mathsf{w}_{k,1},\ldots,\mathsf{w}_{k,n_k}\}$.
      %
      So, for all $i\in I$, $t_i$ has the form
      $\borrow\{\mathsf{w}_{i,1},\ldots,\mathsf{w}_{i,n_i}\}$
      and for all leftvalue $\mathsf{w}$, $\loc(\mathsf{w},\rho\star h)=\ell'$
      implies $\mathsf{w}\in\{\mathsf{w}_{i,1},\ldots,\mathsf{w}_{i,n_i}\}$. So,
      $\sqcap_{i\in I}t_i = \borrow(\cap_{i\in I}\{\mathsf{w}_{i,1},\ldots,\mathsf{w}_{i,n_i}\})$
      is defined and $\ell\approx_{\rho\star h}\sqcap_{i\in I}t_i$.
      Similarly if $t_k=\mutborrow\{\mathsf{w}_{k,1},\ldots,\mathsf{w}_{k,n_k}\}$.
    \end{itemize}
    %
    \item (Induction step) Suppose that $t_i=\boxtype{t'_i}$ for all
    $i\in I$. Let $\rho\star h\in\mathbb{S}_\kappa$ and $\ell\in\Locs$.
    \begin{itemize}
      \item If $\sqcap_{i\in I}t_i$ is defined and
      $\ell\approx_{\rho\star h}\sqcap_{i\in I}t_i$ then
      $\sqcap_{i\in I}t_i = \boxtype{(\sqcap_{i\in I}t'_i)}$ and
      $\sqcap_{i\in I}t'_i$ is defined. Moreover, there is a location $\ell'$
      s.t. $\ell\Rightarrow\ell'\in h$ and
      $h(\ell')\approx_{\rho\star h}\sqcap_{i\in I}t'_i$. Consequently,
      by induction hypothesis, for all $i\in I$ we have
      $h(\ell')\approx_{\rho\star h}t'_i$. Therefore, for all $i\in I$ we have
      $\ell\approx_{\rho\star h}\boxtype{t'_i}$ \ie
      $\ell\approx_{\rho\star h}t_i$.
      \item Now, suppose that for all $i\in I$ we have $\ell\approx_{\rho\star h}t_i$
      \ie $\ell\approx_{\rho\star h}\boxtype{t'_i}$. Then, for all $i\in I$,
      there is a location $\ell'_i$ s.t. $\ell\Rightarrow\ell'_i\in h$ and
      $h(\ell'_i)\approx_{\rho\star h}t'_i$.
      Note that all the $\ell'_i$'s are equal to a same location $\ell'$ because,
      by definition of a heap, there is at most a binding in $h$ for location
      $\ell$. By induction hypothesis, $\sqcap_{i\in I}t'_i$ is defined and
      $h(\ell')\approx_{\rho\star h}\sqcap_{i\in I}t'_i$.
      Consequently, $\sqcap_{i\in I}t_i = \sqcap_{i\in I} (\boxtype{t'_i}) =
      \boxtype{(\sqcap_{i\in I}t'_i)}$ is defined and
      $\ell\approx_{\rho\star h}\boxtype{(\sqcap_{i\in I}t'_i)}$ \ie
      $\ell\approx_{\rho\star h}\sqcap_{i\in I}t_i$.
      \qed
    \end{itemize}
  \end{itemize}
\end{proof}

\begin{lemma}\label{lem:co-additive}
  The map $\gamma_\kappa$ of Def.~\ref{def:concretization} is co-additive.
\end{lemma}
\begin{proof}
  % \issue{TODO}{This proof should be redone}
  Let $I\subseteq\mathbb{N}$ and $\{a_i\}_{i\in I}\subseteq\mathbb{AS}_\kappa$.
  We prove that $\gamma_\kappa(\sqcap_{i\in I}a_i)=\cap_{i\in I}\gamma_\kappa(a_i)$.
  \begin{itemize}
    \item If $\sqcap_{i\in I}a_i=\top$, then for all $i\in I$ we have $a_i=\top$.
    Hence, $\gamma_\kappa(\sqcap_{i\in I}a_i) =
    \gamma_\kappa(\top) =
    \wp(\mathbb{S}_\kappa) =
    \cap_{i\in I}\wp(\mathbb{S}_\kappa) =
    \cap_{i\in I}\gamma_\kappa(\top) =
    \cap_{i\in I}\gamma_\kappa(a_i)$.
    \item Suppose that $\sqcap_{i\in I}a_i=\bot$. Then, either there is $i\in I$ s.t.
    $a_i=\bot$, and so $\gamma_\kappa(\sqcap_{i\in I}a_i) = \varnothing =
    \cap_{i\in I}\gamma_\kappa(a_i)$. Or, for some $a_j,a_k\not=\top,\bot$
    there is $x\to^m t_j\in a_j$ and $x\to^m t_k\in a_k$ s.t.
    $t_j\sqcap t_k$ is undefined. Let us show that
    $\gamma_\kappa(a_j)\cap\gamma_\kappa(a_k)=\varnothing$.
    Let $\rho\star h\in\gamma_\kappa(a_j)$.
    Necessarily, there is a binding $x\to^m\ell$ in $\rho$ and
    we have $\ell\approx_{\rho\star h}t_j$.
    Consequently, by Lem.~\ref{lem:co-additive-type},
    $\ell\approx_{\rho\star h}t_k$ does not hold, so
    $\rho\star h\not\in\gamma_\kappa(a_k)$.
    Finally, we have $\cap_{i\in I}\gamma_\kappa(a_i) = \varnothing =
    \gamma_\kappa(\sqcap_{i\in I}a_i)$.
  \end{itemize}
  From now on, we suppose that $\sqcap_{i\in I}a_i\not=\top,\bot$.
  Then, necessarily, for all $i\in I$ we have $a_i\not=\bot$.
  Moreover, $J=\{i\in I\mid a_i\neq\top\}\not=\varnothing$ and
  $\sqcap_{i\in I}a_i = \sqcap_{i\in J}a_i$ and
  $\cap_{i\in I}\gamma_\kappa(a_i)=\cap_{i\in J}\gamma_\kappa(a_i)$.
  \begin{itemize}
    \item First we prove that
    $\gamma_\kappa(\sqcap_{i\in J}a_i)\subseteq\cap_{i\in J}\gamma_\kappa(a_i)$.
    Let $\rho\star h\in\gamma_\kappa(\sqcap_{i\in J}a_i)$.
    Let $x\to^m\ell\in\rho$. Then, there is $x\to^m t\in \sqcap_{i\in J}a_i$ s.t.
    $\ell\approx_{\rho\star h}t$. By definition of $\sqcap$, for all $i\in J$
    there is $x\to^m t_i\in a_i$, and $t=\sqcap_{i\in J}t_i$ is defined.
    By Lem.~\ref{lem:co-additive-type}, $\ell\approx_{\rho\star h}t_i$
    for all $i\in J$. Therefore, $\rho\star h\in\gamma_\kappa(a_i)$
    for all $i\in J$. Consequently,
    $\rho\star h\in\cap_{i\in J}\gamma_\kappa(a_i)$.
    \item Now we prove that
    $\gamma_\kappa(\sqcap_{i\in J}a_i)\supseteq\cap_{i\in J}\gamma_\kappa(a_i)$.
    Let $\rho\star h\in\cap_{i\in J}\gamma_\kappa(a_i)$.
    Let $x\to^m\ell\in\rho$.
    For all $i\in J$, $\rho\star h\in\gamma_\kappa(a_i)$ and there
    is $x\to^m t_i\in a_i$ s.t. $\ell\approx_{\rho\star h}t_i$.
    % As $\sqcap_{i\in I}a_i\not=\bot$, $\sqcap_{i\in J}t_i$ is defined
    So, by Lem.~\ref{lem:co-additive-type}, $\sqcap_{i\in J}t_i$ is defined
    and $\ell\approx_{\rho\star h}\sqcap_{i\in J}t_i$.
    Moreover, $x\to^m \sqcap_{i\in J}t_i \in \sqcap_{i\in J}a_i$.
    Consequently, we have $\rho\star h\in\gamma_\kappa(\sqcap_{i\in J}a_i)$.
  \end{itemize}
  \qed
\end{proof}

\begin{proposition}\label{prop:abstract_interpretation}
  The domain $\mathbb{AS}_\kappa$ is an abstract interpretation of $\wp(\mathbb{S}_\kappa)$
  with $\gamma_\kappa$ as concretization map.
\end{proposition}
\begin{proof}
  The abstract domain $\mathbb{AS}_\kappa$ is a complete lattice \wrt $\sqsubseteq$
  with $\sqcap$ as greatest lower bound operator (Prop.~\ref{prop:complete_lattice}).
  The domain $\wp(\mathbb{S}_\kappa)$ is a complete lattice \wrt $\subseteq$ with $\cap$ as
  greatest lower bound operator. The map $\gamma_\kappa$ is co-additive (Lem.~\ref{lem:co-additive}).
  The thesis follows by a general result of abstract interpretation~\cite{CousotC77}.
  \qed
\end{proof}

\begin{proposition}[Abstract locations are correct]
  \label{prop:abstract_location_correctness}
  Let $\mathsf{w}\in\Leftvalues$, $s\in\mathbb{S}_\kappa$ be such that
  $\locread(\mathsf{w},s)$ is defined,
  that is, $\locread(\mathsf{w},s)=\langle\ell',p\rangle$ for some $\ell'\in\Locs$
  and some Boolean $p$.
  Let $a\in\mathbb{AS}_\kappa$ be such that $s\in\gamma_\kappa(a)$.
  Then $\locreada(\mathsf{w},a)$ is defined and, in particular,
  $\locreada(\mathsf{w},a)=\langle t,p'\rangle$ with
  \begin{enumerate}
  \item $\ell'\approx_st$
  \item $p=\mathit{true}$ implies $p'=\mathit{true}$.
  \end{enumerate}
\end{proposition}
\begin{proof}
  Let $\rho\star h=s$. We proceed by structural induction on $\mathsf{w}$.
  \begin{itemize}
    \item (Base: $\mathsf{w}\in\dom(\kappa)$)
    By Def.~\ref{def:location_for_reading},
    $\locread(\mathsf{w},s) = \langle\rho(\mathsf{w}),\mathit{false}\rangle$,
    hence $\ell'=\rho(\mathsf{w})$ and $p=\mathit{false}$. So,
    $\mathsf{w}\to\ell'\in\rho$. As $s\in\gamma_\kappa(a)$, there is
    $\mathsf{w}\to^m t\in a$ s.t. $\ell'\approx_s t$.
    Moreover, by Def.~\ref{def:abstract-store},
    $\locreada(\mathsf{w},a)=\langle a(\mathsf{w}),\mathit{false}\rangle
    =\langle t,\mathit{false}\rangle$.
    %
  \item (Induction step) Suppose that the thesis holds for $\mathsf{w}$. Let us prove it
    for $\mathtt{*}\mathsf{w}$.
    As $\locread(\mathtt{*}\mathsf{w},s)$ is defined,
    by Def.~\ref{def:location_for_reading} there are three possibilities.
    \begin{itemize}
      \item Either $\locread(\mathtt{*}\mathsf{w},s)=\langle\ell',\mathit{true}\rangle$,
      $\locread(\mathsf{w},s) = \langle \ell,b\rangle$
      and $\ell\to\ell'\in h$. By induction hypothesis,
      $\locreada(\mathsf{w},a)=\langle t',b'\rangle$ with $\ell\approx_s t'$,
      and $b=\mathit{true}$ implies $b'=\mathit{true}$.
      By Def.~\ref{def:concretization}, as $\ell\to\ell'\in h$,
      necessarily $t'$ has the form $\borrow\{\mathsf{w}_1,\ldots,\mathsf{w}_n\}$
      and $\ell'\sim_s\borrow\{\mathsf{w}_1,\ldots,\mathsf{w}_n\}$.
      By Def.~\ref{def:value_compatibility}, it exists $i\in\{1,\ldots,n\}$
      such that $\loc(\mathsf{w}_i,s)=\ell'$.
      By Lemma~\ref{lem:consistency}, $t'$ is type consistent so that,
      by Def.~\ref{def:abstract-store},
      $t=t(\mathsf{w}_1,\ldots,\mathsf{w}_n,a)$ is defined and
      $\locreada(\mathtt{*}\mathsf{w},a)=\langle t,\mathit{true}\rangle$.
      Therefore, the second point of the thesis holds.
      It remains to prove that $\ell'\approx_st$.
      Consider a $t''\in\unwind(\mathsf{w}_i,a)\cap\mathbb{T}_\kappa$.
      By Lemma~\ref{lem:unwind_types}, it is
      $a(\mathsf{root}(\mathsf{w}_i))=\underbrace{\boxempty\cdots\boxempty}_{\text{$m$}}t''$, with
      $\mathsf{w}_i=\underbrace{\mathtt{*}\cdots\mathtt{*}}_{\text{$m$}}\mathsf{root}(\mathsf{w}_i)$.
      Since $s\in\gamma_\kappa(a)$, it follows that
      $s(\mathsf{root}(\mathsf{w}_i))\approx_s a(\mathsf{root}(\mathsf{w}_i))
      =\underbrace{\boxempty\cdots\boxempty}_{\text{$m$}}t''$ and therefore
      $\loc(\mathsf{w}_i,s)\approx_st''$. It follows that
      $\ell'=\loc(\mathsf{w}_i,s)\approx_st(\mathsf{w}_1,\ldots,\mathsf{w}_n,a)=t$.
    \item Or $b=\mathit{true}$,
      $\locread(\mathsf{w'},s) = \langle \ell',b'\rangle$
      and $\ell'\leadsto\ell\in h$. By induction hypothesis,
      $\locreada(\mathsf{w'},a)=\langle t,b''\rangle$ with $\ell'\approx_s t$
      and $b'=\mathit{true}$ implies $b''=\mathit{true}$.
      By Def.~\ref{def:concretization}, as $\ell'\leadsto\ell\in h$,
      necessarily $t$ has the form $\mutborrow\{\mathsf{w}_1,\ldots,\mathsf{w}_n\}$.
      \item Or $\locread(\mathsf{w'},s) = \langle \ell',b\rangle$
      and $\ell'\Rightarrow\ell\in h$ with $\ell\not\in\mathbb{Z}$.
      By induction hypothesis,
      $\locreada(\mathsf{w'},a)=\langle t',b'\rangle$ with $\ell'\approx_s t'$
      and $b=\mathit{true}$ implies $b'=\mathit{true}$.
      By Def.~\ref{def:concretization}, as $\ell'\Rightarrow\ell\in h$,
      necessarily $t'$ has the form $\boxtype{t}$. Consequently,
      $\locreada(\mathsf{w},a) = \langle t,b'\rangle$.
    \end{itemize}
    \qed
  \end{itemize}
\end{proof}

The concrete semantics uses a precise notion of borrowed locations
(Def.~\ref{def:borrow}). The abstract semantics has abstracted away the concrete locations
and can only use types for determining when a value might be borrowed, which leads
to a coarser notion of \emph{abstract} borrowing for leftvalues (that is, for their value).

\begin{definition}\label{def:abstract_borrow}
  Let $a\in\mathbb{AS}_\kappa$ and $\mathsf{w}\in\Leftvalues$.
  Then $\mathsf{w}$ is
  \emph{borrowed as immutable} in $a$ if there exists $v\in\dom(\kappa)$ such that
  $\beta_\to(a(v),\mathsf{w})$ holds, where
  \begin{align*}
    \beta_\to(\mathsf{int},\mathsf{w})&=\mathit{false}\\
    \beta_\to(\borrow\{\mathsf{w}_1,\ldots,\mathsf{w}_n\},\mathsf{w})&=\exists i.(\mathsf{root}(\mathsf{w}_i)=\mathsf{root}(\mathsf{w}))\\
    \beta_\to(\mutborrow\{\mathsf{w}_1,\ldots,\mathsf{w}_n\},\mathsf{w})&=\mathit{false}\\
    \beta_\to(\boxtype{t},\mathsf{w})&=\beta_\to(t,\mathsf{w})\\
    \beta_\to(\mathtt{dangling},\mathsf{w})&=\mathit{false.}
  \end{align*}
  Moreover, $\mathsf{w}$ is
  \emph{borrowed as mutable} in $a$ if there exists $v\in\dom(\kappa)$ such that
  $\beta_\leadsto(a(v),\mathsf{w})$ holds, where
  \begin{align*}
    \beta_\leadsto(\mathsf{int},\mathsf{w})&=\mathit{false}\\
    \beta_\leadsto(\borrow\{\mathsf{w}_1,\ldots,\mathsf{w}_n\},\mathsf{w})&=\mathit{false}\\
    \beta_\leadsto(\mutborrow\{\mathsf{w}_1,\ldots,\mathsf{w}_n\},\mathsf{w})&=\exists i.(\mathsf{root}(\mathsf{w}_i)=\mathsf{root}(\mathsf{w}))\\
    \beta_\leadsto(\boxtype{t},\mathsf{w})&=\beta_\leadsto(t,\mathsf{w})\\
    \beta_\leadsto(\mathtt{dangling},\mathsf{w})&=\mathit{false.}
  \end{align*}
  Finally, $\mathsf{w}$ is \emph{borrowed} in $a$ if it is borrowed either as immutable
  or as mutable in $a$.
\end{definition}

\begin{proposition}[Abstract borrowing is correct]
  \label{prop:abstract_borrow_correctness}
  Let $\mathsf{w}\in\Leftvalues_\kappa$, $s\in\mathbb{S}_\kappa$
  and $a\in\mathbb{AS}_\kappa$ be
  such that $s\in\gamma_\kappa(a)$.
  If $\mathsf{w}$ is borrowed as immutable in $s$, then
  $\mathsf{w}$ is borrowed as immutable in $a$.
  If $\mathsf{w}$ is borrowed as mutable in $s$, then
  $\mathsf{w}$ is borrowed as mutable in $a$.
  If $\mathsf{w}$ is borrowed in $s$, then
  $\mathsf{w}$ is borrowed in $a$.
\end{proposition}
\begin{proof}
  TODO
\end{proof}

The abstract evaluation of an expression yields the abstraction of the
value of the expression, a potentially updated abstract store
and an arrow that expresses the nature of the value: an owned value, an immutable borrow
or a mutable borrow. Let us start from the leftvalues, then consider the other kinds of
expressions.

\begin{definition}[Abstract location]\label{def:abstract_location}
  Let $\mathsf{w}\in\Leftvalues$ and $a\in\mathbb{AS}_\kappa$. The partial function
  $\loca(\mathsf{w},a)$ yields the type of the location holding the value of $\mathsf{w}$:
  \[
  \loca(\mathsf{w},a)=\begin{cases}
  \text{undefined} & \text{if $\locreada(\mathsf{w},a)$ is undefined}\\
  t & \text{if $\locreada(\mathsf{w},a)=\langle t,b\rangle$.}
  \end{cases}
  \]
\end{definition}

\begin{definition}[Abstract semantics of leftvalues]\label{def:abstract_semantics_leftvalues}
  Let $a\in\mathbb{AS}_\kappa$ and $\mathsf{w}\in\Leftvalues_\kappa$. Then
  \[
  \dena[\mathsf{w}]{a}=\begin{cases}
  \langle\mathsf{dangling},\bot\rangle & \text{if $\loca(\mathsf{w},a)$ is undefined}\\
  \langle t, a\rangle & \text{otherwise, if $t=\loca(\mathsf{w},a)$ and $\copytype(t)$}\\
  \langle t, a[x\to^m\overbrace{\boxempty\cdots\boxempty}^{n}\mathsf{dangling}]\rangle & \text{otherwise, if $t=\loca(\mathsf{w},a)$ and $\movetype(t)$,}\\
  & \quad\text{where $\mathsf{w}=\underbrace{*\ldots*}_{n}x$ and $x\to^mt'\in a$.}
  \end{cases}
  \]
\end{definition}

\begin{proposition}[The abstract semantics of leftvalues is correct]
  \label{prop:abstract_leftvalues_correctness}
  Let $\mathsf{w}\in\Leftvalues_\kappa$, $s=\rho\star h\in\mathbb{S}_\kappa$
  and $a\in\mathbb{AS}_\kappa$ be
  such that $s\in\gamma_\kappa(a)$ and
  $\den[\mathsf{w}]{s}$ is defined. Let
  $\dena[\mathsf{w}]{a}=\langle t,a'\rangle$. Then
  \begin{enumerate}
  \item if $\den[\mathsf{w}]{s}=\langle v,h',\to\rangle$
    then $\rho\star h'\in\gamma_\kappa(a')$, $v\sim_{\rho\star h'}t$
    and $t=\borrow\{\mathsf{w}_1,\ldots,\mathsf{w}_n\}$
  \item if $\den[\mathsf{w}]{s}=\langle v,h',\Rightarrow\rangle$ with $v\in\mathbb{Z}$
    then $\rho\star h'\in\gamma_\kappa(a')$, $v\sim_{\rho\star h'}t$
    and $t=\mathsf{int}$
  \item if $\den[\mathsf{w}]{s}=\langle v,h',\Rightarrow\rangle$ with $v\in\Locs$
    then $\rho\star h'\in\gamma_\kappa(a')$, $v\sim_{\rho\star h'}t$
    and $t=\boxtype{t'}$
  \item if $\den[\mathsf{w}]{s}=\langle v,h',\leadsto\rangle$
    then $\rho\star h'\in\gamma_\kappa(a')$, $v\sim_{\rho\star h'}t$
    and $t=\mutborrow\{\mathsf{w}_1,\ldots,\mathsf{w}_n\}$.
  \end{enumerate}
\end{proposition}
\begin{proof}
  TODO
\end{proof}

\section{Implementation}\label{sec:implementation}

This section will present the implementation of the abstract semantics
for the subset of Rust considered in the paper. This implementation allows
to check the execution of the abstract semantics.

\textbf{TODO}


\section{Conclusion}\label{sec:conclusion} 

\textbf{TODO}



\bibliographystyle{plain}
\bibliography{biblio}

\end{document}
