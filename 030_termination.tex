\section{A Sufficient Condition for Termination}\label{sec:termination}

This section provides a sufficient condition for the termination of the
typing algorithm for leftvalues in Def.~\ref{def:type}. It is based on the
idea that the Rust type system forces programmers to build
\emph{linear} data structures. This translates into a notion of
\emph{linearization} for typings, meaning that they map
variables in a way that does not allow cycles: each variable is mapped
into a type where variables have strictly lower ranks.

\begin{definition}\label{def:linearization}
  A typing $\tau$ over a context $\kappa$ is \emph{linearizable} if there exists
  an injective function $\phi:\kappa\to\mathbb{N}$ such that, for every $x\in\kappa$ and every
  $v$ that occurs in $\tau(x)$, it is $\phi(x)>\phi(v)$. We say that $\phi(y)$ is the \emph{rank} of $y$.
\end{definition}

A linearizable typing induces an ordering between leftvalues. Intuitively, either
the number of dereferences decreases, or the rank of the root decreases.

\begin{definition}\label{def:ordering}
  Given a context $\kappa$ and a linearizable typing $\tau$ over $\kappa$, the
  relation $>$ between leftvalues is the minimal relation such that
  \begin{enumerate}
  \item $\mathtt{*}\mathsf{w}>\mathsf{w}$, and
  \item $\mathsf{w}_1>\mathsf{w}_2$ if $\phi(\mathsf{root}(\mathsf{w}_1))>\phi(\mathsf{root}(\mathsf{w}_2))$.
  \end{enumerate}
\end{definition}

\begin{proposition}\label{prop:well-founded}
  The relation $>$ from Def.~\ref{def:ordering} is well-founded.
\end{proposition}
\begin{proof}
  Assume by contradiction that $>$ is not well-founded. Then there is an infinite sequence of
  leftvalues $s=\mathsf{w}_0>\mathsf{w}_1>\cdots>\mathsf{w}_n>\cdots$.
  Since, in the first rule of Def.~\ref{def:ordering}, it is
  $\mathsf{root}(\mathtt{*}\mathsf{w})=\mathsf{root}(\mathsf{w})$ and consequently
  $\phi(\mathsf{root}(\mathtt{*}\mathsf{w}))=\phi(\mathsf{root}(\mathsf{w}))$,
  we conclude that the rank of the root of the leftvalues
  decreases at most
  $|\kappa|$ times in $s$ or remains constant. Hence, there is a finite $k$ such that
  $\phi(\mathsf{root}(\mathsf{w}_{k}))=\phi(\mathsf{root}(\mathsf{w}_{k+i}))$
  for all $i\ge 0$. This means that, from $k$ onwards, only rule 1 of Def.~\ref{def:ordering}
  is applied. But that rule strictly decreases the size of the leftvalues and consequently cannot
  be applied indefinitely. This is incompatible with the hypothesis that $s$ is infinite.
  \qed
\end{proof}

Since $>$ is well-founded, we can use it in a proof by induction, as in the next result.

\begin{proposition}\label{prop:termination}
  If the typing $\tau$ over $\kappa$ is linearizable, then the algorithm for computing $\type$ given in
  Def.~\ref{def:type} terminates.
\end{proposition}
\begin{proof}
  We actually prove a stronger statement, namely that, given $\mathsf{w}\in\Leftvalues_\kappa$:
  \begin{enumerate}
  \item $\type(\mathsf{w},\tau)$ terminates;
  \item every variable $v$ that occurs in $\type(\mathsf{w},\tau)$
    is such that $\phi(\mathsf{root}(\mathsf{w}))>\phi(v)$.
  \end{enumerate}
  We proceed by induction on $\mathsf{w}$. The base case is when $\mathsf{w}$ is actually the variable $x$
  of lowest rank. By Def.~\ref{def:type}, it is $\type(x,\tau)=\tau(x)$ hence it terminates and
  no variable occurs in it, since (Def.~\ref{def:linearization})
  the rank of those variables should be even lower, which is impossible.
  Assume now that both~1 and~2 hold for all leftvalues $\mathsf{w}'$ such that $\mathsf{w}>\mathsf{w}'$.
  If $\mathsf{w}$ is a variable $x$, then $\type(x,\tau)=\tau(x)$ hence
  $\type(\mathsf{w},\tau)$ terminates and every variable $v$ that occurs in $\tau(x)$ is such that
  $\phi(x)>\phi(v)$ (Def.~\ref{def:linearization}). Hence both~1 and~2 hold for $\mathsf{w}$ as well.
  If, instead, $\mathsf{w}=\mathtt{*}\mathsf{w}'$ for a suitable $\mathsf{w}'$,
  then $\mathsf{w}>\mathsf{w}'$ (Def.~\ref{def:ordering}) and by inductive hypothesis we know that~1
  and~2 hold for $\mathsf{w}'$. The computation of $\type(\mathtt{*}\mathsf{w}',\tau)$
  first recurs on $\type(\mathsf{w}',\tau)$ (Def.~\ref{def:type}).
  In the first, second, third and seventh case of Def.~\ref{def:type} also the
  computation of $\type(\mathtt{*}\mathsf{w}',\tau)$ terminates and property~2 is vacuously true.
  In the sixth case in Def.~\ref{def:type}, the computation of
  $\type(\mathtt{*}\mathsf{w}',\tau)$ terminates and
  $\type(\mathsf{w}',\tau)=\boxtype{\type(\mathtt{*}\mathsf{w}',\tau)}$.
  Since $\mathsf{w}'$ and $\mathtt{*}\mathsf{w}'$ have the same root, condition~2
  lifts from $\mathsf{w}'$ to $\mathtt{*}\mathsf{w}'$.
  In the fourth and fifth cases, by inductive hypothesis we know that~2 holds
  for $\mathsf{w}'$ and consequently the roots of the $\mathsf{w}_i$ in Def.~\ref{def:type}
  have lower rank than the root of $\mathsf{w}'$. That is, $\mathsf{w}'>\mathsf{w}_i$
  for every $i$. By inductive hypothesis, both~1 and~2 hold for those $\mathsf{w}_i$ as well. Hence
  all $\type(\mathsf{w}_i,\tau)$ terminate. If any of them is undefined
  or if $\sqcup_{1\le i\le n}\type(\mathsf{w}_i,\tau)$ is undefined, also
  $\mathsf{type}(\mathtt{*}\mathsf{w}',\tau)$ is undefined and therefore terminates and
  satisfies~2 vacuously. Otherwise
  $\mathsf{type}(\mathtt{*}\mathsf{w}',\tau)$ terminates and yields
  $\sqcup_{1\le i\le n}\type(\mathsf{w}_i,\tau)$.
  Since $\sqcup$ does not introduce any new variable
  (Def.~\ref{def:lub}), every variable that occurs in
  $\sqcup_{1\le i\le n}\type(\mathsf{w}_i,\tau)$ has lower rank than
  $\mathsf{root}(\mathsf{w}')=\mathsf{root}(\mathsf{w})$. Therefore
  both~1 and~2 hold for $\mathsf{w}$ also in this case.
  \qed
\end{proof}
