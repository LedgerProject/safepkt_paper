\section{A Sufficient Condition for Termination}\label{sec:termination}

This section provides a sufficient condition for the termination of the
leftvalue typing algorithm in Def.~\ref{def:type}.

The notion of \emph{linearization} for typings means that they map
variables in a way that does not allow cycles, since a variable is mapped
into a type whose variables have strictly larger rank. This is formally expressed below.

\begin{definition}\label{def:linearization}
  A typing $\tau$ over a context $\kappa$ is \emph{linearizable} if there exists
  an injective function $\phi:\kappa\to\mathbb{N}$ such that, for every $x\in\kappa$ and every
  $v$ that occurs in $\tau(x)$, it is $\phi(x)>\phi(v)$. We say that $\phi(y)$ is the \emph{rank} of $y$.
\end{definition}

A linearizable typing induces an ordering between leftvalues. Intuitively, either
the number of dereferences decreases, or the rank of the root decreases.

\begin{definition}\label{def:ordering}
  Given a context $\kappa$ and a linearizable typing $\tau$ over $\kappa$, the
  relation $>$ between leftvalues is defined as
  \begin{enumerate}
  \item $\mathtt{*}\mathsf{w}>\mathsf{w}$;
  \item $\mathsf{w}_1>\mathsf{w}_2$ if $\phi(\mathsf{root}(\mathsf{w}_1))>\phi(\mathsf{root}(\mathsf{w}_2))$.
  \end{enumerate}
\end{definition}

\begin{proposition}\label{prop:well-founded}
  The relation $>$ from Def.~\ref{def:ordering} is well-founded.
\end{proposition}
\begin{proof}
  Assume by contradiction that $>$ is not well-founded. Then there is an infinite sequence of
  leftvalues $s=\mathsf{w}_0>\mathsf{w}_1>\cdots>\mathsf{w}_n>\cdots$.
  Since, in the first rule of Def.~\ref{def:ordering}, it is
  $\mathsf{root}(\mathtt{*}\mathsf{w})=\mathsf{root}(\mathsf{w})$ and consequently
  $\phi(\mathsf{root}(\mathtt{*}\mathsf{w}))=\phi(\mathsf{root}(\mathsf{w}))$,
  we conclude that the rank of the root of the leftvalues
  decreases at most
  $|\kappa|$ times in $s$ or remains constant. Hence, there is a finite $k$ such that
  $\phi(\mathsf{root}(\mathsf{w}_{k}))=\phi(\mathsf{root}(\mathsf{w}_{k+i}))$
  for all $i\ge 0$. This means that, from $k$ onwards, only rule 1 of Def.~\ref{def:ordering}
  is applied. But that rule strictly decreases the size of the leftvalues and consequently cannot
  be applied indefinitely. This is in incompatible with the hypothesis that $s$ be infinite.
  \qed
\end{proof}

Since $>$ is well-founded, we can use it in a proof by induction, as for the next result.

\begin{proposition}\label{prop:termination}
  If the typing $\tau$ over $\kappa$ is linearizable, then the algorithm for computing $\type$ given in
  Def.~\ref{def:type} terminates.
\end{proposition}
\begin{proof}
  We actually prove a stronger statement, namely that, given $\mathsf{w}\in\Leftvalues_\kappa$:
  \begin{enumerate}
  \item $\type(\mathsf{w},\tau)$ terminates;
  \item every variable $v$ that occurs in $\type(\mathsf{w},\tau)$
    is such that $\phi(\mathsf{root}(\mathsf{w}))>\phi(v)$.
  \end{enumerate}
  We proceed by induction on $\mathsf{w}$. The base case is when $\mathsf{w}$ is actually the variable $x$
  of lowest rank. By Def.~\ref{def:type}, it is $\type(x,\tau)=\tau(x)$ hence it terminates and
  no variable occurs in it, since (Def.~\ref{def:linearization})
  the rank of those variables should be even lower, which is impossible.
  Assume now that both~1 and~2 hold for all leftvalues $\mathsf{w}'$ such that $\mathsf{w}>\mathsf{w}'$.
  If $\mathsf{w}$ is a variable $x$, then $\type(x,\tau)=\tau(x)$ hence
  $\type(\mathsf{w},\tau)$ terminates and every variable $v$ that occurs in $\tau(x)$ is such that
  $\phi(x)>\phi(v)$ (Def.~\ref{def:linearization}). Hence both~1 and~2 hold for $\mathsf{w}$ as well.
  If, instead, $\mathsf{w}=\mathtt{*}\mathsf{w}'$ for a suitable $\mathsf{w}'$,
  then $\mathsf{w}>\mathsf{w}'$ (Def.~\ref{def:ordering}) and by inductive hypothesis we know that~1
  and~2 hold for $\mathsf{w}'$. The computation of $\type(\mathtt{*}\mathsf{w}',\tau)$
  first recurs on $\type(\mathsf{w}',\tau)$ (Def.~\ref{def:type}).
  In the first, second, third and seventh case of Def.~\ref{def:type} also the
  computation of $\type(\mathtt{*}\mathsf{w}',\tau)$ terminates and property~2 is vacuously true.
  In the sixth case in Def.~\ref{def:type}, the computation of
  $\type(\mathtt{*}\mathsf{w}',\tau)$ terminates and
  $\type(\mathsf{w}',\tau)=\boxtype{\type(\mathtt{*}\mathsf{w}',\tau)}$.
  Since $\mathsf{w}'$ and $\mathtt{*}\mathsf{w}'$ have the same root, condition~2
  lifts from $\mathsf{w}'$ to $\mathtt{*}\mathsf{w}'$.
  In the fourth and fifth cases, by inductive hypothesis we know that~2 holds
  for $\mathsf{w}'$ and consequently the roots of the $\mathsf{w}_i$ in Def.~\ref{def:type}
  have lower rank than the root of $\mathsf{w}'$. That is, $\mathsf{w}'>\mathsf{w}_i$
  for every $i$. By inductive hypothesis, both~1 and~2 hold for those $\mathsf{w}_i$ as well. Hence
  all $\type(\mathsf{w}_i,\tau)$ terminate. If any of them is undefined
  or if $\sqcup_{1\le i\le n}\type(\mathsf{w}_i,\tau)$ is undefined, also
  $\mathsf{type}(\mathtt{*}\mathsf{w}',\tau)$ is undefined and therefore terminates and
  satisfies~2 vacuously. Otherwise
  $\mathsf{type}(\mathtt{*}\mathsf{w}',\tau)$ terminates and yields
  $\sqcup_{1\le i\le n}\type(\mathsf{w}_i,\tau)$.
  Since $\sqcup$ does not introduce any new variable
  (Def.~\ref{def:lub}), every variable that occurs in
  $\sqcup_{1\le i\le n}\type(\mathsf{w}_i,\tau)$ has lower rank than
  $\mathsf{root}(\mathsf{w}')=\mathsf{root}(\mathsf{w})$. Therefore
  both~1 and~2 hold for $\mathsf{w}$ also in this case.
  \qed
\end{proof}

The following definition captures how leftvalues \emph{decrease}
during the recursion in Def.~\ref{def:type}.

\begin{definition}[Dependencies between leftvalues]\label{def:dependencies}
  Given a context $\kappa$ and a typing $\tau$ over $\kappa$, the \emph{dependencies between leftvalues}
  induced by $\tau$ are the relation $\gg$ defined as
  \[
  \mathsf{closure}\left(\{\mathtt{*}\mathsf{w}\gg\mathsf{w}\mid\mathsf{w}\in\Leftvalues_\kappa\}
  \cup\bigcup\limits_{x\in\kappa}\mathsf{dependencies}(x,\tau(x))\right)
  \]
  where
  \begin{align*}
    \mathsf{dependencies}(\mathsf{w},\mathsf{int})&=\varnothing\\
    \mathsf{dependencies}(\mathsf{w},\mathsf{dangling})&=\varnothing\\
    \mathsf{dependencies}(\mathsf{w},\boxtype{t})&=\mathsf{dependencies}(\mathtt{*}\mathsf{w},t)\\
    \mathsf{dependencies}(\mathsf{w},\borrow\{\mathsf{w}_1,\ldots,\mathsf{w}_n\})&=\{\mathtt{*}\mathsf{w}\gg\mathsf{w}_i\mid 1\le i\le n\}\\
    \mathsf{dependencies}(\mathsf{w},\mutborrow\{\mathsf{w}_1,\ldots,\mathsf{w}_n\})&=\{\mathtt{*}\mathsf{w}\gg\mathsf{w}_i\mid 1\le i\le n\}
  \end{align*}
  and
  \[
  \mathsf{closure}(R)=R\cup\left\{\underbrace{\mathtt{*}\cdots\mathtt{*}}_{n}\mathsf{w}_1\gg\mathsf{w}_3\left|
  \begin{array}{l}
    \mathsf{w}_1\gg\mathsf{w}_2\in R,\ \underbrace{\mathtt{*}\cdots\mathtt{*}}_{n\ge 0}\mathsf{w}_2\gg\mathsf{w}_3\in R\\
    \text{and $\mathsf{w_3}$ is in a borrow that occurs in $\tau$}
  \end{array}\right.\right\}.
  \]
\end{definition}

\noindent
Note that the closure in Def.~\ref{def:dependencies} makes $\gg$ transitive.

Prop.~\ref{prop:acyclicity} proves that the dependencies between leftvalues
model the pattern of the recursion in Def.~\ref{def:type}. Moreover,
if the $\gg$ relation is acyclical, then $\gg$ (and consequently the recusion
in Def.~\ref{def:type}) are well-founded.

\begin{proposition}\label{prop:acyclicity}
  Given a context $\kappa$ and a typing $\tau$ over $\kappa$
  such that the dependencies between leftvalues induced by $\tau$ are acyclical,
  the recursion used in the definition of function $\type$ is well-founded and consistent with $\gg$.
\end{proposition}
\begin{proof}
  First we prove that $\gg$ is a well-founded relation. Assume the contrary. Then
  there is an infinite sequence of leftvalues
  $\mathsf{w}_1\gg\mathsf{w}_2\gg\cdots\gg\mathsf{w}_n\gg\cdots$.
  Since $\mathsf{w}\gg\mathsf{w}'\in\mathsf{dependencies}(x,t)$ entails that
  $\mathsf{w}'$ is one of the leftvalues that occur in the borrows of $\tau$ and
  since there is only a finite number of such $\mathsf{w}'$, the acyclicity of $\gg$ entails that
  there must be a $\mathsf{w}_k=\underbrace{\mathtt{*}\cdots\mathtt{*}}_{\text{$n$}}x$
  such that the subsequent $\mathsf{w}_i$, $i>k$,
  are not leftvalues that occur in the borrows in $\tau$. By Def.~\ref{def:dependencies},
  it can only be $\mathsf{w}_{k+i}=\underbrace{\mathtt{*}\cdots\mathtt{*}}_{\text{$\le n-i$}}x$
  and consequently the length of the sequence is $k+n$ at most, impossible since we assumed that
  it was infinite.

  We now prove that
  \begin{enumerate}
  \item for every $\mathsf{w}\in\Leftvalues_\kappa$, the
    recursive uses $\type(\mathsf{w}',\tau)$ that occur for the definition
    of $\type(\mathsf{w},\tau)$ are such that $\mathsf{w}\gg\mathsf{w}'$;
  \item when $\type(\mathsf{w},\tau)=\borrow\{\mathsf{w}_1,\ldots,\mathsf{w}_n\}$
    or $\type(\mathsf{w},\tau)=\mutborrow\{\mathsf{w}_1,\ldots,\mathsf{w}_n\}$
    then $\mathtt{*}\mathsf{w}\gg\mathsf{w}_i$ for every $1\le i\le n$.
  \end{enumerate}
  Note that (1) by itself means that
  the recursion used in the
  definition of function $\type$ is well-founded and consistent with $\gg$, but we will also
  need (2) in order to prove (1).
  We prove (1) and (2) by induction on $\mathsf{w}$ with respect to $\gg$.

  \begin{itemize}
  \item (Base case)
    If $\mathsf{w}$
    has no $\mathsf{w}'$ such that $\mathsf{w}\gg\mathsf{w}'$, then it must be $\mathsf{w}\in\Vars$.
    Hence there are no recursive uses of $\type$ in the definition of
    $\type(\mathsf{w},\tau)$ and (1) holds. Moreover, in this case
    $\type(\mathsf{w},\tau)=\tau(\mathsf{w})$ and if
    $\tau(\mathsf{w})=\borrow\{\mathsf{w}_1,\ldots,\mathsf{w}_n\}$ by Def.~\ref{def:dependencies}
    we conclude that $\mathtt{*}\mathsf{w}\gg\mathsf{w}_i$ for every $1\le i\le n$, hence (2) holds.
    Similarly when $\tau(\mathsf{w})=\mutborrow\{\mathsf{w}_1,\ldots,\mathsf{w}_n\}$.
  \item (Inductice case)
    Assume now that (1) and (2) hold for $\mathsf{w}$. Consider
    the recursive uses $\type(\mathsf{w}',\tau)$ in the definition of
    $\type(\mathtt{*}\mathsf{w},\tau)$. One such recursive use is
    $\type(\mathsf{w},\tau)$ and $\mathtt{*}\mathsf{w}\gg\mathsf{w}$.
    Others are inside the computation of $\type(\mathsf{w},\tau)$ and by inductive hypothesis
    (1) holds, that is, they occur on $\mathsf{w}'$ such that $\mathsf{w}\gg\mathsf{w}'$.
    Hence $\mathtt{*}\mathsf{w}\gg\mathsf{w}\gg\mathsf{w}'$ and by transitivity
    $\mathtt{*}\mathsf{w}\gg\mathsf{w}'$. Finally, there are recursive uses
    when $\type(\mathsf{w},\tau)=\borrow\{\mathsf{w}_1,\ldots,\mathsf{w}_n\}$
    or when $\type(\mathsf{w},\tau)=\mutborrow\{\mathsf{w}_1,\ldots,\mathsf{w}_n\}$
    namely, uses of
    $\type(\mathsf{w}_i,\tau)$ with $1\le i\le n$. By inductive hypothesis, we know that
    (2) holds for $\mathsf{w}$, that is, $\mathtt{*}\mathsf{w}\gg\mathsf{w}_i$ for every
    $1\le i\le n$. This concludes the inductive case for the proof of (1).
    Let us prove (2) for $\mathtt{*}\mathsf{w}$ now. Assume then that
    $\type(\mathtt{*}\mathsf{w},\tau)=\borrow\{\mathsf{w}_1,\ldots,\mathsf{w}_n\}$ (the case
    $\type(\mathtt{*}\mathsf{w},\tau)=\mutborrow\{\mathsf{w}_1,\ldots,\mathsf{w}_n\}$ is similar).
    By Def.~\ref{def:type}, there are two possibilities:
    \begin{itemize}
    \item $\mathtt{*}\mathsf{w}=\underbrace{\mathtt{*}\cdots\mathtt{*}}_{\text{$m+1$}}x$
      and $\tau(x)=\underbrace{\boxempty\cdots\boxempty}_{\text{$m+1$}}
      \borrow\{\mathsf{w}_1,\ldots,\mathsf{w}_n\}$
      for some $m\ge 0$, where $x=\mathsf{root}(\mathsf{w})$.
      By Def.~\ref{def:dependencies}, the relation $\gg$ includes
      \begin{align*}
        \mathsf{dependencies}(x,\tau(x))&=\mathsf{dependencies}(x,\underbrace{\boxempty\cdots\boxempty}_{\text{$m+1$}}\borrow\{\mathsf{w}_1,\ldots,\mathsf{w}_n\})\\
        &=\mathsf{dependencies}(\underbrace{\mathtt{*}\cdots\mathtt{*}}_{\text{$m+1$}}x,
        \borrow\{\mathsf{w}_1,\ldots,\mathsf{w}_n\})\\
        &=\mathsf{dependencies}(\mathtt{*}\mathsf{w},\borrow\{\mathsf{w}_1,\ldots,\mathsf{w}_n\})\\
        &=\{\mathtt{**}\mathsf{w}\gg\mathsf{w}_i\mid 1\le i\le n\}
      \end{align*}
      and (2) holds for $\mathtt{*}\mathsf{w}$.
    \item $\borrow\{\mathsf{w}_1,\ldots,\mathsf{w}_n\}=\sqcup_{1\le i\le n'}\type(\mathsf{w}_i',\tau)$
      with $\type(\mathsf{w},\tau)=\borrow\{\mathsf{w}'_1,\ldots,\mathsf{w}'_{n'}\}$.
      The only possibility is that $\type(\mathsf{w}_j',\tau)=\borrow W_j$ for every $1\le j\le n'$,
      with $\{\mathsf{w}_1,\ldots,\mathsf{w}_n\}=\cup_{1\le j\le n'}W_i$.
      By inductive hypothesis of (2), we know that
      $\mathtt{*}\mathsf{w}\gg\mathsf{w}_j'$ for every $1\le j\le n'$ and that
      $\mathtt{*}\mathsf{w}_j'\gg\mathsf{w}''$ for every $\mathsf{w}''\in W_j$ and
      every $1\le j\le n'$. Note that $\mathsf{w}''$ is one of the leftvalues that occur
      in the borrows of $\tau$, by Def.~\ref{def:type}.
      By closure (Def.~\ref{def:dependencies}), we conclude that
      $\mathtt{**}\mathsf{w}\gg\mathsf{w}''$ for every $\mathsf{w}''\in W_j$ and
      every $1\le j\le n'$. Since each $\mathsf{w}_i$ belongs to some $W_j$, we conclude that
      $\mathtt{**}\mathsf{w}\gg\mathsf{w}_i$ for every $1\le i\le n$ and (2) holds for $\mathtt{*}\mathsf{w}$.
    \end{itemize}
  \end{itemize}
  \qed
\end{proof}
