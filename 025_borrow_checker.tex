\section{The Borrow Checker}\label{sec:borrow_checker}

The borrow checker is formalized as a type system whose rules
bind the typing $\tau$ before the evaluation of a syntactical object $s$ to
the typing $\tau'$ resulting after that evaluation.
These rules require the lifetime $l$ of the piece of code that is being typed.
We write this as $\tau,l\vdash s\dashv\tau'$.
On expressions, the typing
rules provide the inferred type $T$ of the expression as well:
$\tau,l\vdash e:T\dashv\tau'$.

\subsection{Typing rules for expressions}

\underline{\textsf{T-Const}}.
This rule applies to integer constants. Their evaluation
yields a value of type $\mathsf{int}$ and does not modify the typing:
\[
\frac
    {}
    {\tau,l\vdash i:\mathsf{int}\dashv\tau}
\]

\noindent
\underline{\textsf{T-Copy}}.
This rule applies to leftvalues whose type
has copy semantics. Their evaluation yields their value, while
the typing remains unchanged. The rule requires that the leftvalue
can be accessed for reading:
\[
\frac
    {T^m=\type(\mathsf{w},\tau)\quad\copytype(T)\quad\neg\mathsf{readProhibited}(\mathsf{w},\tau)}
    {\tau,l\vdash\mathsf{w}:T\dashv\tau}
\]

\noindent
\underline{\textsf{T-Move}}.
This rule applies to leftvalues applies whose type
has move semantics. Their evaluation yields their value,
but its ownership is moved away from the leftvalue. Because of this,
the typing gets modified, by letting the old container of the value
get the $\mathsf{dangling}$ type (it cannot be used anymore).
As a consequence, reading, from a leftvalue, a value with move semantics
amounts to writing into its old container and requires write permission:
\[
\frac
    {T^m=\type(\mathsf{w},\tau)\quad\movetype(T)\quad\neg\mathsf{writeProhibited}(\mathsf{w},\tau)}
    {\tau,l\vdash\mathsf{w}:T\dashv\mathsf{move}(\mathsf{w},\tau)}
\]
where the $\mathsf{move}$ function
modifies the binding for the root of $\mathsf{w}$:
\[
\mathsf{move}(\mathsf{w},\tau)=\tau[\mathsf{root}(\mathsf{w})\mapsto
  \mathsf{strike}(\mathsf{w},\tau(\mathsf{root}(\mathsf{w})))]
\]

with
\begin{align*}
  \mathsf{strike}(x,T^l)&=\mathsf{dangling}^l\\
  \mathsf{strike}(\mathtt{*}\mathsf{w},(\boxtype{T})^l)&=
  (\boxtype{|\mathsf{strike}(\mathsf{w},T^l)|})^l.
\end{align*}
The function $\mathsf{strike}$ is undefined otherwise.

\noindent
\underline{\textsf{T-ImmBorrow}}.
The evaluation of a borrow expression requires the borrowed leftvalue
to be readable and have full type
(only value with full type can be borrowed in Rust):
\[
\frac
    {\mathsf{full}(|\type(\mathsf{w},\tau)|)\quad\neg\mathsf{readProhibited}(\mathsf{w},\tau)}
    {\tau,l\vdash\borrow\mathsf{w}:\borrow\mathsf{w}\dashv\tau}
\]

\noindent
\underline{\textsf{T-MutBorrow}}.
The evaluation of a mutable borrow expression requires the borrowed
leftvalue to be writable and have full type
(only value with full type can be borrowed in Rust). Moreover, Rust requires that
the borrowed leftvalue never traverses an immutable borrow:
\[
\frac
    {\mathsf{full}(|\type(\mathsf{w},\tau)|)\quad\neg\mathsf{writeProhibited}(\mathsf{w},\tau)
    \quad\mathsf{mutable}(\mathsf{w},|\tau(\mathsf{root}(\mathsf{w}))|,\tau)}
    {\tau,l\vdash\mutborrow\mathsf{w}:\mutborrow\mathsf{w}\dashv\tau}
\]
where
\begin{align*}
  \mathsf{mutable}(x,T,\tau)&=\mathit{true}\\
  \mathsf{mutable}(\mathtt{*}\mathsf{w},\boxtype{T},\tau)&=\mathsf{mutable}(\mathsf{w},T,\tau)\\
  \mathsf{mutable}(\mathtt{*}\underbrace{\mathtt{*}\cdots\mathtt{*}}_nx,
  \mutborrow\mathsf{w},\tau)&=\mathsf{mutable}(\underbrace{\mathtt{*}\cdots\mathtt{*}}_n\mathsf{w},
  |\tau(\mathsf{root}(\mathsf{w}))|,\tau).
\end{align*}

\noindent
\underline{\textsf{T-Box}}.
The evaluation of a box expression simply recurs on the boxed expression:
\[
\frac
    {\tau,l\vdash\mathsf{e}:T\dashv\tau'}
    {\tau,l\vdash\mathtt{box}\ \mathsf{e}:\boxtype{T}\dashv\tau'}
\]

\subsection{Typing rules for terms}

\noindent
\underline{\textsf{T-Block}}.
The execution of a block of statements simply recurs on each single statement.
At the end, the variables declared inside the block get dropped away. We assume
that variables cannot be redefined inside a block, hence there is no risk of name clash.
\[
\frac
    {\tau,l\vdash\mathsf{t}_1\dashv\tau_1\quad\ldots\quad\tau_{n-1},l\vdash\mathsf{t}_n\dashv\tau'
    \quad }
    {\tau,l\vdash\{\mathsf{t}_1;\ldots;\mathsf{t}_n\}^m\dashv\mathsf{drop}(m,\tau')}
\]
where
\[
\mathsf{drop}(m,\tau)=\{x\to T^l\mid x\in\dom(\tau),\ \tau(x)=T^l\text{ and }l\not=m\}.
\]

\noindent
\underline{\textsf{T-Declare}}.
The declaration of a fresh new variable $x$ evaluates its initialization expression $\mathsf{e}$ and
binds $x$ to the type of $\mathsf{e}$,
decorated with the lifetime of the block of code where the declaration is evaluated:
\[
\frac
    {x\not\in\dom(\tau)\quad\tau,l\vdash\mathsf{e}:T\dashv\tau'}
    {\tau,l\vdash\mathtt{let\ mut}\ x=\mathsf{e}\dashv\tau'[x\to T^l]}
\]

\noindent
\underline{\textsf{T-Assign}}.
The assignment of a value to a leftvalue $\mathsf{w}$ requires $\mathsf{w}$ to be writable.
In that case, the assigned expression is evaluated and assigned to $\mathsf{w}$.
This is modelled through the $\mathsf{write}$ function below.
Since $\mathsf{w}$ can be more complex than a single variable, the assignment might actually
update a variable in a mutable borrow reachable from the root of $\mathsf{w}$. This is reflected
in the (quite complex) definition of $\mathsf{write}$, that we take from~\cite{Pearce21}
where more details can be found:
\[
\frac
    {\begin{array}{c}
        \tau,l\vdash\mathsf{e}:T\dashv\tau' \quad
      \tau''=\mathsf{write}(\tau',\mathsf{w},T,\mathit{strong}) \quad
      \neg\mathsf{writeProhibited}(\tau'',\mathsf{w})\\
      \type(\mathsf{w},\tau)=D^m\quad\mathsf{survives}(T,m,\tau)
      \end{array}
    }
    {\tau,l\vdash\mathsf{w}=\mathsf{e}\dashv\tau''}
\]
where
\[
\mathsf{write}(\tau,\underbrace{\mathtt{*}\cdots\mathtt{*}}_nx,T,\mathit{modality})=\mathsf{apply}
(x,\update(\tau,n,|\tau(x)|,T,\mathit{modality}))
\]
%
where
%
\begin{align*}
  \update(\tau,0,T',T,\mathit{strong})&=\langle\tau,T\rangle\\
  \update(\tau,0,T',T,\mathit{weak})&=\langle\tau,T'\sqcup T\rangle\\
  \update(\tau,n+1,\boxtype{T'},T,\mathit{modality})
  &=\mathsf{expand}(\update(\tau,n,T',T,\mathit{modality}))\\
  \update(\tau,n+1,\mutborrow\mathsf{w},T,\_)&=
  \langle\mathsf{write}(\tau,\underbrace{\mathtt{*}\cdots\mathtt{*}}_n\mathsf{w},T,\mathit{weak}),
  \mutborrow\mathsf{w}\rangle
\end{align*}
and
\begin{align*}
  \mathsf{apply}(y,\langle\tau,T\rangle)&=\tau[y\to T^l]\quad\text{where $\tau(y)=D^l$}\\
  \mathsf{expand}(\langle\tau,T\rangle)&=\langle\tau,\boxtype{T}\rangle.
\end{align*}
%
\noindent
It is important to observe that $\mathsf{write}$
can either completely replace the type of $x$
(\emph{strong} update) or modify the previous type of variables
inside the mutable borrows in $\tau$ (\emph{weak} update).
No other variable can see its type updated.

Function $\mathsf{survives}(T,m,\tau)$ determines if all leftvalues contained
in the borrows or mutable borrows inside the type $T$ have a type whose lifetime is $m$ or
is larger than $m$. Hence they \emph{survive} to the end of the lifetime $m$.
The motivation of this contraint in rule
\textsf{T-Assign} is to guarantee that, when a variable $v$ can reach another variable $v'$,
the lifetime of $v'$ is equal or larger than the lifetime of $v$. Otherwise, the deallocation of
$v'$ (at the end of its lifetime) would leave a dangling reference reachable from $v$.
