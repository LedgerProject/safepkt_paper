\newcommand{\FR}{}
\def\FR/{$\mathtt{FR}$}


\section{Related Work}


Reed provided an early formalisation of Rust called ``Patina'' which
shares some similarities with \FR/~\cite{Reed15}.  For example, it
employs a flow-sensitive type system for characterising borrow
checking which operates over a ``shadow'' heap.  However, the scope of
Patina was significantly larger than that of \FR/.  As such, soundness
was not established.  Likewise, Wang {\em et al.} presented a formal,
executable operational semantics for Rust called KRust~\cite{WSZZZ18}.
This was defined in $\mathbb{K}$ --- a rewrite-based executable
semantic framework particularly suited at developing operational
semantics~\cite{RS10b}.  A large subset of Rust was defined in this
way and partially validated against the official Rust test suite.

Weiss {\em et al.} presented an unpublished system called {\em Oxide}
which bears striking similarity with \FR/~\cite{WPMA19}.  Oxide was
also inspired by Featherweight Java to produce a relatively lean
formalisation of Rust.  Still, it includes a far larger subset of Rust
than the \FR/ core (perhaps making it more {\em middleweight} than
{\em featherweight}.  However, there are also differences between \FR/
and Oxide.  For example, Oxide doesn't model boxes explicitly and, as
a result, has no clear means to model heap allocated memory.

The work of Jung {\em et al.} provides a comprehensive,
machine-checked formalisation for a realistic subset of
Rust~\cite{JJKD18}.  This includes various notions of concurrency and
extends to libraries using \lstinline{unsafe} features by identifying
{\em library-specific verification conditions} which must be satisfied
to ensure overall safety.  However, concessions were understandably
necessary given the enormity of this formalisation task (which, in
fact, amounts to roughly 17.5KLOC of Coq).  For example, the system
presented does not resemble the surface syntax of Rust but, rather, is
more akin to the {\em Mid-level Intermediate Representation (MIR)}
used within the Rust compiler.  Underpinning this development is {\em
  Iris} --- a framework for high-order concurrent separation
logic~\cite{JKBD16,KDDLV17,JKJBBD18}.  This enables, for example, a
notion of {\em borrow propositions} which correspond with borrowing in
Rust.


The potential hazards of \lstinline{unsafe} code are already a
considerable focus of academic work and, indeed, numerous bugs and
security advisories have already been found in real-world
programs~\cite{BKALK21,XCSZ20}.  In light of these issues, interest
has been growing in using state-of-the-art verification tools to
ensure Rust code is safe.  Such tools vary in their scope and levels
of precision.  For example, Rudra employs a straightforward static
analysis to scan for bug patterns related to error
handling~\cite{BKALK21}.  Nevertheless, the tool identified 74 new
CVE's (including two in the standard library).  SMACK provides another
good example here~\cite{BCDJL06,MB08}.  This tool translates LLVM IR
to Boogie/Z3 and was recently extended to Rust~\cite{BHR18}.
CRUST~\cite{TPT15} is similar, but uses CBMC~\cite{KT14} as the
backend.  CRUST specifically focuses on memory safety violations
(e.g. multiple mutable references to the same data).  An interesting
feature is support for automatically deriving drivers using a
technique reminiscent of that for test case generation~\cite{PE07}.
KLEE employs symbolic execution and was also extended to support
Rust~\cite{LAL18,LFEL19}.  Unlike CRUST this tool considers a larger
number of errors, including arithmetic overflow and buffer overruns
(i.e. not just those related to memory unsafety).  Prusti exploits
automated theorem proving as the core technique, building on
Viper~\cite{AMPS19}.  This makes Prusti more comparable with tools
such as Spec\#~\cite{BFLMSV11,BDFLS04} and
Dafny~\cite{Leino10,Leino12} which require additional programmer
annotations to verify memory-safety properties (e.g.  adding
specifications to clarify method side-effects, etc).  However, Prusti
exploits aliasing information inherent in Rust programs to avoid much
of this.  Instead, programmers can focus on specifying properties of
interest, such as the absence of arithmetic overflow or buffer
overruns.  Unfortunately, Prusti does not consider \lstinline{unsafe}
code (though it presumably could be managed with further
specification).  Finally, other relevant tools include
Miri~\cite{Ols16,JDKJD20} (a partially symbolic interpreter for MIR)
and RustHorn~\cite{MTK20} (a specialised verifier based on Constrained
Horn Clauses).

\subsection{TODO}

An interesting question explored by Jung {\em et al.} is that of
deciding what compiler optimisations should be permitted in unsafe
code~\cite{JDKJD20}.  This is a thorny issue because, within unsafe
code, the usual guarantees provided by Rust may not hold.  For
example, in unsafe code, multiple mutable borrows of the same location
can exist.  The proposed system, {\em Stacked Borrows}, provides an
operational semantics for memory accesses in Rust.  This introduces a
strong notion of {\em undefined behaviour} such that a compiler is
permitted to ignore the possibility of such programs when applying
optimisations (roughly in line with how C compilers handle undefined
behaviour~\cite{MGDKRWS19})
