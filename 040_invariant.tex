\section{Semantical Invariant}\label{sec:invariant}

This section proves an invariant of the typing rules
from~\cite{Pearce21}. Namely,
if those rules are applied to a linearizable typing $\tau$,
they can only lead to a linearizable typing $\tau'$.
By Prop.~\ref{prop:termination}, this means that the recursion used for
typing leftvalues in those rules is well-founded,
hence the implementation of the typing rules from~\cite{Pearce21} terminates.
We prove the invariant by rule induction on the rule for statements in~\cite{Pearce21}.

Some rules from~\cite{Pearce21} obviously satisfy the invariant,
since they do not modify the typing (for them, $\tau=\tau'$).
This is the case of rules
\textsf{T-Const}, \textsf{T-Copy}, \textsf{T-MutBorrow} and
\textsf{T-ImmBorrow}.

Other rules satisfy the invariant by a simple application of
rule induction, such as \textsf{T-Box} and \textsf{T-Seq}.

Rule \textsf{T-Move} is used for typing the evaluation of a leftvalue
whose type has move semantics. This modifies the typing since, in Rust,
the ownership of the leftvalue moves away from itself.
Formally, for this rule it is $\tau'=\mathsf{move}(\tau,\mathsf{w})$
for a suitable leftvalue $\mathsf{w}$,
where the $\mathsf{move}$ function
modifies the binding for the root of $\mathsf{w}$ and is defined as:
\[
\mathsf{move}(\tau,\mathsf{w})=\tau[\mathsf{root}(\mathsf{w})\mapsto
  \mathsf{strike}(\mathsf{w},\tau(\mathsf{root}(\mathsf{w})))]
\]
where
\begin{align*}
  \mathsf{strike}(x,t)&=\mathsf{dangling}\\
  \mathsf{strike}(\mathtt{*}\mathsf{w},\boxtype{t})&=
  \boxtype{\mathsf{strike}(\mathsf{w},t)}.
\end{align*}
The function $\mathsf{strike}$ is undefined otherwise. The intuition is that
$\mathsf{strike}$ can only make the set of variables in the right-hand side
of the typing smaller. Therefore, it can never make $\tau'$
non-linearizable. This is proved below.

\begin{lemma}\label{lem:move_invariant}
  If rule \textsf{T-Move} is applied from a linearizable typing $\tau$
  and leads to a typing $\tau'$, then also $\tau'$ is linearizable.
\end{lemma}
\begin{proof}
  By definition of $\mathsf{move}$, the only
  difference between $\tau$ and $\tau'$ is at $r=\mathsf{root}(\mathsf{w})$,
  where the variables that occur in $\tau'(r)$ are included in those that occur
  in $\tau(r)$ ($\mathsf{strike}$ can only strike away part of the type $\tau(r)$).
  Hence the same function $\phi$ that exists for $\tau$ (Def.~\ref{def:linearization})
  shows that $\tau'$ is linearizable.
  \qed
\end{proof}

Rule \textsf{T-Block} is used at the end of a block of code, where the set $S$ of local
variables declared in the block goes out of scope. As a consequence, this rule removes
the type bindings for those variables in $S$ from the initial typing $\tau$. It performs this
through a function $\mathsf{drop}$ that projects those variables away from $\tau$ and is defined as:
\[
\mathsf{drop}(\tau,S)=\tau|_{-S}.
\]
The invariant for rule \textsf{T-Block} follows by rule induction and by the following result,
whose intuition is that the removal of bindings from a typing can never make it
non-linearizable.
%
\begin{lemma}\label{lem:drop_invariant}
  If \textsf{drop} is applied from a linearizable typing $\tau$
  and leads to a typing $\tau'$, then also $\tau'$ is linearizable.
\end{lemma}
\begin{proof}
  The difference between $\tau$ and $\tau'$ is that $\tau'$ is missing some
  bindings for some variables that have been projected away. Therefore,
  the same function $\phi$ that exists for $\tau$ (Def.~\ref{def:linearization})
  can be used to show that $\tau'$ is linearizable.
  \qed
\end{proof}

Rule \textsf{T-Declare} models the declaration of a new variable $x$,
bound to a term $t$. The evaluation of $t$ leads to a typing $\tau$ that, by rule induction,
satisfies the invariant. As a final step, this rule
enlarges $\tau$ with a binding for $x$. Since $x$ is
fresh (that is, $x\not\in\dom(\tau)$),
variable $x$ does not occur in the right-hand side of that binding. Namely, the rule
leads to a new typing $\tau'=\tau[x\to T]$ where $T$ is the type of $t$, such that
$x$ does not occur in $T$. The invariant for rule \textsf{T-Declare} follows by the next result.

\begin{lemma}\label{lem:declare_invariant}
  Let $\tau$ be a linearizable typing for the context $\kappa$;
  let $x\not\in\kappa$ and $T\in\mathsf{T}_\kappa$ (hence $x$ does not occur in $T$).
  Then $\tau'=\tau[x\to T]$ is linearizable as well.
\end{lemma}
\begin{proof}
  Consider the function $\phi$ that shows that $\tau$ is linearizable (Def.~\ref{def:linearization}).
  Let us extend $\phi$ into an injective function $\phi'$ that gives $x$ the highest rank:
  \[
  \phi'=\phi\left[x\to 1+\max\limits_{y\in\kappa}\phi(y)\right].
  \]
  Given $y\in\kappa$, it is $\phi'(y)=\phi(y)>\phi(v)$ for every $v$ that occurs
  in $\tau(y)=\tau'(y)$. Since $x$ is fresh, such $v$ are distinct from $x$ and we
  conclude that $\phi'(y)>\phi'(v)$ for every $v$ that occurs in $\tau'(y)$.
  Moreover, since $x$ does not occur in $T$, it is
  $\phi'(x)=1+\max_{y\in\kappa}\phi(y)>\phi(v)=\phi'(v)$ for all $v$ that occur in $T=\tau'(x)$.
  \qed
\end{proof}

Rule \textsf{T-Assign} writes a value of type $t$ into a leftvalue $\mathsf{w}$.
It performs this by computing $\tau'=\mathsf{write}(\tau,\mathsf{w},t,\mathit{strong})$,
a function that yields the final typing from the typing $\tau$ resulting after the computation of $t$.
By rule induction, $\tau$ is linearizable. In order to show that $\tau'$ is linearizable as well,
we must show how $\mathsf{write}$ is defined:
\[
\mathsf{write}(\tau,\underbrace{\mathtt{*}\cdots\mathtt{*}}_nx,t,\mathit{modality})=\mathsf{apply}
(x,\update(\tau,n,\tau(x),t,\mathit{modality}))
\]
where
\begin{align*}
  \update(\tau,0,t',t,\mathit{strong})&=\langle\tau,t\rangle\\
  \update(\tau,0,t',t,\mathit{weak})&=\langle\tau,t'\sqcup t\rangle\\
  \update(\tau,n+1,\boxtype{t'},t,\mathit{modality})
  &=\mathsf{expand}(\update(\tau,n,t',t,\mathit{modality}))\\
  \update(\tau,n+1,\underbrace{\mutborrow\{\mathsf{w}_1,\ldots,\mathsf{w}_m\}}_b,t,\_)&=
    \langle\bigsqcup\limits_{1\le i\le m}\mathsf{write}(\tau,\underbrace{\mathtt{*}\cdots\mathtt{*}}_n\mathsf{w}_i,t,\mathit{weak}),b\rangle
\end{align*}
and
\begin{align*}
  \mathsf{apply}(y,\langle\tau,t\rangle)&=\tau[y\to t]\\
  \mathsf{expand}(\langle\tau,t\rangle)&=\langle\tau,\boxtype{t}\rangle.
\end{align*}
%
The details of these functions
are complex and we refer the interested reader to~\cite{Pearce21}.
Here, it is important to observe that $\mathsf{write}$
can either completely replace the type of $x$
(\emph{strong} update) or modify the previous type of variables
inside the mutable borrows in $\tau$ (\emph{weak} update).
No other variables can see their type updated.

\begin{lemma}\label{lem:write_invariant}
  Let $\tau$ be a linearizable typing for the context $\kappa$;
  let $\mathsf{w}\in\Leftvalues_\kappa$ and $t\in\mathsf{T}_\kappa$.
  Let $\tau'=\mathsf{write}(\tau,\mathsf{w},t,\mathit{strong})$ be
  the application of function $\mathsf{write}$ in rule \textsf{T-Assign},
  used there to assign the type $t$ to $\mathsf{w}$.
  Then $\tau'$ is linearizable as well.
\end{lemma}
\begin{proof}
  The function $\mathsf{write}$ modifies a set of variables $v_1,\ldots,v_n$ in $\tau$ to compute $\tau'$.
  The type of the other variables remains unchanged between $\tau$ and $\tau'$. Since the
  type system in~\cite{Pearce21} guarantees that borrowed variables are not modified,
  this means that $v_1,\ldots,v_n$ do not occur in the borrows in $\tau$. Moreover,
  the variables in the borrows in $t$ do not contain $v_1,\ldots,v_n$, because
  such variables are either $x=\mathsf{root}(\mathsf{w})$, and the rule \textsf{T-Assign}
  forbids the presence of $x$ in the borrows in $t$ ($\neg\mathsf{writeProhibited}$ in rule
  \textsf{T-Assign}); or they are inside a mutable borrow in $\tau$ (last case of
  $\update$ above), but hence they would be mutably borrowed and the type system
  in~\cite{Pearce21} would have forbidded to read mutably borrowed variables in order to compute
  the type $t$ (see rule \textsf{T-MutBorrow} at page~24 of~\cite{Pearce21}).
  This means that such $v_1,\ldots,v_n$ only occurs in the left-hand side of the bindings of $\tau'$.
  Consider now the function $\phi$ that shows that $\tau$ is linearizable (Def.~\ref{def:linearization}).
  Let us extend $\phi$ into an injective function $\phi'$ that gives $v_1,\ldots,v_n$ the highest ranks:
  \[
  \phi'=\phi\left[\left.v_i\to i+\max\limits_{y\in\kappa\setminus\{v_1,\ldots,v_n\}}\phi(y)\right|1\le i\le n\right].
  \]
  For every $y\in\kappa\setminus\{v_1,\ldots,v_n\}$, it is
  $\phi'(y)=\phi(x)>\phi(v)=\phi'(v)$ for every $v$ that occurs in $\tau(y)=\tau'(y)$.
  Moreover, by construction $\phi'(v_i)>\phi(v)=\phi'(v)$ for every $v$ that occurs in $\tau'(v_i)$.
  That is, $\phi'$ is linearizable as well.
  \qed
\end{proof}
