\section{Semantical Invariant}\label{sec:invariant}

This section proves an invariant of the typing rules
from Sec.~\ref{sec:borrow_checker}. Namely, it proves that
if those rules are applied from a linearizable typing $\tau$,
they can only lead to a linearizable typing $\tau'$.
By Prop.~\ref{prop:termination}, this means that the recursion used for
typing leftvalues in those rules is well-founded,
hence a borrow checker that implements those typing rules terminates
(assuming that it starts from the empty, linearizable typing).
The proof proceeds by rule induction.

Some rules from Sec.~\ref{sec:borrow_checker} obviously satisfy the invariant,
since they do not modify the typing (for them, $\tau=\tau'$).
This is the case of rules
\textsf{T-Const}, \textsf{T-Copy}, \textsf{T-ImmBorrow} and
\textsf{T-MutBorrow}.
Rule \textsf{T-Box} satisfies the invariant by a simple application of
rule induction.

For rule \textsf{T-Move}, it is $\tau'=\mathsf{move}(\mathsf{w},\tau)$.
The intuition is that
$\mathsf{strike}$ can only make the set of variables in the right-hand side
of the typing smaller. Therefore, it can never make $\tau'$
non-linearizable. This is proved below.

\begin{lemma}\label{lem:move_invariant}
  If \textsf{T-Move} is applied from a linearizable typing $\tau$
  and leads to a typing $\tau'$, then also $\tau'$ is linearizable.
\end{lemma}
\begin{proof}
  By definition of $\mathsf{move}$, the only
  difference between $\tau$ and $\tau'$ is at $r=\mathsf{root}(\mathsf{w})$.
  The variables that occur in $\tau'(r)$ are included in those that occur
  in $\tau(r)$ ($\mathsf{strike}$ can only strike away part of the type $\tau(r)$).
  Hence the same function $\phi$ that exists for $\tau$ (Def.~\ref{def:linearization})
  shows that $\tau'$ is linearizable.
  \qed
\end{proof}

Rule \textsf{T-Block} is used at the end of a
block of code, where the set $S$ of local
variables declared in the block goes out of scope. It removes
the type bindings for the variables in $S$ from the initial typing $\tau$,
through function $\mathsf{drop}$.
The invariant for \textsf{T-Block} follows by rule induction and by the following result,
whose intuition is that the removal of bindings from a typing can never make it non-linearizable.
%
\begin{lemma}\label{lem:drop_invariant}
  If \textsf{drop} is applied from a linearizable typing $\tau$
  and leads to a typing $\tau'$, then also $\tau'$ is linearizable.
\end{lemma}
\begin{proof}
  The difference between $\tau$ and $\tau'$ is that $\tau'$ is missing some
  bindings for some variables that have been projected away. Therefore,
  the same function $\phi$ that exists for $\tau$ (Def.~\ref{def:linearization})
  can be used to show that $\tau'$ is linearizable.
  \qed
\end{proof}

Rule \textsf{T-Declare} models the declaration
of a new variable $x$, bound to an expression $\mathsf{e}$. The evaluation of
$\mathsf{e}$ leads to a
typing $\tau'$ that, by rule induction, satisfies the invariant. As a final
step, this rule enlarges $\tau'$ with a binding for $x$. Since $x$ is
fresh ($x\not\in\dom(\tau)$),
variable $x$ does not occur in the right-hand side of that binding. Namely, the rule
leads to a new typing $\tau''=\tau'[x\to T^l]$ where $T^l$ is the type of $\mathsf{e}$, such that
$x$ does not occur in $T$. The invariant for \textsf{T-Declare} follows by the next result.

\begin{lemma}\label{lem:declare_invariant}
  Let $\tau$ be a linearizable typing for the context $\kappa$;
  let $x\not\in\kappa$, $T\in\mathsf{T}_\kappa$ (hence $x$ does not occur in $T$)
  and $l$ be a lifetime.
  Then $\tau'=\tau[x\to T^l]$ is linearizable as well.
\end{lemma}
\begin{proof}
  Consider the function $\phi$ that shows that $\tau$ is linearizable (Def.~\ref{def:linearization}).
  Let us extend $\phi$ into an injective function $\phi'$ that gives $x$ the highest rank:
  \[
  \phi'=\phi\left[x\to 1+\max\limits_{y\in\kappa}\phi(y)\right].
  \]
  Given $y\in\kappa$, it is $\phi'(y)=\phi(y)>\phi(v)$ for every $v$ that occurs
  in $\tau(y)=\tau'(y)$. Since $x$ is fresh, such $v$ are distinct from $x$ and we
  conclude that $\phi'(y)>\phi'(v)$ for every $v$ that occurs in $\tau'(y)$.
  Since $x$ does not occur in $T$, it is
  $\phi'(x)=1+\max_{y\in\kappa}\phi(y)>\phi(v)=\phi'(v)$ for every $v$ that occurs in $T^l=\tau'(x)$.
  \qed
\end{proof}

Rule \textsf{T-Assign} computes the type $T$ of the value of the assigned expression
$\mathsf{e}$, which leads to a typing $\tau'$. By rule induction, $\tau'$ is
linearizable. Then the rule writes that value into a leftvalue $\mathsf{w}$.
It performs this by computing $\tau''=\mathsf{write}(\tau',\mathsf{w},T,\mathit{strong})$.
The following result shows that $\tau''$ is linearizable as well.

\begin{lemma}\label{lem:write_invariant}
  Let $\tau$ be a linearizable typing for the context $\kappa$;
  let $\mathsf{w}\in\Leftvalues_\kappa$ and $T\in\mathsf{T}_\kappa$.
  Let $\tau'=\mathsf{write}(\tau,\mathsf{w},T,\mathit{strong})$ be
  the application of function $\mathsf{write}$ in rule \textsf{T-Assign},
  used there to assign the type $T$ to $\mathsf{w}$.
  Then $\tau'$ is linearizable as well.
\end{lemma}
\begin{proof}
  The function $\mathsf{write}$ modifies a set of variables $v_1,\ldots,v_n$ in $\tau$ to compute $\tau'$.
  The type of the other variables remains unchanged from $\tau$ to $\tau'$. Since the
  type system guarantees that borrowed variables are not modified~\cite{Pea21},
  this means that $v_1,\ldots,v_n$ do not occur in the borrows in $\tau$. Moreover,
  the variables in the borrows in $T$ do not contain $v_1,\ldots,v_n$, because
  such variables are either $x=\mathsf{root}(\mathsf{w})$, and the rule \textsf{T-Assign}
  forbids the presence of $x$ in the borrows in $T$ ($\neg\mathsf{writeProhibited}$ in rule
  \textsf{T-Assign}); or they are inside mutable borrows in $\tau$ (last case of
  $\update$), in which case they would be mutably borrowed and the type system
  would have forbidden to read mutably borrowed variables in order to compute
  the type $T$ (see rule \textsf{T-MutBorrow}).
  This means that such $v_1,\ldots,v_n$ only occur in the left-hand side of the bindings of $\tau'$.
  Consider now the function $\phi$ that shows that $\tau$ is linearizable
  (Def.~\ref{def:linearization}).
  Let us extend $\phi$ into an injective function $\phi'$
  that gives $v_1,\ldots,v_n$ the highest ranks:
  \[
  \phi'=\phi\left[\left.v_i\to i+\max\limits_{y\in\kappa\setminus\{v_1,\ldots,v_n\}}\phi(y)\right|1\le i\le n\right].
  \]
  For every $y\in\kappa\setminus\{v_1,\ldots,v_n\}$, it is
  $\phi'(y)=\phi(x)>\phi(v)=\phi'(v)$ for every $v$ that occurs in $\tau(y)=\tau'(y)$.
  Moreover, by construction, $\phi'(v_i)>\phi(v)=\phi'(v)$ for every $v$ that occurs in $\tau'(v_i)$.
  That is, $\phi'$ is linearizable as well.
  \qed
\end{proof}
