\section{Semantical Invariant}\label{sec:invariant}

This section proves an invariant of the typing rules
from~\cite{Pearce21}. Namely,
if those rules are applied to a typing $\tau$ whose dependencies are acyclical,
they can only lead to a typing $\tau'$ whose dependencies are acyclical as well.
By Prop.~\ref{prop:acyclicity}, this means that the recursion used for
typing leftvalues in those rules is well-founded,
hence the implementation of the typing rules from~\cite{Pearce21} terminates.

We prove the invariant by rule induction.

Some rules from~\cite{Pearce21} obviously keep the dependencies acyclical,
since they do not change the typing. This is the case of rules
\textsf{T-Const}, \textsf{T-Copy}, \textsf{T-MutBorrow} and
\textsf{T-ImmBorrow}.

Other rules keep the dependencies acyclical by rule induction,
such as \textsf{T-Box} and \textsf{T-Seq}.

Rule \textsf{T-Move} is used for typing the evaluation of a leftvalue
whose type has move semantics. This modifies the typing since, in Rust,
the ownership of the leftvalue moves away from itself.
Formally, for this rule it is $\tau'=\mathsf{move}(\tau,\mathsf{w})$
for a suitable leftvalue $\mathsf{w}$,
where the $\mathsf{move}$ function
modifies the binding for the root of $\mathsf{w}$ and is defined as:
\[
\mathsf{move}(\tau,\mathsf{w})=\tau[\mathsf{root}(\mathsf{w})\mapsto
  \mathsf{strike}(\mathsf{w},\tau(\mathsf{root}(\mathsf{w})))]
\]
where
\begin{align*}
  \mathsf{strike}(x,t)&=\mathsf{dangling}\\
  \mathsf{strike}(\mathtt{*}\mathsf{w},\boxtype{t})&=
  \boxtype{\mathsf{strike}(\mathsf{w},t)}.
\end{align*}
The function $\mathsf{strike}$ is undefined otherwise.

\begin{lemma}\label{lem:move_invariant}
  If rule \textsf{T-Move} is applied from a typing $\tau$ whose dependencies
  are acyclical and leads to a typing $\tau'$, then the dependencies
  of $\tau'$ are acyclical as well.
\end{lemma}
\begin{proof}
  Following~\cite{Pearce21}, it is $\tau'=\mathsf{move}(\tau,\mathsf{w})$
  for a suitable
  leftvalue $\mathsf{w}$. Hence typings $\tau$ and $\tau'$ coincide
  on all bindings but for that for $r=\mathsf{root}(\mathsf{w})$, where
  \[
  \tau'(r)=\mathsf{strike}(\mathsf{w},\tau(r)).
  \]
  Let us show that the set of dependencies between leftvalues induced
  by $\tau'$ is included in the set of dependencies between leftvalues
  induced by $\tau$. This follows from two facts:
  \begin{enumerate}
  \item for every type $t$ and leftvalue $\mathsf{w}''$, the $\mathsf{strike}$
    function reduces the dependencies, that is,
    $\mathsf{dependencies}(\mathsf{w}'',\mathsf{strike}(\mathsf{w},t))
    \subseteq\mathsf{dependencies}(\mathsf{w}'',t)$;
  \item for every type $t$, if a borrow occurs in
    $\mathsf{strike}(\mathsf{w},t)$ then it occurs also in $t$.
  \end{enumerate}
  These two facts are proved by induction on $\mathsf{w}$.
  \begin{itemize}
  \item Base case: $\mathsf{w}=x\in\Vars$. Then $\mathsf{strike}(\mathsf{w},t)
    =\mathsf{dangling}$,
    which entails that $\mathsf{dependencies}
    (\mathsf{w}'',\mathsf{strike}(\mathsf{w},t))=\varnothing$ and no borrow
    occurs in $\mathsf{strike}(\mathsf{w},t)$. Both 1 and 2 hold trivially.
  \item Inductive case: $\mathsf{w}=\mathtt{*}\mathsf{w}'$ and assume that
    1 and 2 hold for $\mathsf{w}'$. Then it must be $t=\boxtype{t'}$
    for some type $t'$ and $\mathsf{strike}(\mathsf{w},t)
    =\boxtype{\mathsf{strike}(\mathsf{w}',t')}$. Hence
    \begin{align*}
      \mathsf{dependencies}(\mathsf{w}'',\mathsf{strike}(\mathsf{w},t))&=
      \mathsf{dependencies}(\mathsf{w}'',\boxtype{\mathsf{strike}(\mathsf{w}',t')})\\
      &=\mathsf{dependencies}(\mathtt{*}\mathsf{w}'',\mathsf{strike}(\mathsf{w}',t'))\\
      \text{(by inductive hypothesis)}&\subseteq
      \mathsf{dependencies}(\mathtt{*}\mathsf{w}'',t')\\
      &=\mathsf{dependencies}(\mathsf{w}'',\boxtype{t'})\\
      &=\mathsf{dependencies}(\mathsf{w}'',t)
    \end{align*}
    hence 1 holds for $\mathsf{w}$. Moreover, if a borrow occurs in
    $\mathsf{strike}(\mathsf{w},t)$ then it must occurs
    in $\boxtype{\mathsf{strike}(\mathsf{w}',t')}$, that is, it must occur in
    $\mathsf{strike}(\mathsf{w}',t')$. By inductive hypothesis, the borrow
    occurs in $t'$ and hence in $t=\boxtype{t'}$. Therefore, 2 holds for
    $\mathsf{w}$ as well.
  \end{itemize}
  \qed
\end{proof}

The \textsf{T-Block} rule is used at the end of a block of code, where the set $S$ of local
variables declared in the block goes out of scope. As a consequence, this rule removes
the type bindings for those variables in $S$ from the initial typing $\tau$. It performs this
through a function $\mathsf{drop}$ that projects away those variables from $\tau$ and is defined as:
\[
\mathsf{drop}(\tau,S)=\tau|_{-S}.
\]
The invariant for rule \textsf{T-Block} follows by rule induction and by the following result:
%
\begin{lemma}\label{lem:drop_invariant}
  If \textsf{drop} is applied from a typing $\tau$ whose dependencies
  are acyclical and leads to a typing $\tau'$, then the dependencies
  of $\tau'$ are acyclical as well.
\end{lemma}
\begin{proof}
  By definition, it is $\tau'=\tau|_{-S}$. Therefore, $\tau'$ consists of a subset of the
  type bindings in $\tau$: it is a typing for a context $\kappa'$ while $\tau$ is a
  typing for a context $\kappa$, with $\kappa'\subseteq\kappa$, and they coincide
  on $\kappa'$. As a consequence,
  \[
  \bigcup\limits_{x\in\kappa'}\mathsf{dependencies}(x,\tau'(x))\subseteq
  \bigcup\limits_{x\in\kappa}\mathsf{dependencies}(x,\tau(x)).
  \]
  Moreover, if a borrow occurs in $\tau'$, it also occurs in $\tau$. It follows
  that the dependencies induced by $\tau'$ are included in those induced by $\tau$
  and if the latter are acyclical, also the former must be acyclical.
  \qed
\end{proof}

The rule \textsf{T-Declare} models the declaration of a new variable $x$,
bound to a term $t$. The evaluation of $t$ leads to a typing $\tau$ that, by rule induction,
satisfies the invariant. As a final step, this rule
enlarges $\tau$ with a binding for $x$. Since $x$ is
fresh (that is, $x\not\in\dom(\tau)$),
variable $x$ does not occur in the right-hand side of that binding. Namely, the rule
leads to a new typing $\tau'=\tau[x\to T]$ where $T$ is the type of $t$, such that
$x$ does not occur in $T$. The invariant for rule \textsf{T-Declare} follows by the next result.

\begin{lemma}\label{lem:declare_invariant}
  Let $\tau$ be a typing for the context $\kappa$ whose dependencies
  are acyclical; let $x\not\in\kappa$ and $T\in\mathsf{T}_\kappa$ (hence $x$ does not occur in $T$).
  Then the dependencies of $\tau'=\tau[x\to T]$ are acyclical as well.
\end{lemma}
\begin{proof}
  Let $\kappa'=\kappa\cup\{x\}$.
  Let us compare the set
  \[
  R_\kappa=\{\mathtt{*}\mathsf{w}\gg\mathsf{w}\mid\mathsf{w}\in\Leftvalues_\kappa\}
  \cup\bigcup\limits_{y\in\kappa}\mathsf{dependencies}(y,\tau(y))
  \]
  with the set
  \begin{align*}
  R_{\kappa'}&=\{\mathtt{*}\mathsf{w}\gg\mathsf{w}\mid\mathsf{w}\in\Leftvalues_{\kappa'}\}
  \cup\bigcup\limits_{y\in\kappa'}\mathsf{dependencies}(y,\tau'(y))\\
  &=\{\mathtt{*}\mathsf{w}\gg\mathsf{w}\mid\mathsf{w}\in\Leftvalues_\kappa\}
  \cup\{\mathtt{*}\mathsf{w}\gg\mathsf{w}\mid\mathsf{w}\in\Leftvalues_{\kappa'}\setminus
  \Leftvalues_{\kappa}\}\\
  &\qquad\cup\bigcup\limits_{y\in\kappa}\mathsf{dependencies}(y,\tau(y))\\
  &\qquad\cup\mathsf{dependencies}(x,\tau'(x))\\
  &=R_\kappa\cup\underbrace{\{\mathtt{*}\mathsf{w}\gg\mathsf{w}\mid\mathsf{w}\in\Leftvalues_{\kappa'}\setminus
  \Leftvalues_{\kappa}\}}_A\cup\underbrace{\mathsf{dependencies}(x,T)}_B\\
  \end{align*}
  The difference are the sets of dependencies $A$ and $B$. By construction:
  \begin{enumerate}
  \item the dependencies in
    $A$ have the form $\underbrace{\mathtt{*}\cdots\mathtt{*}}_{\ge 1}\mathsf{w}\gg\mathsf{w}$ where $x$ occurs in $\mathsf{w}$ and hence
    also in $\underbrace{\mathtt{*}\cdots\mathtt{*}}_{\ge 1}\mathsf{w}$;
  \item the dependencies in $B$ have the form
    $\underbrace{\mathtt{*}\cdots\mathtt{*}}_{\ge 1}x\gg\mathsf{w}$ where $x$ does not occur
    in $\mathsf{w}$.
  \end{enumerate}
  In particular, note that there is no dependency in $A\cup B$ such that
  $x$ occurs in its right-hand side but not in its left-hand side.

  Let us compute the closure of $R_{\kappa'}$ now. For that, it is useful to introduce the notation
  \[
  C_\tau(R_1,R_2)=\left\{\underbrace{\mathtt{*}\cdots\mathtt{*}}_{n}\mathsf{w}_1\gg\mathsf{w}_3\left|
  \begin{array}{l}
    \mathsf{w}_1\gg\mathsf{w}_2\in R_1,\ \underbrace{\mathtt{*}\cdots\mathtt{*}}_{n\ge 0}\mathsf{w}_2\gg\mathsf{w}_3\in R_2\\
    \text{and $\mathsf{w_3}$ is in a borrow that occurs in $\tau$}
  \end{array}\right.\right\}.
  \]
  It is
  \begin{align*}
    \mathsf{closure}(R_{\kappa'})&=R_{\kappa}\cup A\cup B\cup C_{\tau'}(R_\kappa,R_\kappa)\\
    &\qquad\cup C_{\tau'}(R_k,A\cup B)\cup C_{\tau'}(A\cup B, R_\kappa)\cup C_{\tau'}(A\cup B,A\cup B)\\
    &=R_{\kappa}\cup A\cup B\cup C_{\tau}(R_\kappa,R_\kappa)\cup C_{[x\to T]}(R_\kappa,R_\kappa)\\
    &\qquad\cup C_{\tau'}(R_k,A\cup B)\cup C_{\tau'}(A\cup B, R_\kappa)\cup C_{\tau'}(A\cup B,A\cup B)\\
    &=\mathsf{closure}(R_\kappa)\cup A\cup B\cup C_{[x\to T]}(R_\kappa,R_\kappa)\\
    &\qquad\cup C_{\tau'}(R_k,A\cup B)\cup C_{\tau'}(A\cup B, R_\kappa)\cup C_{\tau'}(A\cup B,A\cup B)
  \end{align*}
  Since all dependencies in $A\cup B$ contain $x$ on their left-hand side, that does not
  occur in $R_\kappa$ instead, it is $C_{\tau'}(R_\kappa,A\cup B)=\varnothing$.
  Since all dependencies in $A$ contain $x$ on their right-hand side, that does not
  occur in $R_\kappa$ instead, it is $C_{\tau'}(A\cup B,R_\kappa)=C_{\tau'}(B,R_\kappa)$.
  If a leftvalue occurs in a borrow in $T$, either that same leftvalue occured in a borrow
  of $\tau$ as well, or otherwise it is never in the right-hand side of the dependencies
  in $R_\kappa$. It follows that $C_{[x\to T]}(R_\kappa,R_\kappa)\subseteq\mathsf{closure}(R_\kappa)$.
  Therefore
  \[
    \mathsf{closure}(R_{\kappa'})=\mathsf{closure}(R_\kappa)
    \cup A\cup B\cup C_{\tau'}(B, R_\kappa)\cup C_{\tau'}(A\cup B,A\cup B).
  \]
  Since $A$ and $B$ satisfy the conditions 1 and 2 reported above, respectively, it follows that
  also the dependencies in $C_\tau'(A\cup B,A\cup B)$ satisfy either 1 or 2. Moreover,
  the dependencies in $C_{\tau'}(B,R_\kappa)$ satisfy condition 2. We conclude that the
  dependencies in $\mathit{diff}=\mathsf{closure}(R_{\kappa'})\setminus\mathsf{closure}(R_\kappa)$
  satisfy either condition 1 or condition 2 above.

  Suppose now, by contradiction, that the dependencies in
  $\mathsf{closure}(R_{\kappa'})$ be cyclical. Hence there is a cycle of dependencies
  from $\mathsf{closure}(R_{\kappa'})$: $\mathsf{w}_0\gg\mathsf{w}_1\gg\mathsf{w}_2\gg\cdots\gg
  \mathsf{w}_{k-1}\gg\mathsf{w}_k\gg\mathsf{w}_0$. Such dependencies
  cannot be all in $\mathsf{closure}(R_\kappa)$ since, by hypothesis, the dependencies
  induced by $\tau$ are acyclical. Therefore, there is at least one such dependency
  that belongs to $\mathit{diff}$. Namely, without lost of generality, we can assume that
  $\mathsf{w}_k\gg\mathsf{w}_0\in\mathit{diff}$. As shown above, either 1 or 2 holds for
  $\mathsf{w}_k\gg\mathsf{w}_0$ and we have two corresponding cases:
  \begin{itemize}
  \item $\mathsf{w}_k=\underbrace{\mathtt{*}\cdots\mathtt{*}}_{\ge 1}\mathsf{w}_0$
    and $x$ occurs in $\mathsf{w}_\kappa$. Since $x$ does not occur in the dependencies
    in $\mathsf{closure}(R_\kappa)$, we conclude that also
    $\mathsf{w}_{k-1}\gg\mathsf{w}_k\in\mathit{diff}$. Since $x$ occurs in $\mathsf{w}_k$,
    it follows that 1 must hold for $\mathsf{w}_{k-1}\gg\mathsf{w}_k$ as well and hence
    $\mathsf{w}_{k-1}=\underbrace{\mathtt{*}\cdots\mathtt{*}}_{\ge 1}\mathsf{w}_k=
    \underbrace{\mathtt{*}\cdots\mathtt{*}}_{\ge 2}\mathsf{w}_0$ and $x$
    occurs in $\mathsf{w}_{k-1}$. We can continue with this reasoning for the whole cycle, until
    we reach its starting leftvalue $\mathsf{w}_0$ and conclude that
    $\mathsf{w}_0=\underbrace{\mathtt{*}\cdots\mathtt{*}}_{\ge k+1}\mathsf{w}_0$, which is
    impossible.
  \item $\mathsf{w}_k=\underbrace{\mathtt{*}\cdots\mathtt{*}}_{\ge 1}x$ and $x$ does not occur in
    $\mathsf{w}_0$. Since $x$ occurs in $\mathsf{w}_\kappa$
    and $x$ does not occur in the dependencies
    in $\mathsf{closure}(R_\kappa)$, we conclude that also
    $\mathsf{w}_{k-1}\gg\mathsf{w}_k\in\mathit{diff}$.
    Since $x$ occurs in $\mathsf{w}_k$,
    it follows that 1 must hold for $\mathsf{w}_{k-1}\gg\mathsf{w}_k$ and hence
    $x$ occurs in $\mathsf{w}_{k-1}$ as well. We can continue with
    this reasoning for the whole cycle, until
    we reach its starting leftvalue $\mathsf{w}_0$ and conclude that $x$ occurs in $\mathsf{w}_0$,
    which is impossible.
  \end{itemize}
  \qed
\end{proof}
