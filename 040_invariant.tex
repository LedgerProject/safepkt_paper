\section{Semantical Invariant}\label{sec:invariant}

This section proves an invariant of the typing rules
from~\cite{Pearce21}. Namely,
if those rules are applied to a typing $\tau$ hose dependencies are acyclical,
they can only lead to a typing $\tau'$ whose dependencies are acyclical as well.
By Prop.~\ref{prop:acyclicity}, this means that the recursion used for
typing leftvalues in those rules is well-founded,
hence the implementation of the typing rules from~\cite{Pearce21} terminates.

We prove the invariant by rule induction.

Some rules from~\cite{Pearce21} obviously keep the dependencies acyclical,
since they do not change the typing. This is the case of rules
\textsf{T-Const}, \textsf{T-Copy}, \textsf{T-MutBorrow} and
\textsf{T-ImmBorrow}.

Other rules keep the dependencies acyclical by rule induction,
such as \textsf{T-Box} and \textsf{T-Seq}.

Rule \textsf{T-Move} is used for typing the evaluation of a leftvalue
whose type has move semantics. This modifies the typing since, in Rust,
the ownership of the leftvalue moves away from itself.
Formally, for this rule it is $\tau'=\mathsf{move}(\tau,\mathsf{w})$
for a suitable leftvalue $\mathsf{w}$,
where the $\mathsf{move}$ function
modifies the binding for the root of $\mathsf{w}$ and is defined as:
\[
\mathsf{move}(\tau,\mathsf{w})=\tau[\mathsf{root}(\mathsf{w})\mapsto
  \mathsf{strike}(\mathsf{w},\tau(\mathsf{root}(\mathsf{w})))]
\]
where
\begin{align*}
  \mathsf{strike}(x,t)&=\mathsf{dangling}\\
  \mathsf{strike}(\mathtt{*}\mathsf{w},\boxtype{t})&=
  \boxtype{\mathsf{strike}(\mathsf{w},t)}.
\end{align*}
The function $\mathsf{strike}$ is undefined otherwise.

\begin{lemma}\label{lem:move_imvariant}
  If rule \textsf{T-Move} is applied from a typing $\tau$ whose dependencies
  are acyclical and leads to a typing $\tau'$, then the dependencies
  of $\tau'$ are acyclical as well.
\end{lemma}
\begin{proof}
  Following~\cite{Pearce21}, it is $\tau'=\mathsf{move}(\tau,\mathsf{w})$
  for a suitable
  leftvalue $\mathsf{w}$. Hence typings $\tau$ and $\tau'$ coincide
  on all bindings but for that for $r=\mathsf{root}(\mathsf{w})$, where
  \[
  \tau'(r)=\mathsf{strike}(\mathsf{w},\tau(r)).
  \]
  Let us show that the set of dependencies between leftvalues induced
  by $\tau'$ is included in the set of dependencies between leftvalues
  induced by $\tau$. This follows from two facts:
  \begin{enumerate}
  \item for every type $t$ and leftvalue $\mathsf{w}''$, the $\mathsf{strike}$
    function reduces the dependencies, that is,
    $\mathsf{dependencies}(\mathsf{w}'',\mathsf{strike}(\mathsf{w},t))
    \subseteq\mathsf{dependencies}(\mathsf{w}'',t)$;
  \item for every type $t$, if a borrow occurs in
    $\mathsf{strike}(\mathsf{w},t)$ then it occurs also in $t$.
  \end{enumerate}
  These two facts are proved by induction on $\mathsf{w}$.
  \begin{itemize}
  \item Base case: $\mathsf{w}=x\in\Vars$. Then $\mathsf{strike}(\mathsf{w},t)
    =\mathsf{dangling}$,
    which entails that $\mathsf{dependencies}
    (\mathsf{w}'',\mathsf{strike}(\mathsf{w},t))=\varnothing$ and no borrow
    occurs in $\mathsf{strike}(\mathsf{w},t)$. Both 1 and 2 hold trivially.
  \item Inductive case: $\mathsf{w}=\mathtt{*}\mathsf{w}'$ and assume that
    1 and 2 hold for $\mathsf{w}'$. Then it must be $t=\boxtype{t'}$
    for some type $t'$ and $\mathsf{strike}(\mathsf{w},t)
    =\boxtype{\mathsf{strike}(\mathsf{w}',t')}$. Hence
    \begin{align*}
      \mathsf{dependencies}(\mathsf{w}'',\mathsf{strike}(\mathsf{w},t))&=
      \mathsf{dependencies}(\mathsf{w}'',\boxtype{\mathsf{strike}(\mathsf{w}',t')})\\
      &=\mathsf{dependencies}(\mathtt{*}\mathsf{w}'',\mathsf{strike}(\mathsf{w}',t'))\\
      \text{(by inductive hypothesis)}&\subseteq
      \mathsf{dependencies}(\mathtt{*}\mathsf{w}'',t')\\
      &=\mathsf{dependencies}(\mathsf{w}'',\boxtype{t'})\\
      &=\mathsf{dependencies}(\mathsf{w}'',t)
    \end{align*}
    hence 1 holds for $\mathsf{w}$. Moreover, if a borrow occurs in
    $\mathsf{strike}(\mathsf{w},t)$ then it must occurs
    in $\boxtype{\mathsf{strike}(\mathsf{w}',t')}$, that is, it must occur in
    $\mathsf{strike}(\mathsf{w}',t')$. By inductive hypothesis, the borrow
    occurs in $t'$ and hence in $t=\boxtype{t'}$. Therefore, 2 holds for
    $\mathsf{w}$ as well.
  \end{itemize}
  \qed
\end{proof}
